\scsectionfamily{Часть 5 Стандарта OSTIS. Методы и средства проектирования интеллектуальных компьютерных систем нового поколения}
\label{part_methods_and_tools}

\scsection[
    \protect\scneditors{Шункевич Д.В.;Орлов М.К.}
    \protect\scnmonographychapter{Глава 5.1. Комплексная библиотека многократно используемых семантически совместимых компонентов интеллектуальных компьютерных систем нового поколения}
    ]{Предметная область и онтология комплексной библиотеки многократно используемых семантически совместимых компонентов ostis-систем}
\label{sd_biblio_component}
\begin{SCn}
    \scnsectionheader{Предметная область и онтология комплексной библиотеки многократно используемых семантически совместимых компонентов ostis-систем}
    \begin{scnsubstruct}
        \scntext{эпиграф}{Всё, что можно сделать одинаково, нужно делать одинаково.}
        \scntext{аннотация}{Важнейшим этапом эволюции любой технологии является переход к компонентному проектированию на основе постоянно пополняемой библиотеки многократно используемых компонентов. Идея библиотеки компонентов не нова, но семантическая мощность \textbf{\textit{Библиотеки Экосистемы OSTIS}} значительно выше аналогов за счет того, что подавляющее большинство компонентов библиотеки --- компоненты \textit{базы знаний}, представленные на унифицированном языке смыслового представления знаний (\textit{SC-коде}). Таким образом, в Библиотеке Экосистемы OSTIS обеспечивается высокий уровень семантической совместимости компонентов, что приводит к высокому уровню семантической совместимости \textit{ostis-систем}, использующих комплексную библиотеку многократно используемых семантически совместимых компонентов ostis-систем.}
       
       \begin{scnreltovector}{конкатенация сегментов}
       	\scnitem{Сегмент. Введение в Предметную область и онтологию комплексной библиотеки многократно используемых семантически совместимых компонентов ostis-систем}
       	\begin{scnindent}
       		\scnidtf{Сегмент. Введение в компонентное проектирование интеллектуальных компьютерных систем}
       	\end{scnindent}
       	\scnitem{Сегмент. Анализ библиотек многократно используемых компонентов}
       	\scnitem{Сегмент. Понятие библиотеки многократно используемых компонентов ostis-систем}
       	\scnitem{Сегмент. Понятие многократно используемого компонента ostis-систем}
       	\scnitem{Сегмент. Уточнение спецификации многократно используемого компонента ostis-систем}
       	\scnitem{Сегмент. Понятие менеджера многократно используемых компонентов ostis-систем}
       	\scnitem{Сегмент. Заключение в Предметную область и онтологию комплексной библиотеки многократно используемых семантически совместимых компонентов ostis-систем}
       	\begin{scnindent}
       		\scnidtf{Сегмент. Заключение в компонентное проектирование интеллектуальных компьютерных систем}
       	\end{scnindent}
       \end{scnreltovector}
       
       \begin{scnrelfromlist}{библиография}
	       	\scnitem{\scncite{Zhitko2011a}}
	       	\scnitem{\scncite{Zalivako2012}}
	       	\scnitem{\scncite{Borisov2014}}
	       	\scnitem{\scncite{Golenkov2015}}
	       	\scnitem{\scncite{Zhitko2011b}}
	       	\scnitem{\scncite{Golenkov2013}}
	       	\scnitem{\scncite{Shunkevich2015a}}
	       	\scnitem{\scncite{Zalivako2011}}
	       	\scnitem{\scncite{Golenkov2014a}}
	       	\scnitem{\scncite{Iyengar2021}}
	       	\scnitem{\scncite{Ford2019}}
	       	\scnitem{\scncite{Wilkes1951}}
	       	\scnitem{\scncite{Wilkes1953}}
	       	\scnitem{\scncite{Volchenskova1984}}
	       	\scnitem{\scncite{Blahser2021}}
	       	\scnitem{\scncite{Hepp2008}}
	       	\scnitem{\scncite{Memduhoglu2018}}
	       	\scnitem{\scncite{Fritzson2014}}
	       	\scnitem{\scncite{Prakash2022}}
	       	\scnitem{\scncite{Moskalenko2016}}
	       	\scnitem{\scncite{Pivovarchik2015}}
	       	\scnitem{\scncite{Koronchik2011}}
	       	\scnitem{\scncite{Davydenko2013}}
	       	\scnitem{\scncite{Ivashenko2013}}
	       	\scnitem{\scncite{Golenkov2014b}}
	       	\scnitem{\scncite{Orlov2022a}}
	       	\scnitem{\scncite{Golenkov2011}}
	       	\scnitem{\scncite{Ivashenko2011}}
	       	\scnitem{\scncite{Lazyrkin2011}}
	       	\scnitem{\scncite{Davydenko2011}}
	       	\scnitem{\scncite{Koronchik2012}}
	       	\scnitem{\scncite{Shunkevich2013}}
	       	\scnitem{\scncite{Koronchik2013}}
	       	\scnitem{\scncite{Eliseeva2013}}
	       	\scnitem{\scncite{Shunkevich2015}}
	       	\scnitem{\scncite{Davydenko2017}}
	       	\scnitem{\scncite{Davydenko2018}}
	       	\scnitem{\scncite{Tu1995}}
	       	\scnitem{\scncite{Studer1996}}
	       	\scnitem{\scncite{Benjamins1999}}
       \end{scnrelfromlist}
       
        \scnheader{Предметная область многократно используемых компонентов ostis-систем}
        \scniselement{предметная область}
        \begin{scnhaselementrolelist}{максимальный класс объектов исследования}
        	\scnitem{библиотека многократно используемых компонентов ostis-систем}
        	\scnitem{многократно используемый компонент ostis-систем}
        \end{scnhaselementrolelist}
        \begin{scnhaselementrolelist}{класс объектов исследования}
            \scnitem{многократно используемый компонент базы знаний}
            \scnitem{многократно используемый компонент решателя задач}
            \scnitem{многократно используемый компонент пользовательского интерфейса}
            \scnitem{атомарный многократно используемый компонент ostis-систем}
            \scnitem{неатомарный многократно используемый компонент ostis-систем}
            \scnitem{зависимый многократно используемый компонент ostis-систем}
            \scnitem{независимый многократно используемый компонент ostis-систем}
            \scnitem{платформенно-независимый многократно используемый компонент ostis-систем}
            \scnitem{платформенно-зависимый многократно используемый компонент ostis-систем}
            \scnitem{многократно используемый компонент ostis-систем, хранящийся в виде внешних файлов}
            \scnitem{многократно используемый компонент ostis-систем, хранящийся в виде файлов исходных текстов}
            \scnitem{многократно используемый компонент ostis-систем, хранящийся в виде скомпилированных файлов}
            \scnitem{многократно используемый компонент, хранящийся в виде sc-структуры}
            \scnitem{типовая подсистема ostis-систем;платформа интерпретации sc-моделей компьютерных систем}
            \scnitem{динамически устанавливаемый многократно используемый компонент ostis-систем}
            \scnitem{многократно используемый компонент, при установке которого система требует перезапуска}
            \scnitem{хранилище многократно используемого компонента ostis-систем, хранящегося в виде внешних файлов}
            \scnitem{хранилище многократно используемого компонента ostis-систем, хранящегося в виде файлов исходных текстов}
            \scnitem{хранилище многократно используемого компонента ostis-систем, хранящегося в виде скомпилированных файлов}
            \scnitem{спецификация многократно используемого компонента ostis-систем}
            \scnitem{отношение, специфицирующее многократно используемый компонент ostis-систем}
            \scnitem{параметр, заданный на многократно используемых компонентах ostis-систем\scnsupergroupsign}
            \scnitem{файл, содержащий url-адрес многократно используемого компонента ostis-систем}
            \scnitem{менеджер многократно используемых компонентов ostis-систем}
        \end{scnhaselementrolelist}
        \begin{scnhaselementrolelist}{исследуемое отношение}
            \scnitem{метод установки*}
            \scnitem{адрес хранилища*}
            \scnitem{зависимости компонента*}
            \scnitem{установленные компоненты*}
        \end{scnhaselementrolelist}
        \begin{scnhaselementrolelist}{исследуемый параметр}
            \scnitem{класс многократно используемого компонента ostis-систем\scnsupergroupsign}
        \end{scnhaselementrolelist}
        \begin{scnhaselementrolelist}{отношение, используемое в предметной области}
            \scnitem{автор*}
            \scnitem{ключевой sc-элемент*}
            \scnitem{пояснение*}
            \scnitem{sc-идентификатор*}
            \scnitem{история изменений*}
        \end{scnhaselementrolelist}

\scnheader{семантически смежный раздел*}
\begin{scnhaselementset}
	\scnitem{Предметная область и онтология комплексной библиотеки многократно используемых семантически совместимых компонентов ostis-систем}
	\scnitem{Логико-семантическая модель Метасистемы OSTIS}
\end{scnhaselementset}
\begin{scnindent}
	\scntext{пояснение}{\textit{Метасистема OSTIS} ориентирована на разработку и практическое внедрение методов и средств компонентного проектирования семантически совместимых интеллектуальных систем, которая предоставляет возможность быстрого создания интеллектуальных приложений различного назначения. В состав Метасистемы OSTIS входит \scnkeyword{Библиотека Метасистемы OSTIS}. Сферы практического применения технологии компонентного проектирования семантически совместимых интеллектуальных систем ничем не ограничены.}
	\scntext{пояснение}{Основу для реализации компонентного подхода в рамках \textit{Технологии OSTIS} составляет \scnkeyword{Библиотека Метасистемы OSTIS}.}
\end{scnindent}
\scnheader{семантически смежный раздел*}
\begin{scnhaselementset}
	\scnitem{Предметная область и онтология комплексной библиотеки многократно используемых семантически совместимых компонентов ostis-систем}
	\scnitem{Предметная область и онтология комплексной технологии поддержки жизненного цикла интеллектуальных компьютерных систем нового поколения}
\end{scnhaselementset}
\begin{scnindent}
	\scntext{пояснение}{Основным требованием, предъявляемым к \textit{Технологии OSTIS}, является обеспечение возможности совместного использования в рамках ostis-систем различных \textit{видов знаний} и различных \textit{моделей решения задач} с возможностью \uline{неограниченного} расширения перечня используемых в ostis-системе видов знаний и моделей решения задач без существенных трудозатрат. Следствием данного требования является необходимость реализации компонентного подхода на всех уровнях, от простых компонентов баз знаний и решателей задач до целых встраиваемых ostis-систем.}
\end{scnindent}
\scnheader{дочерний раздел*}
\begin{scnhaselementvector}
	\scnitem{Предметная область и онтология комплексной библиотеки многократно используемых семантически совместимых компонентов ostis-систем}
	\scnitem{Предметная область и онтология многократно используемых компонентов баз знаний ostis-систем}
\end{scnhaselementvector}
\begin{scnindent}
	\scntext{пояснение}{На сегодняшний день разработано большое число \textit{баз знаний} по самым различным предметным областям. Однако в большинстве случаев каждая база знаний разрабатывается отдельно и независимо от других, вследствие отсутствия единой унифицированной формальной основы для представления знаний, а также единых принципов формирования систем понятий для описываемой предметной области. В связи с этим разработанные базы оказываются, как правило, несовместимы между собой и непригодны для повторного использования. Компонентный подход к разработке интеллектуальных компьютерных систем, реализуемый в виде \scnkeyword{библиотеки многократно используемых компонентов ostis-систем}, позволяет решить описанные проблемы.}
\end{scnindent}
\scnheader{дочерний раздел*}
\begin{scnhaselementvector}
	\scnitem{Предметная область и онтология комплексной библиотеки многократно используемых семантически совместимых компонентов ostis-систем}
	\scnitem{Предметная область и онтология многократно используемых компонентов решателей задач ostis-систем}
\end{scnhaselementvector}
\begin{scnindent}
	\scntext{пояснение}{В области разработки \textit{решателей задач} существует большое количество конкретных реализаций, однако вопросы совместимости различных решателей при решении одной задачи практически не рассматриваются.}
\end{scnindent}
\scnheader{дочерний раздел*}
\begin{scnhaselementvector}
	\scnitem{Предметная область и онтология комплексной библиотеки многократно используемых семантически совместимых компонентов ostis-систем}
	\scnitem{Предметная область и онтология многократно используемых компонентов интерфейсов ostis-систем}
\end{scnhaselementvector}

\begin{SCn}
	\scnsectionheader{Сегмент. Введение в Предметную область и онтологию комплексной библиотеки многократно используемых семантически совместимых компонентов ostis-систем}

	\begin{scnsubstruct}
	
	\scnheader{компонентное проектирование интеллектуальных компьютерных систем}
	\begin{scnrelfromset}{основные положения}
		\scnfileitem{Важнейшим этапом эволюции любой технологии является переход к компонентному проектированию на основе постоянно пополняемый библиотеки многократно используемых компонентов.}
		\begin{scnindent}
			\begin{scnrelfromset}{необходимые требования}
				\scnfileitem{Универсальный язык представления знаний.}
				\scnfileitem{Универсальная процедура интеграции знаний в рамках указанного языка.}
				\scnfileitem{Разработка стандарта, обеспечивающего семантическую совместимость интегрируемых знаний (таким стандартом является согласованная система используемых понятий).}
			\end{scnrelfromset}
		\end{scnindent}
		\scnfileitem{Повторное использование готовых компонентов широко применяется во многих отраслях, связанных с проектированием различного рода систем, поскольку позволяет уменьшить трудоемкость разработки и ее стоимость (путем минимизации количества труда за счет отсутствия необходимости разрабатывать какой-либо компонент), повысить качество создаваемого контента и снизить профессиональные требования к разработчикам компьютерных систем. Таким образом, осуществляется  переход от программирования компонентов или целых систем к их проектированию (дизайну, сборке) на основе готовых компонентов. \textbf{\textit{компонентное проектирование интеллектуальных компьютерных систем}} предполагает подбор существующих компонентов, способных решить поставленную задачу целиком или декомпозицию задачи на подзадачи с выделением компонентов для каждой из них.}
		\begin{scnindent}
			\begin{scnrelfromlist}{источник}
				\scnitem{\cite{Zhitko2011b}}
				\scnitem{\cite{Zalivako2012}}
				\scnitem{\cite{Borisov2014}}
			\end{scnrelfromlist}
		\end{scnindent}
	\end{scnrelfromset}
	\scntext{назначение}{Позволяет уменьшить трудоемкость создания компьютерных систем и их стоимость (путем минимизации количества труда за счет отсутствия необходимости разрабатывать какой-либо компонент), повысить качество создаваемых компьютерных систем и снизить профессиональные требования к разработчикам этих систем.}      
	\scntext{пояснение}{Компонентное проектирование интеллектуальных компьютерных систем предполагает подбор существующих компонентов, способных решить поставленную задачу целиком или декомпозицию задачи на подзадачи с выделением компонентов для каждой из них.}  
	\scntext{преимущество}{Проектируемые системы по предлагаемой технологии обладают высоким уровнем гибкости, их разработка осуществляется поэтапно, переходя от одной целостной версии системы к другой. При этом стартовая версия системы может быть ядром соответствующего класса систем, входящим в библиотеку многократно используемых компонентов.}
	\scnheader{технология компонентного проектирования интеллектуальных компьютерных систем}		
	\scnhaselementrole{главный ключевой sc-элемент}{библиотека совместимых многократно используемых компонентов}
	\scntext{преимущество}{Позволяет проектировать интеллектуальные системы, комбинируя уже существующие компоненты, выбирая нужные из соответствующих библиотек. Использование готовых компонентов предполагает, что распространяемый компонент верифицирован и документирован, а возможные ошибки и ограничения устранены либо специфицированы и известны. Создание \textit{библиотеки многократно используемых компонентов} не означает пересоздание всех уже существующих современных продуктов информационных технологий. Технология компонентного проектирования интеллектуальных компьютерных систем предполагает использование огромного опыта в разработке современных компьютерных систем, однако обязательным является \uline{спецификация} каждого компонента (как вновь созданного, так и интегрируемого существующего) для обеспечения возможности его установки и совместимости с другими компонентами и системами. Тем не менее эффективная технология компонентного проектирования появится только тогда, когда сформируется \scnqq{критическая масса} разработчиков прикладных систем, участвующих в пополнении \textit{библиотек многократно используемых компонентов} проектируемых систем.}
	\begin{scnindent}
		\begin{scnrelfromlist}{источник}
			\scnitem{\cite{Zhitko2011b}}
			\scnitem{\cite{Golenkov2013}}
		\end{scnrelfromlist}
	\end{scnindent}
	\scnrelfrom{проблемы текущего состояния}{Проблемы в реализации компонентного проектирования интеллектуальных компьютерных систем}
	\begin{scnindent}
		\scntext{примечание}{Проблемы реализации компонентного подхода к проектированию интеллектуальных компьютерных систем наследуют проблемы современных \textit{технологий проектирования интеллектуальных систем}.}
		\begin{scneqtoset}
			\scnfileitem{\uline{Несовместимость} компонентов, разработанных в рамках разных проектов, вследствие отсутствия унификации в принципах представления различных видов знаний в рамках одной \textit{базы знаний}, и, как следствие, отсутствие унификации в принципах выделения и спецификации \textbf{\textit{многократно используемых компонентов}}, которое приводит к несовместимости компонентов, разработанных в рамках разных проектов.}
			\begin{scnindent}
				\scntext{примечание}{Большинство существующих систем создано как автономные программные продукты, которые не могут быть использованы в качестве компонентов других систем. Необходимо использовать либо целую систему, либо ничего. Небольшое число систем поддерживает компонентно-ориентированную архитектуру способную интегрироваться с другими системами. Однако их интеграция возможна при условии использования одинаковых технологий и только при проектировании одной командой разработчиков.}
				\begin{scnindent}
					\begin{scnrelfromlist}{источник}
						\scnitem{\cite{Iyengar2021}}
						\scnitem{\cite{Ford2019}}
					\end{scnrelfromlist}
				\end{scnindent}
				\scntext{примечание}{Многократная повторная разработка уже имеющихся технических решений обусловлена либо тем, что известные технические решения \uline{плохо} интегрируются в разрабатываемую систему, либо тем, что эти технические решения трудно найти. Данная проблема актуальна как в целом в сфере разработки компьютерных систем, так и в сфере разработки систем, основанных на знаниях, поскольку в системах такого рода степень согласованности различных видов знаний влияет на возможность системы решать нетривиальные задачи.}  
			\end{scnindent}
			\scnfileitem{Невозможность автоматической интеграции компонентов в систему \uline{без} ручного вмешательства пользователя.}
			\scnfileitem{Автоматическое обновление компонентов приводит к рассогласованности как отдельных модулей компьютерных систем, так и самих систем между собой.}
			\scnfileitem{Отсутствие классификации компонентов на различных уровнях детализации.}
			\scnfileitem{Не ведётся разработка стандартов, обеспечивающих совместимость этих компонентов.}
			\scnfileitem{Не проводится тестирование, верификация и анализ качества компонентов, не выделяются преимущества, недостатки, ограничения компонентов.}
			\scnfileitem{Многие компоненты используют для идентификации язык разработчика (как правило, английский), и предполагается, что все пользователи будут использовать этот же язык. Однако для многих приложений это недопустимо --- понятные только разработчику идентификаторы должны быть скрыты от конечных пользователей, которые должны быть в состоянии выбрать язык для идентификаторов, которые они видят.}
			\scnfileitem{Отсутствие средств поиска компонентов, удовлетворяющих заданным критериям.}
		\end{scneqtoset}
		\scnrelfrom{источник}{\cite{Shunkevich2015a}}
	\end{scnindent}
	\scntext{примечание}{\scnkeyword{компонентное проектирование интеллектуальных компьютерных систем} возможно только в том случае, если отбор компонентов будет осуществляться на основе тщательного анализа качества этих компонентов. Одним из важнейших критериев такого анализа является уровень семантической совместимости анализируемых компонентов со всеми компонентами, имеющимися в текущей версии библиотеки.}
	\scnrelfrom{предъявляемые требования}{Требования к реализации компонентного проектирования интеллектуальных компьютерных систем}
	\begin{scnindent}
		\begin{scneqtoset}
			\scnfileitem{Обеспечение совместимости (интегрируемости) компонентов интеллектуальных компьютерных систем на основе унификации представления этих компонентов.}
			\scnfileitem{Четкое разделение процесса разработки формальных описаний интеллектуальных компьютерных систем и процесса их реализации по этому описанию.}
			\scnfileitem{Четкое разделение разработки формального описания проектируемой интеллектуальной системы от разработки различных вариантов интерпретации таких формальных описаний систем.}
			\scnfileitem{Наличие онтологии компонентного проектирования интеллектуальных компьютерных систем, включающей (1) описание методов компонентного проектирования, (2) модель \textit{библиотеки многократно используемых компонентов}, (3) модель \textit{спецификации многократно используемых компонентов}, (4) полную \textit{классификацию многократно используемых компонентов}, (5) описание средств взаимодействия разрабатываемой интеллектуальной компьютерной системы с \textit{библиотеками многократно используемых компонентов}.}
			\scnfileitem{Наличие \textit{библиотек многократно используемых компонентов интеллектуальных компьютерных систем}, включающих спецификации компонентов.}
			\scnfileitem{Наличие средств взаимодействия разрабатываемой интеллектуальной компьютерной системы с библиотеками многократно используемых компонентов для установки любых видов компонентов и управления ими в создаваемой системе.}
			\begin{scnindent}
				\scnnote{Под установкой компонента понимается его транспортировка в систему (копирование sc-элементов и/или скачивание файлов компонента), а также выполнение вспомогательных действий для того, чтобы компонент мог функционировать в создаваемой системе.}
			\end{scnindent}
		\end{scneqtoset}
		\begin{scnrelfromlist}{источник}
			\scnitem{\cite{Zalivako2011}}
			\scnitem{\cite{Golenkov2013}}
			\scnitem{\cite{Golenkov2014af}}
		\end{scnrelfromlist}
		\scntext{примечание}{Для того, чтобы решить возникшие проблемы при проектировании интеллектуальных систем и библиотек их многократно используемых компонентов, необходимо придерживаться общих принципов технологии проектирования интеллектуальных компьютерных систем, а также выполнить эти требования.}
	\end{scnindent}
	
		\bigskip
	\end{scnsubstruct}
	\scnsourcecomment{Завершили \scnqqi{Сегмент. Введение в Предметную область и онтологию комплексной библиотеки многократно используемых семантически совместимых компонентов ostis-систем}}
\end{SCn}
\begin{SCn}
	\scnsectionheader{Сегмент. Анализ библиотек многократно используемых компонентов}

\begin{scnsubstruct}
	\scnheader{библиотека многократно используемых компонентов}
	\scntext{примечание}{На данный момент не существует комплексной библиотеки многократно используемых семантически совместимых компонентов компьютерных систем в целом, не говоря об интеллектуальных. Существуют некоторые попытки создания библиотек типовых методов и программ для традиционных компьютерных систем, однако такие библиотеки не решают \scnkeyword{Проблемы в реализации компонентного проектирования интеллектуальных систем}.}
	
	\scnheader{библиотека подпрограмм}
	\scntext{примечание}{Термин \scnqq{библиотека подпрограмм}, одними из первых упомянули Уилкс М., Уиллер Д. и Гилл С. в качестве одной из форм организации вычислений на компьютере. Исходя из изложенного в их книге, под библиотекой понимался набор \scnqq{коротких, заранее заготовленных программ для отдельных, часто встречающихся (стандартных) вычислительных операций}. Стоит отметить, что компонентами библиотек являются не только программы, но и компоненты интерфейсов и баз знаний.}
	\begin{scnindent}
		\begin{scnrelfromlist}{источник}
			\scnitem{\cite{Wilkes1951}}
			\scnitem{\cite{Wilkes1953}}
			\scnitem{\cite{Volchenskova1984}}
		\end{scnrelfromlist}
	\end{scnindent}
	
	\scnheader{библиотека многократно используемых компонентов}
	\scntext{примечание}{К традиционным решениям относятся \scnkeyword{пакетные менеджеры} языков программирования и операционных систем, а также отдельные системы и платформы со встроенными компонентами и средствами для сохранения создаваемых компонентов.}
	\scntext{проблема текущего состояния}{Компоненты библиотеки могут быть реализованы на разных языках программирования (что приводит к тому, что	для каждого языка программирования разрабатываются свои библиотеки со своими решениями различных часто	встречаемых ситуаций), а также могут располагаться в разных местах, что приводит к тому, что в библиотеке необходимо средство для поиска компонентов и их установки.}
	
	\scnheader{пакетный менеджер}
	\begin{scnhaselementrolelist}{пример}
		\scnitem{pip}
		\scnitem{npm}
		\scnitem{poetry}
		\scnitem{maven}
		\scnitem{apt}
		\scnitem{pacman}
	\end{scnhaselementrolelist}
	\scntext{преимущество}{Решение конфликтов при установке зависимых компонентов.}
	\begin{scnrelfromset}{проблемы текущего состояния}
		\scnfileitem{\scnkeyword{пакетные менеджеры} не учитывают семантику компонентов, а только лишь устанавливают компоненты по идентификатору. Библиотеки таких компонентов являются только лишь хранилищем компонентов, никак не учитывающим назначение компонентов, их преимущества и недостатки, сферы применения, иерархию компонентов и другую информацию, необходимую для интеллектуализации компонентного проектирования компьютерных систем.}
		\scnfileitem{Поиск компонентов в библиотеках компонентов, соответствующих данных пакетным менеджерам сводится к поиску по идентификатору компонента. Современные библиотеки компонентов ориентированы только на какой-то определенный язык программирования, операционную систему или платформу.}
		\begin{scnindent}
			\scnrelfrom{источник}{\cite{Blahser2021}}
		\end{scnindent}
		\scnfileitem{Современные пакетные менеджеры являются лишь \scnqq{установщиками} без автоматической интеграции компонентов в систему.}
		\scnfileitem{Существенным недостатком современного подхода является	платформенная зависимость компонентов. Современные библиотеки компонентов ориентированы только на какой-то определенный язык программирования, операционную систему или платформу.}
	\end{scnrelfromset}
	\scntext{примечание}{Пакетные менеджеры языков программирования и операционных систем устроены по следующему принципу: существует хранилище компонентов (библиотека), которая представляет собой множество пакетов этого языка программирования или операционной системы и с которым взаимодействует менеджер компонентов.}
	
	\scnheader{pip}
	\scnrelto{пакетный менеджер}{Python}
	\scntext{примечание}{пакетный менеджер pip является системой управления пакетами, которая используется для установки пакетов из Python Package Index, который является некоторой библиотекой таких пакетов. Зачастую pip устанавливается вместе с Python.}
	\begin{scnrelfromset}{функциональные возможности}
		\scnfileitem{Установка пакета.}
		\scnfileitem{Установка пакета специализированной версии.}
		\scnfileitem{Удаление пакета.}
		\scnfileitem{Переустановка пакета.}
		\scnfileitem{Отображение установленных пакетов.}
		\scnfileitem{Поиск пакетов.}
		\scnfileitem{Верификация зависимостей пакетов.}
		\scnfileitem{Создание файла конфигурации со списком установленных пакетов и их версий.}
		\scnfileitem{Установка множества пакетов из файла конфигурации.}				
	\end{scnrelfromset}
	\scnrelfrom{рисунок}{\scnfileimage[35em]{Contents/part_methods_tools/src/images/sd_ostis_library/configuration_file.png}}
	\begin{scnindent}
		\scntext{пояснение}{Пример файла конфигурации пакетов pip.}
	\end{scnindent}
	\scntext{преимущество}{Хорошо работает с зависимостями, отображает безуспешно установленные пакеты, а также отображает информацию о требуемой версии пакета при конфликте с другим пакетом.}
	\scnrelfrom{альтернатива}{\scnkeyword{poetry}}
	
	\scnheader{poetry}
	\scntext{преимущество}{Автоматически работает с виртуальными окружениями, способен самостоятельно их находить и создавать.}
	\scntext{преимущество}{Файл конфигурации для пакетов poetry является более богатым, чем у pip, он хранит такие сведения, как имя проекта, версия проекта, его описание, лицензия, список авторов, URL проекта, его документации и сайта, список ключевых слов проекта и список PyPI классификаторов.}
	\scnrelfrom{рисунок}{\scnfileimage[35em]{Contents/part_methods_tools/src/images/sd_ostis_library/configuration_file2.png}}
	\begin{scnindent}
		\scntext{пояснение}{Пример файла конфигурации пакетов poetry}
		\scntext{примечание}{Такой вид спецификации не позволяет достичь совместимости между компонентами даже в рамках Python проектов и предназначена преимущественно только для чтения разработчиком.}
	\end{scnindent}
	\scntext{пояснение}{Автоматизировать проектирование компьютерных систем с помощью пакетного менеджера  \scnkeyword{poetry} или \scnkeyword{pip} невозможно, так как требуется вмешательство разработчика, который должен вручную совместить интерфейсы устанавливаемых пакетов.}
	
	\scnheader{Библиотека STL}
	\scnidtf{Библиотека стандартных шаблонов С++}
	\scniselement{библиотека подпрограмм языка программирования}
	\scniselement{C++}
	\scntext{пояснение}{Библиотека STL представляет собой набор согласованных обобщенных алгоритмов, контейнеров, средств доступа к их содержимому и различных вспомогательных функций в C++.}
	\begin{scnrelfromset}{включение}
		\scnitem{контейнер}
		\begin{scnindent}
			\scntext{назначение}{Хранение набора объектов в памяти.}
		\end{scnindent}
		\scnitem{итератор}
		\begin{scnindent}
			\scntext{назначение}{Обеспечение средств доступа к содержимому контейнера.}
		\end{scnindent}
		\scnitem{алгоритм}
		\begin{scnindent}
			\scntext{назначение}{Определение вычислительной процедуры.}
		\end{scnindent}
		\scnitem{адаптер}
		\begin{scnindent}
			\scntext{назначение}{Адаптация компонентов для обеспечения различного интерфейса.}
		\end{scnindent}
		\scnitem{функциональный объект}			
		\begin{scnindent}
			\scntext{назначение}{Сокрытие функции в объекте для использования другими компонентами.}
		\end{scnindent}
	\end{scnrelfromset}
	\scnrelfrom{рисунок}{\scnfileimage[35em]{Contents/part_methods_tools/src/images/sd_ostis_library/structure_stl.png}}
	\scntext{примечание}{Составляющие Библиотеки STL позволяют уменьшить количество создаваемых компонентов. Например, вместо написания отдельной функции поиска элемента для каждого типа контейнера обеспечивается единственная версия, которая работает с каждым из них, пока соблюдаются основные требования.}
	\scntext{примечание}{Совместимость компонентов (контейнеров) в Библиотеке STL обеспечивается общим интерфейсом использования этих компонентов.}
	
	\scnheader{компонентное проектирование компьютерных систем}
	\scntext{примечание}{Компонентный подход к проектированию компьютерных систем может реализовываться в рамках различных языков, платформ и приложений.}
	\begin{scnrelfromset}{примеры реализации}
		\scnitem{OWL}
		\begin{scnindent}
			\scntext{примечание}{Онтология, реализованная на языке \textit{OWL} (Web Ontology Language), представляет собой множество декларативных утверждений о сущностях словаря предметной области. \textit{OWL} предполагает концепцию \scnqq{открытого мира}, в соответствии с которой применимость описаний предметной области, помещенных в конкретном физическом документе, не ограничивается лишь рамками этого документа --- содержание онтологии может быть использовано и дополнено другими документами, добавляющими новые факты о тех же сущностях или описывающими другую предметную область в терминах данной. \scnqq{Открытость мира} достигается путем добавления URI каждому элементу онтологии, что позволяет воспринимать описанную на \textit{OWL} онтологию как часть всеобщего объединенного знания.}
			\begin{scnindent}
				\scnrelfrom{источник}{\cite{Hepp2008}}
			\end{scnindent}
		\end{scnindent}
		\scnitem{WebProtege}
		\begin{scnindent}
			\scntext{примечание}{\scnkeyword{WebProtege} представляет собой многопользовательский веб-интерфейс, позволяющий редактировать и хранить онтологии в формате \textit{OWL} в совместной среде. Данный проект позволяет не только создавать новые онтологии, но также загружать уже существующие онтологии, которые хранятся на сервере университета Стэнфорда. К преимуществу данного проекта можно отнести автоматическую проверку ошибок в процессе создания объектов онтологий. Данный проект является примером попытки решения проблемы накопления, систематизации и повторного использования уже существующих решений, однако, недостатком данного решения является обособленность разрабатываемых онтологий. Каждый разработанный компонент имеет свою иерархию понятий, подход к выделению классов и сущностей, которые зависят от разработчиков данных онтологий, так как в рамках данного подхода не существует универсальной модели представления знаний, а также формальной спецификации компонентов, представленных в виде онтологий. Следовательно, возникает проблема их семантической несовместимости, что, в свою очередь, приводит к невозможности повторного использования разработанных онтологий при проектировании баз знаний. Данный факт подтверждается наличием на сервере университета Стэнфорда многообразия различных онтологий на одни и те же темы.}
			\begin{scnindent}
				\scnrelfrom{источник}{\cite{Memduhoglu2018}}
			\end{scnindent}
		\end{scnindent}
		\scnitem{Modelica}
		\begin{scnindent}
			\scntext{примечание}{На основе языка \scnkeyword{Modelica} разработано большое число свободно доступных библиотек компонентов, одной из которых является библиотека Modelica\_StateGraph2, включающая компоненты для моделирования дискретных событий, реактивных и гибридных систем с помощью иерархических диаграмм состояния. Основным недостатком систем на базе языка \textit{Modelica} является отсутствие совместимости компонентов и достаточной документации, а также узкая направленность разрабатываемых компонентов.}
			\begin{scnindent}
				\scnrelfrom{источник}{\cite{Fritzson2014}}
			\end{scnindent}
		\end{scnindent}
		\scnitem{Microsoft Power Apps}
		\begin{scnindent}
			\scntext{примечание}{\scnkeyword{Microsoft Power Apps} --- это набор приложений, служб и соединителей, а также платформа данных, которая предоставляет среду разработки для эффективного создания пользовательских приложений для бизнеса. Платформа \textit{Microsoft Power Apps} имеет средства для создания библиотеки многократно используемых компонентов графического интерфейса, а также предварительно созданные модели распознавания текста (чтение визитных карточек или чеков) и средство обнаружения объектов, которые можно подключить к разрабатываемому приложению. Библиотека компонентов \textit{Microsoft Power Apps} представляет собой множество создаваемых пользователем компонентов, которые можно использовать в любых приложениях. Преимущество библиотеки в том, что компоненты могут настраивать свойства по умолчанию, которые можно гибко редактировать в любых приложениях, использующих компоненты. Недостаток в том, что отсутствует семантическая совместимость компонентов, спецификация компонентов, не решена проблема существования семантически эквивалентных компонентов, нет иерархии компонентов и средств поиска этих компонентов. Компоненты платформы \textit{Microsoft Power Apps} являются многократно используемыми только для однотипных приложений, которые создаются одним и тем же разработчиком.}
			\begin{scnindent}
				\scnrelfrom{источник}{\cite{Prakash2022}}
			\end{scnindent}
		\end{scnindent}
		\scnitem{Платформа IACPaaS}
		\begin{scnindent}
			\scntext{примечание}{\scnkeyword{Платформа IACPaaS} (Intelligent Applications, Control and Platform as a Service) --- облачная платформа для разработки, управления и удаленного использования интеллектуальных облачных сервисов. Она предназначена для обеспечения поддержки разработки, управления и удаленного использования прикладных и инструментальных мультиагентных облачных сервисов (прежде всего интеллектуальных) и их компонентов для различных предметных областей.}
			\begin{scnindent}
				\scnrelfrom{источник}{\cite{Moskalenko2016}}
			\end{scnindent}
			\begin{scnrelfromset}{предоставляет доступ}
				\scnfileitem{Прикладным пользователям (специалистам в различных предметных областях) --- к прикладным сервисам.}
				\scnfileitem{Разработчикам прикладных и инструментальных сервисов и их компонентов --- к инструментальным сервисам.}
				\scnfileitem{Управляющим интеллектуальными сервисами.}
				\scnfileitem{К сервисам управления.}
			\end{scnrelfromset}
			\begin{scnrelfromset}{поддерживает}
				\scnfileitem{Базовую технологию разработки прикладных и специализированных инструментальных (интеллектуальных) сервисов с использованием базовых инструментальных сервисов платформы, поддерживающих эту технологию.}
				\scnfileitem{Множество специализированных технологий разработки прикладных и специализированных инструментальных (интеллектуальных) сервисов, с использованием специализированных инструментальных сервисов платформы, поддерживающих эти технологии.}
			\end{scnrelfromset}
			\scntext{недостаток}{\textit{Платформа IACPaaS} не имеет средств для унифицированного представления компонентов интеллектуальных компьютерных систем и средств для их спецификации и автоматической интеграции компонентов.}
		\end{scnindent}
		\scntext{примечание}{На текущем состоянии развития информационных технологий \uline{не существует} комплексной библиотеки многократно используемых семантически совместимых компонентов компьютерных систем. Таким образом, предлагается комплексная библиотека многократно используемых семантически совместимых компонентов ostis-систем.}
	\end{scnrelfromset}
	 
	
		\bigskip
	\end{scnsubstruct}
	\scnsourcecomment{Завершили \scnqqi{Сегмент. Анализ библиотек многократно используемых компонентов}}
\end{SCn}
\begin{SCn}
	\scnsectionheader{Сегмент. Понятие библиотеки многократно используемых компонентов ostis-систем}

\begin{scnsubstruct}
	
	 \scnheader{библиотека многократно используемых компонентов ostis-систем}
	 \scntext{часто используемый sc-идентификатор}{библиотека компонентов ostis-систем}
	 \scntext{часто используемый sc-идентификатор}{библиотека компонентов}
	 \scnidtf{библиотека совместимых многократно используемых компонентов}
	 \scnidtf{комплексная библиотека многократно используемых семантически совместимых компонентов ostis-систем}
	 \scnidtf{библиотека многократно используемых и совместимых компонентов интеллектуальных компьютерных систем нового поколения}
	 \scnidtf{библиотека типовых компонентов ostis-систем}
	 \scnidtf{библиотека многократно используемых компонентов OSTIS}
	 \scnidtf{библиотека повторно используемых компонентов OSTIS}
	 \scnidtf{библиотека intelligent property компонентов ostis-систем}
	 \begin{scnindent}
	 	\scntext{сокращение}{библиотека ip-компонентов ostis-систем}
	 \end{scnindent}
	 \scntext{примечание}{библиотека многократно используемых компонентов ostis-систем позволяет использовать проектный опыт по разработке и модернизации ostis-систем различного назначения.}
	 \scnhaselementrole{типичный пример}{\scnkeyword{Библиотека Метасистемы OSTIS}}
	 \begin{scnindent}
	 	\scnidtf{Распределенная библиотека типовых (многократно используемых) компонентов ostis-систем в составе Метасистемы OSTIS}
	 	\scnidtf{Библиотека многократно используемых компонентов ostis-систем в составе \textit{Метасистемы OSTIS}}
	 \end{scnindent}
	 \scnhaselementrole{типичный пример}{\scnkeyword{Библиотека Экосистемы OSTIS}}
	 \begin{scnindent}
	 	\scntext{часто используемый sc-идентификатор}{Библиотека OSTIS}
	 	\scnidtf{Библиотека многократно используемых и совместимых компонентов интеллектуальных компьютерных систем нового поколения}
	 	\scnidtf{Библиотека типовых компонентов интеллектуальных компьютерных систем нового поколения}
	 	\scnidtf{Распределенная библиотека типовых (многократно используемых) компонентов ostis-систем в составе Экосистемы OSTIS}
	 	\scnidtf{Библиотека многократно используемых компонентов ostis-систем в составе \textit{Экосистемы OSTIS}}
	 	\scntext{примечание}{Все библиотеки в рамках \textit{Экосистемы OSTIS} объединяются в \textit{Библиотеку Экосистемы OSTIS}.}
	 	\scntext{примечание}{Постоянно расширяемая Библиотека Экосистемы OSTIS существенно сокращает сроки разработки новых интеллектуальных компьютерных систем.}
	 	\scntext{назначение}{Основное назначение Библиотеки Экосистемы OSTIS --- создание условий для эффективного, осмысленного и массового проектирования ostis-систем и их компонентов путём создания среды для накопления и совместного использования компонентов ostis-систем.}
	 	\begin{scnindent}
	 		\scntext{примечание}{Такие условия осуществляются путём неограниченного расширения постоянно эволюционируемых семантически совместимых ostis-систем и их компонентов, входящих в \textit{Экосистему OSTIS}.}
	 	\end{scnindent}
	 	\scntext{примечание}{Различные \textit{многократно используемые компоненты ostis-систем} объединяются в \textit{библиотеки многократно используемых компонентов ostis-систем}. Разработчики \uline{любой} \textit{ostis-системы} могут включить в ее состав библиотеку, которая позволит им накапливать и распространять результаты своей деятельности среди других участников \textit{Экосистемы OSTIS} в виде \scnkeyword{многократно используемых компонентов}. Решение о включении компонента в библиотеку принимается экспертным сообществом разработчиков, ответственным за качество этой библиотеки. Верификацию компонентов можно автоматизировать путем проверки наличия обязательной части их спецификации, а также тестированием корректности автоматической установки, интеграции и функционирования компонентов.}
	 \end{scnindent}
	 \scntext{примечание}{В рамках \textit{Экосистемы OSTIS} существует множество библиотек многократно используемых компонентов ostis-систем, являющихся подсистемами соответствующих материнских ostis-систем. Главной библиотекой многократно используемых компонентов ostis-систем является \textit{Библиотека Метасистемы OSTIS}. \textit{Метасистема OSTIS} выступает \scnkeyword{материнской системой} для всех разрабатываемых ostis-систем, поскольку содержит все базовые компоненты.}
	 \begin{scnindent}
	 	\scnrelfrom{описание примера}{\scnfileimage[35em]{Contents/part_methods_tools/src/images/sd_ostis_library/ecosystem_architecture.png}}
	 	\begin{scnindent}
	 		\scnrelfrom{смотрите}{менеджер многократно используемых компонентов ostis-систем}
	 	\end{scnindent}
	 \end{scnindent}
	 \begin{scnrelfromset}{функциональные возможности}
	 	\scnfileitem{Хранение многократно используемых компонентов ostis-систем и их спецификаций.}
	 	\begin{scnindent}
	 		\scntext{примечание}{При этом часть компонентов, специфицированных в рамках библиотеки, могут физически храниться в другом месте ввиду особенностей их  технической реализации (например, исходные тексты платформы интерпретации sc-моделей компьютерных систем могут физически храниться в каком-либо отдельном репозитории, но специфицированы как компонент будут в соответствующей библиотеке). В этом случае спецификация компонента в рамках библиотеки должна также включать описание (1) того, где располагается компонент, и (2) сценария его автоматической или хотя бы ручной установки в дочернюю ostis-систему.}
	 		\begin{scnindent}
	 			\scnrelfrom{смотрите}{менеджер многократно используемых компонентов ostis-систем}
	 		\end{scnindent}
	 	\end{scnindent}
	 	\scnfileitem{Просмотр имеющихся компонентов и их спецификаций, а также поиск компонентов по фрагментам их спецификации.}
	 	\scnfileitem{Хранение сведений о том, в каких ostis-системах-потребителях какие из компонентов библиотеки и какой версии используются (были скачаны). Это необходимо как минимум для учета востребованности того или иного компонента, оценки его важности и популярности.}
	 	\scnfileitem{Систематизация многократно используемых компонентов ostis-систем.}
	 	\scnfileitem{Обеспечение версионирования многократно используемых компонентов ostis-систем.}
	 	\scnfileitem{Поиск зависимостей между многократно используемыми компонентами в рамках библиотеки компонентов.}
	 	\scnfileitem{Обеспечение автоматического обновления компонентов, заимствованных в дочерние ostis-системы. Данная функция может включаться и отключаться по желанию разработчиков дочерней ostis-системы.}
	 	\begin{scnindent}
	 		\scntext{примечание}{Одновременное обновление одних и тех же компонентов во всех системах, его использующих, не должно ни в каком контексте приводить к несогласованности между этими системами. Это требование может оказаться довольно сложным, но без него работа Экосистемы невозможна.}
	 	\end{scnindent}
	 \end{scnrelfromset}
	 \begin{scnindent}
	 	\scnrelfrom{источник}{\cite{Koronchik2011}}
	 \end{scnindent}
	 \scntext{примечание}{\scnkeyword{библиотека многократно используемых компонентов ostis-систем} позволяет избавиться от дублирования семантически эквивалентных информационных компонентов. А также от многообразия форм технической реализации используемых моделей решения задач.}
	 \scntext{примечание}{Проблема интеграции многократно используемых компонентов ostis-систем решается путем взаимодействия компонентов через общую базу знаний. Компоненты могут использоватьобщие ключевые узлы (понятия) в базе знаний. Интеграция многократно используемых компонентов ostis-систем сводится к отождествлению (склеиванию) ключевых узлов по различным признакам и устранению возможных дублирований и противоречий исходя из спецификации компонента и его содержания. Такой способ интеграции компонентов позволяет разрабатывать их параллельно и независимо друг от друга, что значительно сокращает сроки проектирования. Отождествление sc-элементов происходит в ходе выполнения \scnkeyword{действие. отождествить два указанных sc-элемента}. Автоматическая интеграция компонентов интеллектуальных систем представляет широкие возможности для существенного сокращения сроков проектирования интеллектуальных систем, поскольку позволяет использовать опыт прошлых разработок. Интеграция любых компонентов ostis-систем происходит автоматически, без вмешательства разработчика. Это достигается за счет использования SC-кода и его преимуществ, унификации спецификации многократно используемых компонентов и тщательного отбора компонентов в библиотеках экспертным сообществом, ответственным	за эту библиотеку.}
	 \begin{scnrelfromlist}{источник}
	 	\scnitem{\cite{Ivashenko2011}}
	 	\scnitem{\cite{Ivashenko2013}}
	 	\scnitem{\cite{Golenkov2014b}}
	 \end{scnrelfromlist}
	 \begin{scnindent}
	 	\scntext{примечание}{Это достигается за счёт использования \textit{SC-кода} и его преимуществ, унификации спецификации многократно используемых компонентов и тщательного отбора компонентов в библиотеках экспертным сообществом, ответственным за эту библиотеку.}
	 \end{scnindent}
	 \scnheader{ostis-система}
	 \scnsuperset{материнская ostis-система}
	 \begin{scnindent}
	 	\scntext{пояснение}{ostis-система, имеющая в своем составе библиотеку многократно используемых компонентов.}
	 	\scnhaselement{Метасистема OSTIS}
	 	\scntext{примечание}{материнская ostis-система в свою очередь может являться дочерней ostis-системой для какой-либо другой ostis-системы, заимствуя компоненты из библиотеки, входящей в состав этой другой ostis-системы.}
	 	\scntext{примечание}{материнская ostis-система отвечает за какой-то класс компонентов и является САПРом для этого класса, например, содержит методики разработки таких компонентов, их классификацию, подробные пояснения ко всем подклассам компонентов. Таким образом, формируется иерархия \scnkeyword{материнских ostis-систем}.}
	 \end{scnindent}
	 \scnsuperset{дочерняя ostis-система}
	 \begin{scnindent}
	 	\scntext{пояснение}{ostis-система, в составе которой имеется компонент, заимствованный из какой-либо библиотеки многократно используемых компонентов.}
	 \end{scnindent}
	 \scnheader{библиотека многократно используемых компонентов ostis-систем}
	 \begin{scnreltoset}{объединение}
	 	\scnitem{библиотека многократно используемых компонентов баз знаний ostis-систем}
	 	\scnitem{библиотека многократно используемых компонентов решателей задач ostis-систем}
	 	\scnitem{библиотека многократно используемых компонентов интерфейсов ostis-систем}
	 	\scnitem{библиотека встраиваемых ostis-систем}
	 	\scnitem{библиотека ostis-платформ}
	 \end{scnreltoset}
	 \scntext{примечание}{библиотека многократно используемых компонентов ostis-систем является подсистемой ostis-систем, которая имеет свою базу знаний, свой решатель задач и свой интерфейс. Однако не каждая ostis-система обязана иметь библиотеку компонентов.}
	 \begin{scnrelfromset}{обобщенная декомпозиция}
	 	\scnitem{база знаний библиотеки многократно используемых компонентов ostis-систем}
	 	\begin{scnindent}
	 		\scntext{примечание}{база знаний библиотеки многократно используемых компонентов ostis-систем представляет собой иерархию многократно используемых компонентов ostis-систем и их спецификаций.}
	 	\end{scnindent}
	 	\scnitem{решатель задач библиотеки многократно используемых компонентов ostis-систем}
	 	\begin{scnindent}
	 		\scntext{примечание}{решатель задач библиотеки многократно используемых компонентов ostis-систем реализует функциональные возможности библиотеки ostis-систем.}
	 	\end{scnindent}
	 	\scnitem{интерфейс библиотеки многократно используемых компонентов ostis-систем}
	 	\begin{scnindent}
	 		\scntext{примечание}{\scnkeyword{интерфейс библиотеки многократно используемых компонентов ostis-систем} обеспечивает доступ к многократно используемым компонентам и возможностям библиотеки ostis-систем для пользователя и других систем.}
	 		\begin{scnindent}
	 			\scnrelfrom{источник}{\cite{Koronchik2011}}
	 		\end{scnindent}
	 		\begin{scnrelfromset}{декомпозиция}
	 			\scnitem{минимальный интерфейс библиотеки многократно используемых компонентов ostis-систем}
	 			\begin{scnindent}
	 				\scntext{примечание}{Данный вид интерфейса позволяет \textit{менеджеру многократно используемых компонентов ostis-систем}, входящему в состав какой-либо дочерней ostis-системы, подключиться к библиотеке многократно используемых компонентов ostis-систем и использовать ее функциональные возможности, то есть, например, получить доступ к спецификации компонентов и установить выбранные компоненты в дочернюю ostis-систему, получить сведения о доступных версиях компонента, его зависимостях и так далее.}
	 			\end{scnindent}
	 			\scnitem{расширенный интерфейс библиотеки многократно используемых компонентов ostis-систем}
	 			\begin{scnindent}
	 				\scnidtf{графический интерфейс библиотеки многократно используемых компонентов ostis-систем}
	 				\scntext{примечание}{В частном случае у библиотеки может быть расширенный пользовательский интерфейс, который, в отличие от минимального интерфейса, позволяет не только получить доступ к компонентам для дальнейшей работы с ними, но и просматривать существующую структуру библиотеки, а также компоненты и их элементы в удобном и интуитивно понятном для пользователя виде.}
	 			\end{scnindent}
	 		\end{scnrelfromset}
	 	\end{scnindent}
	 \end{scnrelfromset}
	
		\bigskip
	\end{scnsubstruct}
	\scnsourcecomment{Завершили \scnqqi{Сегмент. Понятие библиотеки многократно используемых компонентов ostis-систем}}
\end{SCn}
\begin{SCn}
	\scnsectionheader{Сегмент. Понятие многократно используемого компонента ostis-систем}

\begin{scnsubstruct}
	
	\scnheader{многократно используемый компонент ostis-систем}
	\scnidtf{типовой компонент ostis-систем}
	\scnidtf{повторно используемый компонент ostis-систем}
	\scnidtf{многократно используемый компонент OSTIS}
	\scnidtf{ip-компонент ostis-систем}
	\scnidtftext{часто используемый sc-идентификатор}{многократно используемый компонент}
	\scnrelfrom{аббревиатура}{\scnfilelong{МИК ostis-систем}}
	\scnsubset{sc-структура}
	\scnsubset{компонент ostis-системы}
	\begin{scnindent}
		\scntext{пояснение}{Целостная часть ostis-системы, которая содержит все те (и только те) sc-элементы, которые необходимы для её функционирования в ostis-системе.}
	\end{scnindent}
	\scntext{определение}{многократно используемый компонент ostis-систем --- компонент некоторой ostis-системы, который может быть использован в рамках другой ostis-системы.}
	\begin{scnindent}
		\scnrelfrom{источник}{\cite{Shunkevich2015a}}
	\end{scnindent}
	\scntext{пояснение}{Компонент ostis-системы, который может быть использован в других ostis-системах (\scnkeyword{дочерних ostis-системах}).}
	\scntext{пояснение}{Компонент некоторой \scnkeyword{материнской ostis-системы}, который может быть использован в некоторой \scnkeyword{дочерней ostis-системе}.}
	\scntext{пояснение}{многократно используемый компонент ostis-систем — это общий компонент (общая часть) для некоторого множества ostis-систем, который многократно используется, дублируется и входит в состав некоторого множества ostis-систем.}
	\scntext{примечание}{Для включения многократно используемого компонента в некоторую систему, его необходимо установить в эту систему, то есть скопировать в нее все sc-элементы компонента и, при необходимости, вспомогательные файлы, такие как исходные или скомпилированные файлы компонента.}
	\scntext{примечание}{многократно используемый компонент ostis-систем должен иметь унифицированную спецификацию и иерархию для поддержки \uline{совместимости} с другими компонентами.}
	\scntext{примечание}{Совместимость многократно используемых компонентов приводит систему к новому качеству, к дополнительному расширению множества решаемых задач при интеграции компонентов.}
	\begin{scnrelfromset}{необходимые требования}
		\scnfileitem{Существует техническая возможность встроить многократно используемый компонент в \scnkeyword{дочернюю ostis-систему}.}
		\scnfileitem{Полнота многократно используемого компонента: компонент должен в полной мере выполнять свои функции, соответствовать своему назначению.}
		\scnfileitem{Связность многократно используемого компонента: компонент должен логически выполнять только одну задачу из предметной области, для которой он предназначен. Многократно используемый компонент должен выполнять свои функции наиболее общим образом, чтобы круг возможных систем, в которые он может быть встроен, был наиболее широким.}
		\scnfileitem{Совместимость многократно используемого компонента: компонент должен стремиться повышать уровень \uline{договороспособности} ostis-систем, в которые он встроен, и иметь возможность \uline{автоматической} интеграции в другие системы.}
		\scnfileitem{Самодостаточность компонентов (или групп компонентов) технологии, то есть способности их функционировать отдельно от других компонентов без утраты целесообразности их использования.}
	\end{scnrelfromset}
	\scnheader{следует отличать*}
	\begin{scnhaselementset}
		\scnitem{многократно используемый компонент ostis-систем}
		\scnitem{компонент ostis-системы}
	\end{scnhaselementset}
	\scntext{отличие}{многократно используемый компонент ostis-систем имеет \uline{спецификацию, достаточную для установки} этого компонента в \scnkeyword{дочернюю ostis-систему}. Спецификация является частью базы знаний \scnkeyword{библиотеки многократно используемых компонентов} соответствующей \scnkeyword{материнской ostis-системы}. Есть техническая возможность встроить многократно используемый компонент в дочернюю ostis-систему.}
	\scnheader{параметр, заданный на многократно используемых компонентах ostis-систем\scnsupergroupsign}
	\scnsubset{параметр}
	\scnhaselement{класс многократно используемого компонента ostis-систем\scnsupergroupsign}
	\begin{scnindent}
		\scntext{примечание}{класс многократно используемого компонента ostis-систем является важной частью спецификации компонента, позволяющей лучше понять назначение и область применения данного компонента, а также класс многократно используемого компонента является важнейшим критерием поиска компонентов в библиотеке ostis-систем.}
	\end{scnindent}
	\scnhaselement{начало\scnsupergroupsign}
	\scnhaselement{завершение\scnsupergroupsign}
	\scnheader{многократно используемый компонент ostis-систем}
	\scntext{примечание}{Интеллектуальная система, спроектированная по \textit{Технологии OSTIS}, представляет собой интеграцию \textit{многократно используемых компонентов баз знаний}, \textit{многократно используемых компонентов решателей задач} и \textit{многократно используемых компонентов интерфейсов}.}
	\begin{scnindent}
		\scnrelfrom{источник}{\cite{Pivovarchik2015}}
	\end{scnindent}
	\scntext{примечание}{Основной признак классификации многократно используемых компонентов является признак предметной области, к которой относится компонент. Здесь структура может быть довольно богатой в соответствии с иерархией областей человеческой деятельности. Существует также множество предметно-независимых многократно используемых компонентов, которые могут использоваться в любой предметной области.}
	\begin{scnindent}
		\scnrelfrom{источник}{\cite{Orlov2022a}}
	\end{scnindent}
	\begin{scnrelfromset}{разбиение}
		\scnitem{многократно используемый компонент базы знаний ostis-систем}
		\begin{scnindent}
			\scnidtf{многократно используемый компонент базы знаний}
			\scniselement{класс многократно используемого компонента ostis-систем\scnsupergroupsign}
			\scntext{примечание}{Важнейшим признаком классификации многократно используемых компонентов баз знаний является вид знаний.}
			\begin{scnindent}
				\scnrelfrom{смотрите}{вид знаний}
			\end{scnindent}
			\scnsuperset{семантическая окрестность}
			\begin{scnindent}
				\scnhaselement{Семантическая окрестность города Минска}
				\scnhaselement{Семантическая окрестность понятия множество}
			\end{scnindent}
			\scnsuperset{предметная область и онтология}
			\begin{scnindent}
				\scnhaselement{Предметная область и онтология треугольников}
			\end{scnindent}
			\scnsuperset{база знаний}
			\scnsuperset{шаблон типового компонента ostis-систем}
			\begin{scnindent}
				\scnhaselement{Шаблон описания предметной области}
				\scnhaselement{Шаблон описания отношения}
			\end{scnindent}
		\end{scnindent}
		\scnitem{многократно используемый компонент решателя задач ostis-систем}
		\begin{scnindent}
			\scnidtf{многократно используемый компонент решателя задач}
			\scniselement{класс многократно используемого компонента ostis-систем\scnsupergroupsign}
			\scntext{примечание}{Важнейшим признаком классификации многократно используемых компонентов баз решателя задач является используемая модель решения задачи.}
			\scnsuperset{атомарный абстрактный sc-агент}
			\begin{scnindent}
				\scnhaselement{Абстрактный sc-агент подсчета мощности множества}
			\end{scnindent}
			\scnsuperset{программа обработки знаний}
			\scnsuperset{scp-машина}
			\scnsuperset{scl-машина}
		\end{scnindent}
		\scnitem{многократно используемый компонент интерфейса ostis-систем}
		\begin{scnindent}
			\scnidtf{многократно используемый компонент интерфейса}
			\scniselement{класс многократно используемого компонента ostis-систем\scnsupergroupsign}
			\scntext{примечание}{Важнейшим признаком классификации многократно используемых компонентов баз решателя задач является вид интерфейса в соответствии с классификацией интерфейсов.}
			\scnsuperset{многократно используемый компонент пользовательских интерфейсов ostis-систем}
		\end{scnindent}
	\end{scnrelfromset}
	\scnrelfrom{разбиение}{\scnkeyword{Типология компонентов ostis-систем по атомарности\scnsupergroupsign}}
	\begin{scnindent}
		\scnsubset{класс многократно используемого компонента ostis-систем\scnsupergroupsign}
		\begin{scneqtoset}
			\scnitem{атомарный многократно используемый компонент ostis-систем}
			\begin{scnindent}
				\scnhaselement{Абстрактный sc-агент подсчета мощности множества}
				\scntext{пояснение}{Многократно используемый компонент, который в текущем состоянии библиотеки ostis-систем рассматривается как неделимый, то есть не содержит в своем составе других компонентов.}
				\scntext{примечание}{Принадлежность МИК ostis-систем классу атомарных компонентов зависит от того, как специфицирован этот компонент, и от существующих на данный момент компонентов в библиотеке.}
				\begin{scnindent}
					\scntext{примечание}{В библиотеку ostis-систем нельзя опубликовать многократно используемый компонент как атомарный, в составе которого есть какой-либо другой известный библиотеке ostis-систем компонент.}
				\end{scnindent}
				\scntext{примечание}{В общем случае атомарный компонент может перейти в разряд неатомарных в случае, если будет принято решение выделить какую-то из его частей в качестве отдельного компонента. Все вышесказанное, однако, не означает, что даже в случае его платформенной независимости, атомарный компонент всегда хранится в sc-памяти как сформированная sc-структура. Например, платформенно-независимая реализация sc-агента всегда будет представлена набором \textit{scp-программ}, но будет \textit{атомарным многократно используемым компонентом ostis-систем} в случае, если ни одна из этих программ не будет представлять интереса как самостоятельный компонент.}
			\end{scnindent}
			\scnitem{неатомарный многократно используемый компонент ostis-систем}
			\begin{scnindent}
				\scnidtf{составной многократно используемый компонент ostis-систем}
				\scnhaselement{Решатель задач по геометрии}
				\scntext{пояснение}{Многократно используемый компонент, который в текущем состоянии библиотеки ostis-систем содержит в своем составе другие атомарные или неатомарные компоненты.}
				\scntext{примечание}{Неатомарный многократно используемый компонент не зависит от своих частей. Без какой-либо части неатомарного компонента его назначение сужается.}
				\scntext{примечание}{В общем случае неатомарный компонент может перейти в разряд атомарных в случае, если будет принято решение по каким-либо причинам исключить все его части из рассмотрения в качестве отдельных компонентов. Следует отметить, что неатомарный компонент необязательно должен складываться \uline{полностью} из других компонентов, в его состав могут также входить и части, не являющиеся самостоятельными компонентами. Например, в состав реализованного на \textit{Языке SCP} \textit{sc-агента}, являющего \textit{неатомарным многократно используемым компонентом} могут входить как \textit{scp-программы}, которые могут являться многократно используемыми компонентами (а могут и не являться), а также агентная \textit{scp-программа}, которая не имеет смысла как многократно используемый компонент.}
				\scntext{примечание}{Спецификация неатомарного многократно используемого компонента должна содержать информацию о том, из каких компонентов он состоит, используя отношение декомпозиция*. При этом sc-структура, обозначающая неатомарный компонент не обязана содержать все sc-элементы компонентов, на которые она декомпозируется, достаточно, чтобы неатомарному компоненту принадлежали знаки всех тех компонентов, из которых он состоит. Должно быть полное перечисление составных компонентов.}
			\end{scnindent}
		\end{scneqtoset}
	\end{scnindent}
	\scnrelfrom{разбиение}{\scnkeyword{Типология компонентов ostis-систем по зависимости от других компонентов\scnsupergroupsign}}
	\begin{scnindent}
		\scnsubset{класс многократно используемого компонента ostis-систем\scnsupergroupsign}
		\begin{scneqtoset}
			\scnitem{зависимый многократно используемый компонент ostis-систем}
			\begin{scnindent}
				\scnhaselement{визуальный редактор системы по химии}
				\scntext{пояснение}{Многократно используемый компонент, который зависит хотя бы от одного другого компонента библиотеки ostis-систем, то есть не может быть встроен в дочернюю ostis-систему без компонентов, от которых он зависит.}
			\end{scnindent}
			\scnitem{независимый многократно используемый компонент ostis-систем}
			\begin{scnindent}
				\scnhaselement{Предметная область множеств}
				\scntext{пояснение}{Многократно используемый компонент, который не зависит ни от одного другого компонента библиотеки ostis-систем.}
			\end{scnindent}
		\end{scneqtoset}
	\end{scnindent}
	\scnrelfrom{разбиение}{\scnkeyword{Типология компонентов ostis-систем по способу их хранения\scnsupergroupsign}}
	\begin{scnindent}
		\scnsubset{класс многократно используемого компонента ostis-систем\scnsupergroupsign}
		\begin{scneqtoset}
			\scnitem{многократно используемый компонент ostis-систем, хранящийся в виде внешних файлов}
			\begin{scnindent}
				\begin{scnrelfromset}{разбиение}
					\scnitem{многократно используемый компонент ostis-систем, хранящийся в виде файлов исходных текстов}
					\scnitem{многократно используемый компонент ostis-систем, хранящийся в виде скомпилированных файлов}
				\end{scnrelfromset}
			\end{scnindent}
			\scnitem{многократно используемый компонент, хранящийся в виде sc-структуры}
		\end{scneqtoset}
		\scntext{примечание}{На данном этапе развития \textit{Технологии OSTIS} более удобным является хранение компонентов в виде исходных текстов.}
	\end{scnindent}
	\scnrelfrom{разбиение}{\scnkeyword{Типология компонентов ostis-систем по зависимости от платформы\scnsupergroupsign}}
	\begin{scnindent}
		\scnsubset{класс многократно используемого компонента ostis-систем\scnsupergroupsign}
		\begin{scneqtoset}
			\scnitem{платформенно зависимый многократно используемый компонент ostis-систем}
			\begin{scnindent}
				\scntext{пояснение}{Под платформенно-зависимым многократно используемым компонентом OSTIS понимается компонент, частично или полностью реализованный при помощи каких-либо сторонних с точки зрения \textit{Технологии OSTIS} средств.}
				\scntext{недостаток}{Интеграция таких компонентов в интеллектуальные системы может сопровождаться дополнительными трудностями, зависящими от конкретных средств реализации компонента.}
				\scntext{преимущество}{В качестве возможного преимущества платформенно-зависимых многократно используемых компонентов ostis-систем можно выделить их, как правило, более высокую производительность за счет реализации их на более приближенном к платформе уровне.}
				\scntext{примечание}{С точки зрения \textit{Технологии OSTIS} любая ostis-платформа является платформенно-зависимым многократно используемым компонентом.}
				\scntext{примечание}{В общем случае платформенно-зависимый многократно используемый компонент ostis-систем может поставляться как в виде набора исходных кодов, так и в скомпилированном виде. Процесс интеграции платформенно-зависимого многократно используемого компонента ostis-систем в дочернюю систему, разработанную по \textit{Технологии OSTIS}, сильно зависит от технологий реализации данного компонента и в каждом конкретном случае может состоять из различных этапов. Каждый платформенно-зависимый многократно используемый компонент ostis-систем должен иметь соответствующую подробную, корректную и понятную инструкцию по его установке и внедрению в дочернюю систему.}
				\scnsuperset{ostis-платформа}
				\scnsuperset{абстрактный sc-агент, не реализуемый на Языке SCP}
			\end{scnindent}
			\scnitem{платформенно-независимый многократно используемый компонент ostis-систем}
			\begin{scnindent}
				\scntext{пояснение}{Под платформенно-независимым многократно используемым компонентом ostis-систем понимается компонент, который полностью представлен на \textit{SC-коде}.}
				\scnsuperset{многократно используемый компонент базы знаний}
				\scnsuperset{платформенно-независимый scp-агент}
				\scnsuperset{scp-программа}
				\scntext{примечание}{В случае \textit{неатомарного многократно используемого компонента} платформенная независимость означает, что \uline{все} более простые компоненты, входящие в его состав также обязаны быть платформенно-независимыми многократно используемыми компонентами ostis-систем.}
				\scntext{примечание}{Процесс интеграции платформенно-зависимого многократно используемого компонента ostis-систем в дочернюю систему, разработанную по Технологии OSTIS, существенно упрощается за счет использования общей унифицированной формальной основы представления и обработки знаний.}
				\scntext{примечание}{Наиболее ценными являются платформенно-независимые многократно используемые компоненты ostis-систем.}
			\end{scnindent}
		\end{scneqtoset}
	\end{scnindent}
	\scnrelfrom{разбиение}{\scnkeyword{Типология компонентов ostis-систем по динамичности их установки\scnsupergroupsign}}
	\begin{scnindent}
		\scnsubset{класс многократно используемого компонента ostis-систем\scnsupergroupsign}
		\begin{scneqtoset}
			\scnitem{динамически устанавливаемый многократно используемый компонент ostis-систем}
			\begin{scnindent}
				\scnidtf{многократно используемый компонент, при установке которого система не требует перезапуска}
				\begin{scnrelfromset}{декомпозиция}
					\scnitem{многократно используемый компонент, хранящийся в виде скомпилированных файлов}
					\scnitem{многократно используемый компонент базы знаний}
				\end{scnrelfromset}
			\end{scnindent}
			\scnitem{многократно используемый компонент, при установке которого система требует перезапуска}
		\end{scneqtoset}
		\scntext{примечание}{Процесс интеграции компонентов разных видов на разных этапах жизненного цикла osits-систем бывает разным. Наиболее ценными являются компоненты, которые могут быть интегрированы в работающую систему \uline{без} прекращения её функционирования. Некоторые системы, особенно системы управления, нельзя останавливать, а устанавливать и обновлять компоненты нужно.}
	\end{scnindent}
	\scnsuperset{встраиваемая ostis-систем}
	\begin{scnindent}
		\scnidtf{типовая подсистема ostis-систем}
		\scnsubset{ostis-система}
		\scnsubset{неатомарный многократно используемый компонент ostis-систем}
		\scntext{пояснение}{\scnkeyword{встраиваемая ostis-система} --- это \textit{неатомарный многократно используемый компонент}, который состоит из \textit{базы знаний}, \textit{решателя задач} и \textit{интерфейса}.}
		\begin{scnrelfromset}{декомпозиция}
			\scnitem{многократно используемый компонент базы знаний ostis-систем}
			\scnitem{многократно используемый компонент решателей задач ostis-систем}
			\scnitem{многократно используемый компонент интерфейсов ostis-систем}
		\end{scnrelfromset}
		\scniselement{класс многократно используемого компонента ostis-систем\scnsupergroupsign}
		\scnhaselement{Среда коллективной разработки баз знаний ostis-систем}
		\scnhaselement{Визуальный web-ориентированный редактор sc.g-текстов}
		\scnhaselement{Естественно-языковой интерфейс ostis-системы}
		\scnsuperset{менеджер многократно используемых компонентов ostis-систем}
		\scnsuperset{интеллектуальная обучающая ostis-система}
		\scnsuperset{система тестирования и верификации ostis-систем}
		\scntext{примечание}{Особенность \textit{встраиваемых ostis-систем} в том, что интеграция целых интеллектуальных систем предполагает интеграцию баз знаний этих систем, интеграцию их решателей задач и интеграцию их интеллектуальных интерфейсов. При интеграции встраиваемых ostis-систем база знаний встраиваемой системы становится частью базы знаний системы, в которую она встраивается. Решатель задач встраиваемой ostis-системы становится частью решателя задач системы, в которую она встраивается. И интерфейс встраиваемой ostis-системы становится частью интерфейса системы, в которую она встраивается. При этом встраиваемая система является целостной и может функционировать отдельно от других ostis-систем, в отличие от других многократно используемых компонентов.}
		\scntext{примечание}{\textit{встраиваемые ostis-системы} зачастую являются предметно-независимыми многократно используемыми компонентами. Таким образом, например, встраиваемая ostis-система в виде среды проектирования баз знаний может быть встроена как в систему из предметной области по геометрии, так и в систему управления аграрными объектами.}
		\scntext{примечание}{\textit{встраиваемая ostis-система}, как и любой другой многократно используемый компонент ostis-систем, должна поддерживать семантическую совместимость ostis-систем. Как сама встраиваемая ostis-система, так и все ее компоненты должны быть специфицированы и согласованы. Компоненты встраиваемых ostis-систем могут быть заменены на другие, имеющие то же назначение, например, естественно-языковой интерфейс может иметь различные варианты базы знаний в зависимости от естественного языка, поддерживаемого системой, различные варианты интерфейса, в зависимости от требований и удобства пользователей и также различные варианты реализации решателя задач для обработки естественного языка, которые могут использовать различные модели, однако решать одну и ту же задачу. Встраиваемая ostis-система связывается с системой, в которую она встроена с помощью отношения \textbf{\textit{встроенная ostis-система*}}, которое является подмножеством отношения \textit{встроенная кибернетическая система*}.}
	\end{scnindent}
	
		\bigskip
	\end{scnsubstruct}
	\scnsourcecomment{Завершили \scnqqi{Сегмент. Понятие многократно используемого компонента ostis-систем}}
\end{SCn}
\begin{SCn}
	\scnsectionheader{Сегмент. Уточнение спецификации многократно используемого компонента ostis-систем}
	
	\begin{scnsubstruct}
		
	\scnheader{спецификация многократно используемого компонента ostis-систем}
	\scnsubset{спецификация}
	\scnidtf{описание многократно используемого компонента ostis-систем}
	\scnrelfrom{ключевой sc-элемент}{многократно используемый компонент ostis-систем}
	\scntext{примечание}{Каждый \textit{многократно используемый компонент ostis-систем} должен быть специфицирован в рамках библиотеки. Данные спецификации включают в себя основные знания о компоненте, которые позволяют обеспечить построение полной иерархии компонентов и их зависимостей, а также обеспечивают \uline{беспрепятственную} интеграцию компонентов в \scnkeyword{дочерние ostis-системы}. Для спецификации компонента используются как отношения, так и классы компонента.}
	\begin{scnindent}
		\begin{scnrelfromlist}{источник}
			\scnitem{\cite{Orlov2022a}}
			\scnitem{\cite{Davydenko2013}}
		\end{scnrelfromlist}
	\end{scnindent}
	\begin{scnindent}
		\scntext{примечание}{Указание класса \scnkeyword{многократно используемый компонент ostis-систем} является обязательным.}
	\end{scnindent}
	\scntext{примечание}{Сам многократно используемый компонент в рамках спецификации является \textit{ключевым sc-элементом\scnrolesign}, а также может иметь множество своих ключевых sc-элементов.}
	\scnrelfrom{параметры, специфицирующие многократно используемый компонент ostis-систем}{параметр, заданный на многократно используемых компонентах ostis-систем\scnsupergroupsign}
	\scnrelfrom{классы отношений, специфицирующие многократно используемый компонент ostis-систем}{отношение, специфицирующее многократно используемый компонент ostis-систем\scnsupergroupsign}
	\scnrelfrom{описание примера}{\scnfileimage[40em]{Contents/part_methods_tools/src/images/sd_ostis_library/component_specification_example.png}}
	\scnheader{отношение, специфицирующее многократно используемый компонент ostis-систем\scnsupergroupsign}
	\scnidtf{отношение, которое используется при спецификации многократно используемого компонента ostis-систем}
	\begin{scnrelfromset}{разбиение}
		\scnitem{необходимое для установки отношение, специфицирующее многократно используемый компонент ostis-систем}
		\begin{scnindent}
			\scntext{примечание}{Чтобы многократно используемый компонент мог быть принят в библиотеку, нужно специфицировать его, используя каждое отношение из множества \textit{необходимое для установки отношение, специфицирующее многократно используемый компонент ostis-систем}. Здесь описана спецификация, общая для любых типов компонентов, однако в зависимости от типа компонента, спецификация может расширяться.}
			\scnhaselement{метод установки*}
			\scnhaselement{адрес хранилища*}
			\scnhaselement{зависимости компонента*}
		\end{scnindent}
		\scnitem{необязательное для установки отношение, специфицирующее многократно используемый компонент ostis-систем}
		\begin{scnindent}
			\scntext{примечание}{\textit{необязательное для установки отношение, специфицирующее многократно используемый компонент ostis-систем} помогает лучше понять суть компонента, упрощает поиск, но не является обязательным для того, чтобы компонент мог быть установлен в ostis-систему.}
			\scnhaselement{сопутствующие компоненты*}
			\scnhaselement{история изменений*}
			\scnhaselement{модификации компонентов*}
			\scnhaselement{авторы*}
			\scnhaselement{примечание*}
			\scnhaselement{пояснение*}
			\scnhaselement{идентификатор*}
			\scnhaselement{ключевой sc-элемент*}
			\scnhaselement{назначение*}
			\scnhaselement{требования полноты*}
			\scnhaselement{требования безошибочности*}
			\scnhaselement{преимущества*}
			\scnhaselement{недостатки*}
		\end{scnindent}
	\end{scnrelfromset}
	\scnheader{метод установки*}
	\scniselement{бинарное отношение}
	\scniselement{ориентированное отношение}
	\scntext{пояснение}{Пользователь может установить компонент вручную, а \scnkeyword{менеджер компонентов} - автоматически.}
	\scnrelfrom{первый домен}{многократно используемый компонент ostis-систем}
	\scnrelfrom{второй домен}{метод установки многократно используемого компонента}
	\begin{scnindent}
		\scnsubset{метод}
		\scnsuperset{метод установки динамически устанавливаемого многократно используемого компонента ostis-систем}
		\begin{scnindent}
			\scntext{примечание}{При динамической установке необходимо только скачать компонент и, при необходимости, его зависимые компоненты, и он сразу же функционирует в системе.}
			\scnrelfrom{описание примера}{\scnfileimage[35em]{Contents/part_methods_tools/src/images/sd_ostis_library/install_dynamic_method.png}}
			\begin{scnindent}
				\scniselement{sc.g-текст}
			\end{scnindent}
		\end{scnindent}
		\scnsuperset{метод установки многократно используемого компонента, при установке которого система требует перезапуска}
		\begin{scnindent}
			\scntext{примечание}{Установка таких компонентов происходит путём скачивания компонента и его трансляции в память системы.}
			\scnrelfrom{описание примера}{\scnfileimage[35em]{Contents/part_methods_tools/src/images/sd_ostis_library/install_with_reboot_method.png}}
			\begin{scnindent}
				\scniselement{sc.g-текст}
			\end{scnindent}
		\end{scnindent}
	\end{scnindent}
	\scnheader{адрес хранилища*}
	\scniselement{бинарное отношение}
	\scniselement{ориентированное отношение}
	\scntext{пояснение}{Связки отношения \textit{адрес хранилища*} связывают многократно используемый компонент, хранящийся в виде внешних файлов и файл, содержащий url-адрес многократно используемого компонента ostis-систем.}
	\scnrelfrom{первый домен}{многократно используемый компонент ostis-систем, хранящийся в виде внешних файлов}
	\scnrelfrom{второй домен}{файл, содержащий url-адрес многократно используемого компонента ostis-систем}
	\begin{scnindent}
		\scnsuperset{файл}
		\scnsubset{файл, содержащий url-адрес на GitHub многократно используемого компонента ostis-систем}
		\scnsubset{файл, содержащий url-адрес на Google Drive многократно используемого компонента ostis-систем}
		\scnsubset{файл, содержащий url-адрес на Docker Hub многократно используемого компонента ostis-систем}
	\end{scnindent}
	\scnheader{зависимости компонента*}
	\scniselement{квазибинарное отношение}
	\scniselement{ориентированное отношение}
	\scntext{пояснение}{Связки отношения \textit{зависимости компонента*} связывают многократно используемый компонент, и множество компонентов, без которых устанавливаемый компонент \uline{не может быть} встроен в \scnkeyword{дочернюю ostis-систему}.}
	\scnrelfrom{первый домен}{многократно используемый компонент ostis-систем}
	\scnrelfrom{второй домен}{множество многократно используемых компонентов ostis-систем}
	\scnheader{сопутствующие компоненты*}
	\scniselement{квазибинарное отношение}
	\scniselement{ориентированное отношение}
	\scntext{пояснение}{В некоторых случаях может оказаться, что для использования одного многократно используемого компонента OSTIS целесообразно или даже необходимо дополнительно использовать несколько других \textit{многократно используемых компонентов OSTIS}. Например, может оказаться целесообразным вместе с каким-либо sc-агентом информационного поиска использовать соответствующую команду интерфейса, которая представлена отдельным компонентом и позволит пользователю задавать вопрос для указанного sc-агента через интерфейс системы. В таких случаях для связи компонентов используется отношение \textit{сопутствующие компоненты*}. Наличие таких связей позволяет устранить возможные проблемы неполноты знаний и навыков в дочерней системе, из-за которых какие-либо из компонентов могут не выполнять свои функции. Связки отношения \textit{сопутствующий компонент*} связывают многократно используемые компоненты ostis-систем, которые целесообразно использовать в дочерней системе вместе. Каждая такая связка может дополнительно быть снабжена sc-комментарием или sc-пояснением, отражающим суть указываемой зависимости.}
	\scnrelfrom{первый домен}{многократно используемый компонент ostis-систем}
	\scnrelfrom{второй домен}{множество многократно используемых компонентов ostis-систем}
	\scnheader{история изменений*}
	\scniselement{бинарное отношение}
	\scniselement{ориентированное отношение}
	\scntext{пояснение}{Отношение \textit{история изменений*} позволяет специфицировать различные версии компонента и, при необходимости, устанавливать выбранную пользователем версию. Различные версии, как правило, отражают какие-либо улучшения или исправления ошибок.}
	\scnrelfrom{первый домен}{многократно используемый компонент ostis-систем}
	\scnrelfrom{второй домен}{история изменений}
	\scnheader{модификации компонентов*}
	\scniselement{бинарное отношение}
	\scniselement{ориентированное отношение}
	\scntext{пояснение}{\textit{модификации компонентов*} --- это функционально эквивалентные реализации одного и того же компонента, которые могут быть синтаксически эквивалентны (например, реализация одного и того же sc-агента на платформенно-зависимом и платформенно-независимом уровнях). Развитие \textit{Библиотеки Экосистемы OSTIS} происходит не только за счет ее пополнения новыми компонентами, но и за счет появления новых версий и модификаций уже существующих компонентов.}
	\scnrelfrom{первый домен}{многократно используемый компонент ostis-систем}
	\scnrelfrom{второй домен}{множество многократно используемых компонентов ostis-систем}
	\scnheader{авторы*}
	\scntext{пояснение}{Связки отношения \textit{авторы*} связывают многократно используемый компонент со множеством авторов этого компонента. Спецификация может также содержать дополнительную информацию об авторах при необходимости.}
	\scnheader{назначение*}
	\scntext{пояснение}{Отношение \textit{назначение*} позволяет описать ожидаемый сценарий, выделить рекомендации использования многократно используемого компонента. Требования полноты и безошибочности специфицируют возможные ограничения и ошибки компонента, область его использования.}
	
	\scnheader{параметр, заданный на многократно используемых компонентах ostis-систем\scnsupergroupsign}
	\scntext{примечание}{Для уточнения типа компонента могут использоваться другие классы, например дата публикации первой и последней версии компонента, качественно-количественные характеристики, такие как уровень семантической совместимости компонентов, сложность реализации компонента, уровень производительности компонента (для программ можно использовать O-нотацию), количество sc-элементов, входящих в состав многократно используемого компонента, количество ключевых узлов компонента, рейтинг компонента в рамках \textit{Экосистемы OSTIS}, количество скачиваний компонента и другие. Параметр \textit{лицензия многократно используемого компонента} используется для обозначения условий использования и распространения компонента. По умолчанию лицензия компонента открытая, если не указано иное.}
	\begin{scnindent}
		\scnrelfrom{источник}{\cite{Davydenko2013}}
	\end{scnindent}
		
		\bigskip
	\end{scnsubstruct}
	\scnsourcecomment{Завершили \scnqqi{Сегмент. Уточнение спецификации многократно используемого компонента ostis-систем}}
\end{SCn}
\begin{SCn}
\scnsectionheader{Сегмент. Понятие менеджера многократно используемых компонентов ostis-систем}

\begin{scnsubstruct}
		
\scnheader{хранилище многократно используемых компонентов ostis-систем, хранящихся в виде внешних файлов}
\scntext{примечание}{Для того, чтобы хранить \textit{многократно используемые компоненты ostis-систем}, необходимо некоторое хранилище. Таким хранилищем может выступать как какая-либо ostis-система, так и стороннее хранилище, например, облачное. Помимо исходных файлов компонента в хранилище должна находиться его \uline{спецификация}.}
\scnsuperset{хранилище многократно используемого компонента ostis-систем, хранящегося в виде файлов исходных текстов}
\begin{scnindent}
	\scntext{пояснение}{Место хранения файлов исходных текстов многократно используемого компонента.}
	\scnsuperset{хранилище на основе системы контроля версий Git}
	\begin{scnindent}
		\scnsuperset{репозиторий GitHub}
		\scntext{примечание}{На данном этапе в рамках \textit{Технологии OSTIS} (в силу открытости технологии, а также хранения компонентов в виде файлов исходных текстов) для хранения компонентов чаще всего используются хранилища на основе системы контроля версий Git.}
	\end{scnindent}
\end{scnindent}
\scnsuperset{хранилище многократно используемого компонента ostis-систем, хранящегося в виде скомпилированных файлов}
\begin{scnindent}
	\scntext{пояснение}{Место хранения скомпилированных файлов многократно используемого компонента.}
\end{scnindent}
\scntext{примечание}{Помимо внешних файлов компонента в хранилище должна находиться его \uline{спецификация}.}

\scnheader{менеджер многократно используемых компонентов ostis-систем}
\scnidtftext{часто используемый sc-идентификатор}{менеджер многократно используемых компонентов}
\scnidtftext{часто используемый sc-идентификатор}{менеджер компонентов}
\scnsubset{платформенно-зависимый многократно используемый компонент ostis-систем}
\scntext{пояснение}{менеджер многократно используемых компонентов ostis-систем --- подсистема ostis-системы, с помощью которой происходит взаимодействие с библиотекой компонентов ostis-систем.}
\scnhaselement{Реализация менеджера многократно используемых компонентов ostis-систем}
\begin{scnindent}
	\scntext{адрес компонента}{https://github.com/ostis-ai/sc-component-manager}
\end{scnindent}
\begin{scnrelfromset}{обобщенная декомпозиция}
	\scnitem{база знаний менеджера многократно используемых компонентов ostis-систем}
	\begin{scnindent}
		\scntext{примечание}{база знаний менеджера компонентов содержит все те знания, которые необходимы для установки многократно используемого компонента в \scnkeyword{дочернюю ostis-систему}. К таким знаниям относятся знания о спецификации многократно используемых компонентов, методы установки компонентов, знание о библиотеках ostis-систем, с которыми происходит взаимодействие, \textit{классификация компонентов} и другие.}
	\end{scnindent}
	\scnitem{решатель задач менеджера многократно используемых компонентов ostis-систем}
	\begin{scnindent}
		\scntext{примечание}{решатель задач менеджера компонентов взаимодействует с библиотекой ostis-систем и позволяет установить и интегрировать многократно используемые компоненты в \scnkeyword{дочернюю ostis-систему}, также выполнять поиск, обновление, публикацию, удаление компонентов и другие операции с ними.}
		\begin{scnrelfromset}{декомпозиция абстрактного sc-агента}
			\scnitem{Абстрактный sc-агент поиска многократно используемых компонентов ostis-систем}
			\scnitem{Абстрактный sc-агент установки многократно используемых компонентов ostis-систем}
			\scnitem{Абстрактный sc-агент управления отслеживаемых менеджером компонентов библиотек}
			\begin{scnindent}
				\begin{scnrelfromset}{декомпозиция абстрактного sc-агента}
					\scnitem{Абстрактный sc-агент добавления отслеживаемой менеджером компонентов библиотеки}
					\scnitem{Абстрактный sc-агент удаления отслеживаемой менеджером компонентов библиотеки}
				\end{scnrelfromset}
			\end{scnindent}
		\end{scnrelfromset}
	\end{scnindent}
	\scnitem{интерфейс менеджера многократно используемых компонентов ostis-систем}
	\begin{scnindent}
		\scntext{примечание}{интерфейс менеджера многократно используемых компонентов обеспечивает удобное для пользователя и других систем использование менеджера компонентов.}
	\end{scnindent}
\end{scnrelfromset}
\scnrelfrom{минимальные функциональные возможности}{Минимальные функциональные возможности менеджера компонентов}
\begin{scnindent}
	\scntext{примечание}{Используя минимальные функциональные возможности, менеджер компонентов может установить компоненты, которые будут расширять его же функционал.}
	\begin{scneqtoset}
		\scnfileitem{\textbf{Поиск многократно используемых компонентов ostis-систем.} Множество возможных критериев поиска соответствует спецификации многократно используемых компонентов. Такими критериями могут быть классы компонента, его авторы, идентификатор, фрагмент примечания, назначение, принадлежность какой-либо предметной области, вид знаний компонента и другие.}
		\scnfileitem{\textbf{Установка многократно используемого компонента ostis-систем.} Установка многократно используемого компонента происходит вне зависимости от типологии, способа установки и местоположения компонента. Необходимое условие для возможности установки многократно используемого компонента --- наличие \textbf{\textit{спецификации многократно используемого компонента ostis-систем}}. Перед установкой многократно используемого компонента в дочернюю систему необходимо установить все зависимые компоненты. Также для платформенно-зависимых компонентов может быть необходимо установить иные зависимости, которые не являются компонентами какой-либо библиотеки ostis-систем. После успешной установки компонента в базе знаний дочерней системы генерируется информационная конструкция, обозначающая факт установки компонента в систему с помощью отношения \textit{установленные компоненты*}.}
		\scnfileitem{\textbf{Добавление и удаление отслеживаемых менеджером компонентов библиотек.} Менеджер компонентов содержит информацию о множестве источников для установки компонентов, перечень которых можно дополнять. По умолчанию менеджер компонентов отслеживает \textit{Библиотеку Метасистемы OSTIS}, однако можно создавать и добавлять дополнительные библиотеки ostis-систем.}
	\end{scneqtoset}
\end{scnindent}
\scnrelfrom{расширенные функциональные возможности}{Расширенные функциональные возможности менеджера компонентов}
\begin{scnindent}
	\scntext{примечание}{Компоненты, реализующие расширенный функционал менеджера компонентов являются частью \textit{Библиотеки Метасистемы OSTIS}.}
	\begin{scneqtoset}
		\scnfileitem{\textbf{Спецификация} многократно используемого компонента ostis-систем. Менеджер компонентов позволяет специфицировать компоненты, входящие в состав библиотеки ostis-систем, а также специфицировать новые создаваемые компоненты, которые будут публиковаться в библиотеку ostis-систем. При этом спецификация может происходить как автоматически, так и вручную. Например, менеджер компонентов может обновить спецификацию используемого компонента в соответствии с тем, в какие новые ostis-системы его установили, обновить спецификацию авторства компонента при его редактировании в библиотеке ostis-систем, спецификацию ошибок, выявленных в процессе эксплуатации компонента и так далее.}
		\scnfileitem{\textbf{Формирование} многократно используемого компонента ostis-систем по шаблону с заданными параметрами. При установке шаблона многократно используемого компонента ostis-систем менеджер компонентов позволяет сформировать по нему конкретный компонент. Для этого пользователю предлагается определить значения всех sc-переменных в шаблоне для формирования конкретного компонента из некоторой предметной области. Например, для формирования многократно используемого компонента баз знаний, представляющего собой семантическую окрестность некоторого отношения, нужно определить значения всех переменных, кроме переменной, являющейся ключевым sc-элементом данной структуры.}
		\begin{scnindent}
			\scnrelfrom{описание примера}{\scnfileimage[30em]{Contents/part_methods_tools/src/images/sd_ostis_library/relation_template.png}}
			\begin{scnindent}
				\scntext{пояснение}{Пример шаблона многократно используемого компонента ostis-систем.}
			\end{scnindent}
		\end{scnindent}
		\scnfileitem{\textbf{Публикация} многократно используемого компонента ostis-систем в библиотеку ostis-систем. При публикации компонента в библиотеку ostis-систем происходит верификация на основе спецификации компонента. Публикация компонента может сопровождаться сборкой неатомарного компонента из существующих атомарных. Также существует возможность обновления версии опубликованного компонента сообществом его разработчиков.}
		\scnfileitem{\textbf{Обновление} установленного многократно используемого компонента ostis-систем.}
		\scnfileitem{\textbf{Удаление} установленного многократно используемого компонента. Как и в случае установки после удаления многократно используемого компонента из ostis-системы в базе знаний системы устанавливается факт удаления компонента. Эта информация является важной частью \uline{истории эксплуатации} ostis-системы.}
		\scnfileitem{\textbf{Редактирование} многократно используемого компонента в библиотеке ostis-систем.}
		\scnfileitem{\textbf{Сравнение} многократно используемых компонентов ostis-систем.}
	\end{scneqtoset}
\end{scnindent}
\scntext{примечание}{Для того, чтобы создать новую ostis-систему \scnqq{с нуля}, используя \textit{ostis-платформу}, необходимо установить некоторый \textit{Программный вариант реализации ostis-платформы} с помощью сторонних средств. Такими средствами могут быть (1) хранилища исходного кода платформы, например, облачные хранилища, такие как GitHub репозиторий, с соответствующим набором инструкций по установке платформы или (2) средства установки заранее скомпилированной программной реализации платформы, например, средство установки программного обеспечения apt. Далее установка многократно используемых компонентов в ostis-систему (независимо от типа компонентов) осуществляется с помощью менеджера компонентов. При установке платформенно-зависимых компонентов менеджер компонентов должен управлять соответствующими средствами сборки таких компонентов (CMake, Ninja, npm, grunt и другие).}
\scntext{примечание}{Компонент находится в некотором хранилище --- (1) \textit{библиотеке компонентов ostis-систем} или (2) в виде файлов в некотором облачном хранилище. В случае, когда компонент хранится в библиотеке, для его установки менеджер компонентов копирует все sc-элементы, которые представляют собой компонент, в дочернюю ostis-систему. В случае, когда компонент хранится в виде файлов в облачном хранилище, менеджер компонентов скачивает файлы компонента и устанавливает их в соответствии со спецификацией. Адреса хранилищ спецификаций компонентов должны храниться в базе знаний менеджера компонентов, чтобы иметь доступ к спецификациям компонентов для их последующего использования (поиска, установки и так далее).}
\scntext{примечание}{\textit{менеджер многократно используемых компонентов ostis-систем} является \uline{необязательной} подсистемой \textit{ostis-платформы}. Однако система, имеющая менеджер компонентов, может устанавливать компоненты не только сама в себя, но и в другие системы при наличии доступа. Таким образом, одна система может заменить \textit{ostis-платформу} другой системы, оставив при этом \textit{sc-модель кибернетической системы}. Таким же образом некоторая ostis-система может порождать другие ostis-системы, используя компонентный подход.}
\scntext{примечание}{Включение компонента в \textit{дочернюю ostis-систему} в общем случае состоит из следующих этапов:
	\begin{itemize}
		\item поиск подходящего компонента во множестве доступных библиотек;
		\item выделение компонента в виде, удобном для транспортировки в дочернюю ostis-систему с указанием версии и модификации при необходимости (например, выбор доступного хранилища компонента, выбор оптимального варианта реализации компонента с учетом состава дочерней системы);
		\item установка многократно используемого компонента и его зависимостей (с указанием версии и модификации при необходимости);
		\item интеграция компонента в дочернюю систему;
		\item поиск и устранение ошибок и противоречий в дочерней системе.
\end{itemize}}

\scnheader{установленные компоненты*}
\scniselement{квазибинарное отношение}
\scniselement{ориентированное отношение}
\scntext{пояснение}{Квазибинарное отношение, связывающее некоторую ostis-систему и компоненты, которые установлены в ней.}
\scnrelfrom{первый домен}{ostis-система}
\scnrelfrom{второй домен}{множество многократно используемых компонентов ostis-систем}
\begin{scnindent}
	\scntext{пояснение}{множество многократно используемых компонентов ostis-систем --- это множество, все элементы которого являются многократно используемыми компонентами ostis-систем.}
\end{scnindent}
\scntext{примечание}{Данное отношение позволяет хранить сведения о системах и компонентах, которые установлены в них, тем самым предоставляя возможность анализировать функциональные возможности системы.}
\scntext{примечание}{Данное отношение позволяет оценивать частоту скачивания компонентов, то есть их использования в \scnkeyword{дочерних ostis-системах}.}
	
	\bigskip
\end{scnsubstruct}
\scnsourcecomment{Завершили \scnqqi{Сегмент. Понятие менеджера многократно используемых компонентов ostis-систем}}
\end{SCn}   
\begin{SCn}
\scnsectionheader{Сегмент. Заключение в Предметную область и онтологию комплексной библиотеки многократно используемых семантически совместимых компонентов ostis-систем}
	
\begin{scnsubstruct}
	\scnheader{компонентное проектирование интеллектуальных компьютерных систем}  
	\scntext{примечание}{Компонентный подход является ключевым в технологии проектирования интеллектуальных компьютерных систем. Вместе с этим, технология компонентного проектирования тесно связана с остальными составляющими \textit{технологии проектирования интеллектуальных компьютерных систем} и обеспечивает их совместимость, производя мощнейший синергетический эффект при использовании всего комплекса частных технологий проектирования интеллектуальных систем. Важнейшим принципом в реализации компонентного подхода является семантическая совместимость многократно используемых компонентов, что позволяет минимизировать участие программистов в создании новых компьютерных систем и в совершенствовании существующих компьютерных систем.}
	\scntext{примечание}{Для реализации компонентного подхода предлагается \textit{библиотека многократно используемых совместимых компонентов интеллектуальных компьютерных систем на основе Технологии OSTIS}, введена классификация и спецификация многократно используемых компонентов ostis-систем, предложена модель менеджера компонентов, позволяющего взаимодействовать \textit{ostis-системам} с \textit{библиотеками многократно используемых компонентов} и управлять компонентами в системе, рассмотрена архитектура экосистемы \textit{интеллектуальных компьютерных систем} с точки зрения использования библиотеки многократно используемых компонентов.}
	\scntext{примечание}{Полученные результаты позволят повысить эффективность проектирования интеллектуальных систем и средств автоматизации разработки таких систем, а также обеспечить возможность не только разработчику, но и интеллектуальной системе автоматически дополнять систему новыми знаниями и навыками.}
		\bigskip
\end{scnsubstruct}
\scnsourcecomment{Завершили \scnqqi{Сегмент. Заключение в Предметную область и онтологию комплексной библиотеки многократно используемых семантически совместимых компонентов ostis-систем}}
\end{SCn}
        
        \bigskip
    \end{scnsubstruct}
\end{SCn}


\scsubsection[
    \protect\scneditor{Банцевич К.А.}
    \protect\scnmonographychapter{Глава 5.1. Комплексная библиотека многократно используемых семантически совместимых компонентов интеллектуальных компьютерных систем нового поколения}
    ]{Предметная область и онтология многократно используемых компонентов баз знаний ostis-систем}
\label{sd_know_base_component}
\begin{SCn}
    \scnsectionheader{Предметная область и онтология многократно используемых компонентов баз знаний ostis-систем}
    \begin{scnsubstruct}
	    \scntext{аннотация}{Для широкого применения интеллектуальных систем, способных повысить качество решения прикладных задач, разработано большое число \textit{баз знаний} по самым различным предметным областям. Однако в большинстве случаев каждая база знаний разрабатывается отдельно и независимо от других, в отсутствие единой унифицированной формальной основы для представления знаний, а также единых принципов формирования систем понятий для описываемой предметной области. В связи с этим разработанные базы оказываются, как правило, несовместимы между собой и не пригодны для повторного использования. Для быстрой разработки достаточного количества баз знаний, кроме наличия средств разработки интеллектуальных систем, обеспечивающих разработку и проектирование различных компонентов интеллектуальной системы, включая базу знаний, требуется наличие соответствующей отлаженной технологии проектирования баз знаний.}
	    \begin{scnindent}
	        \begin{scnrelfromlist}{источник}
				\scnitem{Ivashenko2011}
			\end{scnrelfromlist}
		\end{scnindent}
    	
    	
        \scnheader{Предметная область многократно используемых компонентов баз знаний ostis-систем}
        \scniselement{предметная область}
        \scnrelto{частная предметная область}{Предметная область многократно используемых компонентов ostis-систем}
        \begin{scnhaselementrolelist}{класс объектов исследования}
            \scnitem{многократно используемый компонент баз знаний ostis-систем}
        \end{scnhaselementrolelist}
        \scnhaselementrole{класс объектов исследования}{отношение, специфицирующее многократно используемый компонент баз знаний ostis-систем}
        \begin{scnrelfromlist}{ключевое понятие}
        	\scnitem{компонентное проектирование баз знаний интеллектуальных систем}
        \end{scnrelfromlist}
    
        \begin{scnrelfromlist}{библиографическая ссылка}
    		\scnitem{Ivashenko2011}
    		\scnitem{Golenkov2013}
    		\scnitem{Davydenko2013}
    	\end{scnrelfromlist}
        
        \scnheader{многократно используемый компонент баз знаний}
        \scnsuperset{предметная область и онтология}
        \scnsubset{раздел базы знаний}
        \scnsuperset{семантическая окрестность}
        \scnrelfrom{смотрите}{\nameref{sd_sem_neigh}}
        \scnsuperset{базовые фрагменты предметных областей и онтологий}
        \scntext{примечание}{Базовый фрагмент предметной области и онтологии включает в себя теоретико-множественную и логическую онтологию, а также фрагменты терминологической онтологии, описывающие основные идентификаторы объектов исследования предметной области.}
        \scntext{примечание}{Данный вид многократно используемых компонентов позволяет использовать только те знания, которые непосредственно необходимы для функционирования интеллектуальных систем, исключив то, что никак не влияет на работу конечной системы (пояснения, примеры, дидактический материал и т.д.).}
        \scnhaselement{Базовый фрагмент теории логических формул, высказываний и логических sc-языков}
        \scnhaselement{Базовый фрагмент теории множеств}
        \scnhaselement{Базовый фрагмент теории связок и отношений}
        \scnsuperset{база знаний прикладной ostis-системы}
        \scntext{примечание}{Целые базы знаний могут быть многократно используемыми компонентами в случае разработки интеллектуальных систем, имеющих схожие функциональные требования.}
        
        \scnheader{многократно используемый компонент базы знаний}
        \scnhaselement{Расширенное ядро базы знаний}
        \scnhaselement{Ядро базы знаний}
        \scntext{пояснение}{\textit{Ядро базы знаний} представляет собой компонент, входящий в состав каждой базы знаний, разрабатываемой по \textit{Технологии OSTIS}, и устанавливаемый в первую очередь.}
        
        \scnheader{отношение, специфицирующее многократно используемый компонент баз знаний ostis-систем}
        \scnsubset{отношение, специфицирующее многократно используемый компонент ostis-систем}
        \scnhaselement{максимальный класс объектов исследования\scnrolesign}
        \scnhaselement{немаксимальный класс объектов исследования\scnrolesign}
        \scnhaselement{исследуемое отношение\scnrolesign}
        
        \scnnote{Компонентный подход к разработке интеллектуальных компьютерных систем, реализуемый в виде \textbf{\textit{библиотеки многократно используемых компонентов ostis-систем}}, позволяет решить описанные проблемы. \textit{Библиотека многократно используемых компонентов баз знаний ostis-систем в составе Метасистемы OSTIS} является важнейшим фрагментом \textit{Метасистемы OSTIS}, который обеспечивает надежность и совместимость проектируемых фрагментов баз знаний, а также повышение скорости разработки \textit{баз знаний} \textit{интеллектуальных компьютерных систем}.
        
        Описываемая библиотека включает в себя множество компонентов баз знаний и их спецификаций.}
        
        \scnheader{многократно используемый компонент баз знаний ostis-систем}
        \scnsuperset{предметная область и онтология}
        \begin{scnindent}
        	\scnsubset{раздел базы знаний}
        \end{scnindent}
        \scnsuperset{семантическая окрестность}
        \begin{scnindent}
        	\scnsuperset{семантическая окрестность по инцидентным коннекторам}
        	\scnsuperset{полная семантическая окрестность}
        	\scnsuperset{базовая семантическая окрестность}
        	\scnsuperset{специализированная семантическая окрестность}
        \end{scnindent}
        \scnsuperset{базовые фрагменты предметных областей и онтологий}
        \begin{scnindent}
        	\scnnote{Базовый фрагмент предметной области и онтологии включает в себя теоретико-множественную, логическую онтологии, а также терминологические фрагменты.}
        	\scnnote{Данный вид многократно используемых компонентов позволяет использовать только те знания, которые непосредственно необходимы для функционирования интеллектуальных систем, исключив то, что никак не влияет на работу конечной системы (пояснения, примеры, дидактический материал и так далее).}
        	\scnhaselement{Базовый фрагмент теории логических формул, высказываний и логических sc-языков}
        	\scnhaselement{Базовый фрагмент теории множеств}
        	\scnhaselement{Базовый фрагмент теории связок и отношений}
        \end{scnindent}
        \scnsuperset{база знаний}
        \begin{scnindent}
        	\scnnote{Целые базы знаний могут быть многократно используемыми компонентами в случае разработки интеллектуальных систем, назначение которых совпадает.}
        \end{scnindent}
		\scntext{примечание}{Список приведенных классов многократно используемых компонентов не является окончательным. В случае, когда разработчик базы знаний интеллектуальной системы считает, что разработанный им компонент сможет стать неотъемлемой частью библиотеки, то компонент будет добавлен в библиотеку, как многократно используемый, в случае, если компонент прошел верификацию и соответствует требованиям разработчиков библиотеки.}
        
        \scnheader{онтология предметной области}
        \begin{scnindent}
        	\scnnote{\textit{онтологии предметных областей}, описывающие виды знаний, которые являются основой для построения \textit{базы знаний} любой \textit{интеллектуальной системы}, входят в \textit{Ядро базы знаний}, поскольку являются \textit{онтологиями верхнего уровня}. Следовательно, \textit{Ядро базы знаний} представляет собой компонент, входящий в состав каждой базы знаний, разрабатываемой по \textit{Технологии OSTIS}, и устанавливающийся в первую очередь.}
        	\scnnote{\textit{онтологии предметных областей}, которые используются в большинстве интеллектуальных систем, являются частью \textit{Расширенного ядра базы знаний}.}
        \end{scnindent}
        
        \scnheader{многократно используемый компонент баз знаний ostis-систем}
        \scnhaselement{Ядро базы знаний}
        \begin{scnindent}
        	\scnhaselement{Предметная область и онтология множеств}
        	\scnhaselement{Предметная область и онтология связок и отношений}
        	\scnhaselement{Предметная область и онтология структур}
        	\scnhaselement{Предметная область и онтология семантических окрестностей}   
        	\scnhaselement{Предметная область и онтология предметных областей}
        	\scnhaselement{Предметная область и онтология онтологий}
        \end{scnindent}
       
        
        \scnheader{многократно используемый компонент баз знаний ostis-систем}
        \scnhaselement{Расширенное ядро базы знаний}
        \begin{scnindent}
        	\scnrelfrom{включение}{Ядро базы знаний}
        	\scnexplanation{В отличие от \textit{Ядра базы знаний} \textit{Расширенное ядро базы знаний} содержит в себе не только обязательные для установки \textit{онтологии предметных областей}, но и такие \textit{онтологии предметных областей}, которые используются в большинстве \textit{интеллектуальных компьютерных систем}. Следовательно, являются компонентами, которые наиболее часто устанавливаются пользователями \textit{Библиотеки многократно используемых компонентов баз знаний ostis-систем в составе Метасистемы OSTIS}.}
        	\scnhaselement{Предметная область и онтология параметров, величин и шкал}
        	\scnhaselement{Предметная область и онтология чисел и числовых структур}
        	\scnhaselement{Предметная область и онтология темпоральных сущностей}
        	\scnhaselement{Предметная область и онтология пространственных сущностей различных форм}
        	\scnhaselement{Предметная область и онтология материальных сущностей}
        
	        \scnnote{Представленный список \textit{многократно используемых компонентов баз знаний} не является окончательным. В случае, когда разработчик \textit{базы знаний} интеллектуальной системы считает, что разработанный им компонент сможет стать неотъемлемой частью библиотеки, то компонент будет добавлен в библиотеку, как многократно используемый, если:
	        \begin{scnitemize}
	        	\item компонент специфицирован;
	        	\item компонент прошел верификацию и соответствует требованиям разработчиков библиотеки
	        \end{scnitemize}
	    	}
	        
	        \scnnote{Чтобы \textit{многократно используемый компонент баз знаний} мог быть принят в библиотеку, он должен быть корректно специфицирован. Для этого используются отношения класса \textit{необходимое для установки отношение, специфицирующее многократно используемый компонент ostis-систем}, а также \textit{необязательное для установки отношение, специфицирующее многократно используемый компонент ostis-систем}. Однако в зависимости от типа компонента, спецификация может расширяться. Рассмотрим \textit{необязательное для установки отношение, специфицирующее многократно используемый компонент баз знаний ostis-систем}, его поиска и установки в \textit{дочернюю ostis-систему}, если таковым компонентом является \textit{предметная область и онтология}.}
    	\end{scnindent}
        
        \scnheader{необязательное для установки отношение, специфицирующее многократно используемый компонент баз знаний ostis-систем}
        \scnsubset{необязательное для установки отношение, специфицирующее многократно используемый компонент ostis-систем}
        \scnhaselement{максимальный класс объектов исследования'}
        \begin{scnindent}
        	\scnidtf{класс объектов исследования, для которого в заданной предметной области отсутствует другой класс объектов исследования, который был бы его надмножеством'}
        	\scnrelfrom{первый домен}{предметная область и онтология}
        	\scnrelfrom{второй домен}{понятие}
        \end{scnindent}
        \scnhaselement{немаксимальный класс объектов исследования'}
        \begin{scnindent}
        	\scnrelfrom{первый домен}{предметная область и онтология}
        	\scnrelfrom{второй домен}{понятие}
        \end{scnindent}
        \scnhaselement{исследуемое отношение'}
        \begin{scnindent}
        	\scnrelfrom{первый домен}{предметная область и онтология}
        	\scnrelfrom{второй домен}{понятие}
        \end{scnindent}
        
        \scnnote{Для компонентов, которые являются частью \textit{Библиотеки многократно используемых компонентов баз знаний ostis-систем в составе Метасистемы OSTIS}, также существуют средства поиска, обновления.} 
        
        \scnnote{Корректно спроектированные спецификации компонентов позволят построить полную иерархию зависимостей компонентов, а также их структуру, что в свою очередь позволит беспрепятственное использование компонентов и их фрагментов в рамках компонентного проектирования баз знаний.}
        
    \end{scnsubstruct}
	\begin{scnrelfromvector}{заключение}
		\scnfileitem{Проектирование и анализ качества \textit{баз знаний} являются важнейшими этапами разработки \textit{интеллектуальных компьютерных систем}, так как они во многом определяют качество всей интеллектуальной системы.}
		\scnfileitem{Предложенная методология коллективной разработки базы знаний на основе \textit{Технологии OSTIS}, которая включает в себя модель верификации и контроля качества \textit{базы знаний}, а также \textit{компонентный подход} к проектированию \textit{баз знаний}, позволяет повысить эффективность проектирования \textit{интеллектуальных компьютерных систем} и средств автоматизации разработки таких систем}
	\end{scnrelfromvector}
\end{SCn}


\scsubsection[
    \protect\scneditor{Шункевич Д.В.}
    \protect\scnmonographychapter{Глава 5.3. Методика и средства компонентного проектирования решателей задач интеллектуальных компьютерных систем нового поколения}
    ]{Предметная область и онтология многократно используемых компонентов решателей задач ostis-систем}
\label{sd_problem_solver_component}

\scsubsubsection[
    \protect\scneditor{Шункевич Д.В.}
    \protect\scnmonographychapter{Глава 5.3. Методика и средства компонентного проектирования решателей задач интеллектуальных компьютерных систем нового поколения}
    ]{Предметная область и онтология многократно используемых методов, хранимых в памяти ostis-систем и интерпретируемых их внутренними агентами}
\label{sd_method_agent}

\scsubsubsection[
    \protect\scneditors{Шункевич Д.В.;Зотов Н.В.;Орлов М.К.}
    \protect\scnmonographychapter{Глава 5.3. Методика и средства компонентного проектирования решателей задач интеллектуальных компьютерных систем нового поколения}
    ]{Предметная область и онтология многократно используемых внутренних агентов ostis-систем}
\label{sd_internal_agent}

\scsubsection[
    \protect\scneditor{Садовский М.Е.}
    \protect\scnmonographychapter{Глава 5.4. Методика и средства компонентного проектирования интерфейсов ostis-систем}
    ]{Предметная область и онтология многократно используемых компонентов интерфейсов ostis-систем}
\label{sd_component_interface}
\begin{SCn}
	\scnsectionheader{Предметная область и онтология многократно используемых компонентов интерфейсов ostis-систем}
	\scntext{введение}{Большое разнообразие \textit{интерфейсов} влечет за собой разработку большого числа компонентов. В качестве \textit{многократно используемых компонентов интерфейсов ostis-систем} могут выступать как уже спроектированные \textit{интерфейсы}, так и специфицированные \textit{компоненты интерфейсов}. Большое число \textit{многократно используемых компонентов интерфейсов ostis-систем} создает проблему их хранения и поиска. Чтобы решить эту проблему, в технологию включена \textit{библиотека многократно используемых компонентов пользовательских интерфейсов ostis-систем} и \textit{менеджер многократно используемых компонентов ostis-систем}.}
	\begin{scnindent}
		\scnrelfrom{смотрите}{Предметная область и онтология комплексной библиотеки многократно используемых семантически совместимых компонентов ostis-систем}
	\end{scnindent}
	
	\begin{scnsubstruct}
		\scnheader{Предметная область многократно используемых компонентов интерфейсов ostis-систем}
		\scniselement{предметная область}
		\scnhaselementrole{максимальный класс объектов исследования}{многократно используемый компонент пользовательских интерфейсов ostis-систем}
		\begin{scnhaselementrolelist}{класс объектов исследования}
			\scnitem{пользовательский интерфейс библиотеки многократно используемых компонентов интерфейсов ostis-систем}
			\scnitem{просмотрщик содержимого \textit{файлов ostis-системы}}
			\scnitem{редактор содержимого \textit{файлов ostis-системы}}
			\scnitem{транслятор содержимого \textit{файла ostis-системы} в \textit{SC-код}}
			\scnitem{транслятор из \textit{SC-кода} в содержимое \textit{файла ostis-системы}}
			\scnitem{транслятор \textit{базы знаний} во внешнее представление}
		\end{scnhaselementrolelist}
		
		\scnheader{многократно используемый компонент интерфейсов ostis-систем}
		\scnrelfrom{разбиение}{Типология многократно используемых компонентов интерфейсов ostis-систем по признаку решаемых задач}
		\begin{scnindent}
			\begin{scneqtoset}
				\scnitem{просмотрщик содержимого \textit{файлов ostis-системы}}
				\scnitem{редактор содержимого \textit{файлов ostis-системы}}
				\scnitem{транслятор содержимого \textit{файла ostis-системы} в \textit{SC-код}}
				\scnitem{транслятор из \textit{SC-кода} в содержимое \textit{файла ostis-системы}}
				\scnitem{транслятор \textit{базы знаний} во внешнее представление}
				\begin{scnindent}
					\begin{scnrelfromlist}{смотрите}
						\scnitem{\scncite{Koronchik2011}}
						\scnitem{\scncite{Koronchik2014}}
					\end{scnrelfromlist}
				\end{scnindent}
			\end{scneqtoset}
			\scntext{примечание}{\textit{Технология OSTIS} позволяет интегрировать в качестве компонентов редакторы и просмотрщики, разработанные с использованием других технологий (далее их будем называть \textit{платформенно-зависимыми многократно используемыми компонентами интерфейса ostis-систем}). В основном они используются для просмотра и редактирования содержимого \textit{файлов ostis-системы}. Это значительно позволяет сэкономить время при их разработке.}
		\end{scnindent}
		
		\scnheader{пользовательский интерфейс библиотеки многократно используемых компонентов интерфейсов ostis-систем}
		\scntext{примечание}{\textit{пользовательский интерфейс} \textit{библиотеки многократно используемых компонентов интерфейсов ostis-систем} строится на основе sc.g-интерфейса (комплекс информационно-программных средств обеспечивающих общение интеллектуальных систем с пользователями на основе \textit{SCg-кода}, как способа внешнего представления информации). Однако, это не исключает возможность использования других способов диалога пользователя с \textit{библиотекой многократно используемых компонентов интерфейсов ostis-систем}.}
		\scntext{примечание}{В рамках \textit{библиотеки многократно используемых компонентов интерфейсов ostis-систем} могут содержаться различные версии и модификации какого-либо компонента. К примеру, компонент просмотра \textit{sc.g-конструкций} может иметь модификации, для отображения в которых может использоваться двумерная, трехмерная или же многослойная визуализация, при этом каждая из модификаций компонента может иметь различные версии.}
		\scntext{примечание}{Использование \textit{библиотеки многократно используемых компонентов интерфейсов ostis-систем} при проектировании \textit{интерфейса} прикладной системы позволяет значительно сократить сроки проектирования, а также снизить требования, предъявляемые к начальной квалификации разработчика. Это достигается за счет проектирования \textit{интерфейса} из уже заранее подготовленных моделей интерфейса, что также позволяет повысить качество проектируемого \textit{интерфейса}.}
		
		\bigskip
	\end{scnsubstruct}
\end{SCn}


\scsubsection[
    \protect\scneditor{Шункевич Д.В.}
    \protect\scnmonographychapter{Глава 5.1. Комплексная библиотека многократно используемых семантически совместимых компонентов интеллектуальных компьютерных систем нового поколения}
    ]{Предметная область и онтология многократно используемых встраиваемых ostis-систем}
\label{sd_embed_sys}

\scsection[
    \protect\scneditor{Шункевич Д.В.}
    ]{Предметная область и онтология действий и методик проектирования ostis-систем}
\label{sd_actions_methodology_design}

\scsubsection[
    \protect\scneditors{Банцевич К.А.;Бутрин С.В.}
    \protect\scnmonographychapter{Глава 5.2. Методика и средства проектирования и анализа качества баз знаний интеллектуальных компьютерных систем нового поколения}
    ]{Предметная область и онтология действий и методик проектирования баз знаний ostis-систем}
\label{sd_actions_methodology_know_base_design}

\scsubsection[
    \protect\scneditor{Шункевич Д.В.}
    \protect\scnmonographychapter{Глава 5.3. Методика и средства компонентного проектирования решателей задач интеллектуальных компьютерных систем нового поколения}
    ]{Предметная область и онтология действий и методик проектирования решателей задач ostis-систем}
\label{sd_actions_methodology_problem_solver_design}

\scsubsection[
    \protect\scneditor{Садовский М.Е.}
    \protect\scnmonographychapter{Глава 5.4. Методика и средства компонентного проектирования интерфейсов интеллектуальных компьютерных систем нового поколения}
    ]{Предметная область и онтология действий и методик проектирования интерфейсов ostis-систем}
\label{sd_actions_methodology_interface_design}
\begin{SCn}
	\scnsectionheader{Предметная область и онтология действий и методик проектирования интерфейсов ostis-систем}
	\scntext{аннотация}{Проектирование \textit{интерфейса компьютерных систем} --- это один из наиболее важных этапов разработки любой системы. Пользователь при использовании \textit{интерфейса} должен представить себе, какая информация о выполняемой задаче у него существует, и в каком состоянии находятся средства, с помощью которых он будет решать данную задачу. Эффективность работы пользователя и его интерес обеспечивает правильно сформулированная \uline{методика разработки и проектирования пользовательского интерфейса}. В рамках предметной области и онтологии рассмотрены этапы проектирования \textit{пользовательских интерфейсов} и этапы проектирования \textit{адаптивных интеллектуальных мультимодальных пользовательских интерфейсов}.}
	\begin{scnrelfromvector}{введение}
		\scnfileitem{В настоящее время организация взаимодействия пользователя с компьютерной системой основана на парадигме \uline{грамотного пользователя}, который знает, как управлять системой и несет полную ответственность за качество взаимодействия с ней.}
		\scnfileitem{Многообразие форм и видов \textit{интерфейсов} приводит к необходимости пользователя адаптироваться к каждой конкретной системе, обучаться принципам взаимодействия с ней для решения необходимых ему задач.}
		\scnfileitem{Проектирование \textit{пользовательских интерфейсов} включает в себя ряд последовательных этапов.}
		\scnfileitem{Методика проектирования \textit{пользовательских интерфейсов} является важной частью \textit{Технологии OSTIS}, так как она описывает этапы проектирования \textit{пользовательских интерфейсов ostis-систем}, что позволяет ускорить процесс разработки, обеспечивает создание удобных \textit{пользовательских интерфейсов}, улучшает опыт использования \textit{интеллектуальной системы} и повышает эффективность работы пользователей, учитывая их потребности и предпочтения.}
	\end{scnrelfromvector}
	
	\begin{scnrelfromlist}{ключевое понятие}
		\scnitem{библиотека многократно используемых компонентов пользовательских интерфейсов ostis-систем}
		\scnitem{метод оценки пользовательских интерфейсов}
		\scnitem{качественный метод оценки пользовательских интерфейсов}
		\scnitem{количественный метод оценки пользовательских интерфейсов}
		\scnitem{тестирование удобства использования пользовательских интерфейсов}
		\scnitem{отслеживание движения глаз}
		\scnitem{экспертная оценка пользовательских интерфейсов}
		\scnitem{A/Б тестирование пользовательских интерфейсов}
		\scnitem{древовидное тестирование пользовательских интерфейсов}
	\end{scnrelfromlist}
	
	\begin{scnrelfromlist}{библиография}
		\scnitem{\scncite{Ehlert2003}}
		\scnitem{\scncite{Kong2011}}
		\scnitem{\scncite{Sadouski2022a}}
		\scnitem{\scncite{Ivory2001}}
		\scnitem{\scncite{Jeffries1991}}
		\scnitem{\scncite{ISO9241-161}}
		\scnitem{\scncite{Zeng2019}}
		\scnitem{\scncite{ISO16982}}
		\scnitem{\scncite{Koronchik2011}}
		\scnitem{\scncite{Koronchik2014}}
	\end{scnrelfromlist}
	
	\begin{scnsubstruct}
		\scnheader{Предметная область действий и методик проектирования интерфейсов ostis-систем}
		\scniselement{предметная область}
		\scnhaselementrole{максимальный класс объектов исследования}{проектирование адаптивных интеллектуальных мультимодальных пользовательских интерфейсов}
		\begin{scnhaselementrolelist}{класс объектов исследования}
			\scnitem{анализ методов оценки пользовательских интерфейсов}
		\end{scnhaselementrolelist}
		
		\scnheader{Существующие методики проектирования адаптивных интеллектуальных мультимодальных пользовательских интерфейсов}
		\scnhaselement{Проектирование адаптивных интеллектуальных мультимодальных пользовательских интерфейсов (Ehlert)}
		\scnhaselement{Проектирование адаптивных интеллектуальных мультимодальных пользовательских интерфейсов (Kong)}
		
		\scnheader{Проектирование адаптивных интеллектуальных мультимодальных пользовательских интерфейсов (Ehlert)}
		\begin{scnrelfromset}{этапы}
			\scnitem{анализ}
			\begin{scnindent}
				\scntext{примечание}{Этап анализа является, вероятно, самой важной фазой в любом процессе проектирования, в том числе в проектировании \textit{интерфейсов ostis-систем}. В процессе проектирования традиционного \textit{интерфейса} необходимо проанализировать, кто является обычным пользователем, какие задачи \textit{интерфейс} должен поддерживать.}
				\scnsuperset{функциональный анализ}
				\begin{scnindent}
					\scntext{примечание}{В рамках \textit{функционального анализа} необходимо дать ответ на вопрос: \scnqqi{каковы основные функции системы}.}
				\end{scnindent}
				\scnsuperset{анализ данных}
				\begin{scnindent}
					\scntext{примечание}{В рамках \textit{анализа данных} необходимо определить \uline{значение и структуру данных}, используемых в приложении.}
				\end{scnindent}
				\scnsuperset{анализ пользователей}
				\begin{scnindent}
					\scntext{примечание}{В рамках \textit{анализа пользователей} необходимо выделить \uline{типы пользователей и их возможности} в интеллектуальном и когнитивном плане.}
				\end{scnindent}
				\scnsuperset{анализ среды}
				\begin{scnindent}
					\scntext{примечание}{В рамках \textit{анализа среды} необходимо определить \uline{требования, предъявляемые к среде}, в которой будет работать система. Результатом данного этапа является \uline{cпецификация целей и информационных потребностей пользователя}, а также \uline{спецификация функций и информации}, которые требуются системе.}
				\end{scnindent}
			\end{scnindent}
			\scnitem{разработка \textit{интерфейса}}
			\begin{scnindent}
				\begin{scnrelfromset}{этапы}
					\scnfileitem{\textit{Создание модели интерфейса} в соответствии с этапом анализа.}
					\scnfileitem{Реализация модели \textit{интерфейса}.}
				\end{scnrelfromset}
				\scntext{результат}{Результатом данного этапа является \textit{пользовательский интерфейс}, который, по мнению разработчика, удовлетворяет требованиям пользователей и соответствует требованиям, сформулированным на этапе анализа.}
			\end{scnindent}
			\scnitem{оценка \textit{интерфейса}}
			\begin{scnindent}
				\begin{scnrelfromset}{предполагает}
					\scnfileitem{Требования, которые были сформулированы на этапе \textit{анализа}, должны быть удовлетворены.}
					\scnfileitem{Эффективность модели \textit{интерфейса} должна быть исследована.}
				\end{scnrelfromset}
				\scntext{примечание}{На этапе \textit{оценки интерфейса} необходимо вернуться к требованиям \textit{этапа анализа}. Требования, которые были сформулированы на \textit{этапе анализа}, должны быть выполнены, а также должна быть исследована эффективность модели \textit{интерфейса}. Чтобы определить эту эффективность, необходимо определить критерии эффективности.}
				\scntext{примечание}{Очень важным, но субъективным критерием является удовлетворенность пользователя. Поскольку пользователь должен работать с \textit{интерфейсом}, он имеет право голоса в вопросе о том, удобно ли работать с \textit{интерфейсом}, насколько привлекательным является интерфейс.}
				\begin{scnrelfromset}{критерии эффективности}
					\scnfileitem{Количество ошибок.}
					\scnfileitem{Время выполнения задачи.}
					\scnfileitem{Отношение пользователя к \textit{интерфейсу}.}
				\end{scnrelfromset}
			\end{scnindent}
			\scnitem{доработка и усовершенствование}
			\begin{scnindent}
				\scntext{примечание}{\textit{Доработка и усовершенствование} осуществляется на основе проблем, выявленных на этапе оценки. В рамках данного этапа вносится ряд улучшений в модель \textit{интерфейса}. Затем начинается новый цикл проектирования. Этот итеративный процесс будет продолжаться до тех пор, пока результат оценки не будет удовлетворять обозначенным требованиям.}
			\end{scnindent}
		\end{scnrelfromset}
		\begin{scnindent}
			\scnrelfrom{источник}{\scncite{Ehlert2003}}
		\end{scnindent}
		\scntext{примечание}{В \textit{пользовательском интерфейсе} часто нет среднего пользователя. В идеале, \textit{пользовательский интерфейс} должен быть способен адаптироваться к любому пользователю в любой среде. Поэтому используемая техника адаптации должна быть разработана таким образом, чтобы она могла поддерживать все типы пользователей.}
		
		\scnheader{Проектирование адаптивных интеллектуальных мультимодальных пользовательских интерфейсов (Kong)}
		\begin{scnrelfromset}{этапы}
			\scnfileitem{Моделирование \textit{пользовательского интерфейса} (описание абстрактного \textit{пользовательского интерфейса}).}
			\scnfileitem{Проектирование \textit{пользовательского интерфейса} по умолчанию (стандартная версия, конкретный \textit{пользовательский интерфейс}).}
			\scnfileitem{Разработка \textit{пользовательского интерфейса} (расширение или замена \textit{пользовательского интерфейса} по умолчанию) --- этот шаг опускается, когда система генерирует \textit{пользовательский интерфейс} по умолчанию автоматически.}
			\scnfileitem{Создание контекста использования (идентификация и создание контекста использования --- модели пользователя, модель устройства и модель среды/платформы).}
			\scnfileitem{Адаптация \textit{пользовательского интерфейса} --- автоматически (адаптация пользовательского интерфейса во время выполнения для соответствия конкретного контекста использования).}
			\scnfileitem{Кастомизация \textit{пользовательского интерфейса} --- настройка \textit{пользовательского интерфейса} самим пользователем (адаптируемость).}
		\end{scnrelfromset}
		\begin{scnindent}
			\scnrelfrom{источник}{\scncite{Kong2011}}
		\end{scnindent}
		
		\scnheader{Существующие методики проектирования адаптивных интеллектуальных мультимодальных пользовательских интерфейсов}
		\begin{scnrelfromset}{общие этапы}
			\scnitem{анализ контекста использования и задач пользователей}
			\scnitem{проектирование и разработка \textit{интерфейса}}
			\scnitem{оценка качества спроектированного \textit{интерфейса}}
		\end{scnrelfromset}
		\begin{scnrelfromset}{недостатки}
			\scnfileitem{Знания по каждому этапу проектирования находятся у разных специалистов в неформализорованном неунифицированном виде.}
			\scnfileitem{Отсутствие \textit{этапа формализованного документирования} этапов проектирования приводит в дальнейшем к необходимости создания отдельных help-систем для пользователей, разработчиков и так далее.}
		\end{scnrelfromset}
		
		\scnheader{Предлагаемая методика проектирования интерфейсов ostis-систем}
		\begin{scnrelfromvector}{этапы}
			\scnitem{анализ пользователя, его задач и целей, а также контекста использования}
			\begin{scnindent}
				\scntext{примечание}{Результаты первого этапа, такие как: модель конкретного пользователя, его потребности и контекст использования системы (устройство, окружающая среда) должны быть формализованы в рамках соответствующих онтологий \textit{базы знаний} \textit{интерфейса}. При этом в процессе формализации по необходимости должны быть переиспользованы \textit{компоненты базы знаний} из \textit{библиотеку многократно используемых компонентов ostis-систем}, а новые компоненты могут пополнить эту же библиотеку.}
			\end{scnindent}
			\scnitem{анализ требований к \textit{пользовательскому интерфейсу} и спецификация проектируемого \textit{пользовательского интерфейса}}
			\begin{scnindent}
				\scntext{примечание}{Результатом второго этапа являются конечные требования к \textit{интерфейсу}, которые должны быть сформулированы относительно модели пользователя и его цели, а также относительно контекста использования.}
				\scntext{примечание}{Спецификация включает в себя список задач решаемых интерфейсом, описание \textit{внешних языков представления знаний}.}
				\scntext{примечание}{Результаты должны быть формализованы, а в процессе выполнения могут быть использованы существующие компоненты из \textit{библиотеки многократно используемых компонентов интерфейсов ostis-систем}.}
			\end{scnindent}
			\scnitem{задачно-ориентированная декомпозиция \textit{пользовательского интерфейса}}
			\begin{scnindent}
				\scntext{примечание}{На этапе задачно-ориентированной декомпозиции \textit{пользовательского интерфейса} специфицированный интерфейс разбивается на интерфейсные подсистемы, которые могут разрабатываться параллельно. Это позволяет сократить сроки проектирования \textit{пользовательского интерфейса}. Целесообразно проводить разбиение таким образом, чтобы максимальное количество подсистем уже имелось в \textit{библиотеке многократно используемых компонентов пользовательских интерфейсов ostis-систем}.}
			\end{scnindent}
			\scnitem{проектирование \textit{пользовательского интерфейса} по умолчанию}
			\begin{scnindent}
				\scntext{примечание}{В соответствии с требованиями к \textit{пользовательскому интерфейсу}, строится модель \textit{адаптивного интеллектуального мультимодального пользовательского интерфейса}, которая является результатом третьего этапа.}
				\scntext{примечание}{Такая модель будет включать в себя формализованную модель \textit{базы знаний} и \textit{решателя задач}. При проектировании могут быть использованы компоненты \textit{интерфейса}, \textit{базы знаний} и \textit{решателя задач}. Компоненты будут записаны в унифицированном виде, что позволит обеспечить их автоматическую совместимость.}
			\end{scnindent}
			\scnitem{разработка \textit{пользовательского интерфейса}}
			\begin{scnindent}
				\scntext{примечание}{Результатом четвертого этапа является реализация спроектированного \textit{пользовательского интерфейса}.}
				\scntext{примечание}{После разработки \textit{пользовательского интерфейса} выделяются типовые фрагменты интерфейса. Специфицируя фрагменты интерфейса необходимым образом следует включать их в \textit{библиотеку многократно используемых компонентов пользовательских интерфейсов ostis-систем}.}
				\scntext{примечание}{При разработке пользовательского интерфейса также можно использовать готовые компоненты интерфейса из \textit{библиотеки многократно используемых компонентов пользовательских интерфейсов ostis-систем}.}
			\end{scnindent}
			\scnitem{анализ \textit{пользовательского интерфейса} и его адаптация}
			\begin{scnindent}
				\scntext{примечание}{На данном этапе используются готовые компоненты \textit{решателя задач}: \textit{sc-агенты анализа пользовательского интерфейса} и \textit{sc-агенты изменения модели пользовательского интерфейса на основе логических правил адаптации}.}
				\scntext{примечание}{Таким образом будет сформирована \textit{база знаний} проектируемого \textit{интерфейса}, которая автоматически может быть использована в качестве help-системы для пользователей, разработчиков и так далее.}
			\end{scnindent}
		\end{scnrelfromvector}
		\begin{scnindent}
			\scnrelfrom{источник}{\scncite{Sadouski2022a}}
			\scntext{примечание}{Поскольку знания о конкретном этапе обычно находятся у разных экспертов, особенностью предлагаемого подхода является обязательное формализованное документирование знаний в унифицированном виде и применение на каждом из этапов компонентного подхода.}
			\scntext{примечание}{Для применения компонентного подхода предлагается использовать \textit{библиотеку многократно используемых компонентов ostis-систем}.}
		\end{scnindent}
		
		\scnheader{анализ методов оценки пользовательских интерфейсов}
		\scntext{примечание}{Оценка \textit{пользовательского интерфейса} необходима для улучшения коммуникации между \textit{интеллектуальными системами} и их пользователями. Существует множество методов для оценки \textit{пользовательских интерфейсов}, направленных, в основном, на выявление проблем с использованием системы и минимизацию риска ошибок. Однако до сих пор отсутствует комплексный подход к оценке \textit{пользовательских интерфейсов интеллектуальных систем}.}
		\begin{scnindent}
			\begin{scnrelfromlist}{смотрите}
				\scnitem{\scncite{Ivory2001}}
				\scnitem{\scncite{Jeffries1991}}
			\end{scnrelfromlist}
		\end{scnindent}
		\scntext{примечание}{При оценке \textit{пользовательских интерфейсов} большую роль играет человеческий фактор, а основными участниками являются пользователи и эксперты. Очень сложно оценить правильность решения по адаптации \textit{пользовательского интерфейса интеллектуальной системы} и оценить его без участия человека.}
		\scntext{примечание}{Оценка \textit{пользовательских интерфейсов интеллектуальных систем} имеет свои особенности и требует специальных методов и инструментов.}
		
		\scnheader{пользовательский интерфейс}
		\begin{scnrelfromset}{правила построения}
			\scnfileitem{\textit{интерфейс} должен быть интуитивно понятен для конечного пользователя.}
			\scnfileitem{\textit{интерфейс} должен быть доступным для пользователей с ограниченными возможностями и пользователей, впервые сталкивающимися с информационными технологиями.}
			\scnfileitem{Достижение цели пользователем должно осуществляться наименее возможным количеством шагов.}
			\begin{scnindent}
				\scnrelfrom{источник}{\scncite{ISO9241-161}}
			\end{scnindent}
		\end{scnrelfromset}
		
		\scnheader{методы оценки пользовательских интерфейсов}
		\scnsuperset{оценка точности и полноты}
		\begin{scnindent}
			\scntext{примечание}{Это метод, при котором оценивается точность и полнота ответов системы на запросы пользователей.}
			\scntext{примечание}{Точность означает, насколько хорошо \textit{пользовательский интерфейс} отражает действительный опыт пользователя и предлагает ему наиболее подходящий контент в соответствии с его потребностями и предпочтениями. Оценить точность возможно путем анализа, насколько легко использовать интерфейс, насколько он отражает действительные возможности интеллектуальной системы и насколько он помогает пользователям достигать своих целей более эффективно.}
			\scntext{примечание}{Полнота, с другой стороны, означает, насколько хорошо \textit{пользовательский интерфейс} предоставляет всю необходимую информацию и функциональность, которые пользователь может потребовать в процессе взаимодействия с системой. Оценить полноту возможно путем анализа того, насколько полное и четкое описание функций, возможностей и ограничений системы предоставляется через \textit{пользовательский интерфейс}, в каком объеме пользователю доступна необходимая информация для принятия решений и насколько система может эффективно реагировать на запросы и потребности пользователя.}
		\end{scnindent}
		\scnsuperset{оценка персонализации}
		\begin{scnindent}
			\scntext{примечание}{Это метод, при котором оценивается способность системы адаптироваться к потребностям и предпочтениям каждого пользователя. Оценка персонализации выполняется с помощью анализа пользовательских данных, а также проведения опросов среди пользователей, чтобы определить, насколько хорошо система учитывает их потребности.}
			\begin{scnrelfromset}{параметры оценки}
				\scnfileitem{Работоспособность --- действительно ли \textit{пользовательский интерфейс} адаптивный и соответствует ли он потребностям пользователя?}
				\scnfileitem{Уникальность --- насколько уникальным и индивидуальным является персонализированный \textit{пользовательский интерфейс}?}
				\scnfileitem{Понятность --- насколько просто и легко пользователь может настроить и использовать персонализацию?}
				\scnfileitem{Эффективность --- насколько хорошо персонализированный \textit{пользовательский интерфейс} помогает пользователю в выполнении задач?}
				\scnfileitem{Удовлетворенность --- насколько положительным и удовлетворительным является \textit{пользовательский интерфейс}?}
			\end{scnrelfromset}
			\scntext{примечание}{Оценка персонализации \textit{адаптивных пользовательских интерфейсов} может быть выполнена различными методами, включая тестирование с использованием фокус-групп, опросы пользователей, анализ данных и другие. Важно отметить, что оценка персонализации должна проходить на всех этапах разработки для повышения пользовательского опыта и улучшения качества \textit{пользовательских интерфейсов интеллектуальных систем}.}
		\end{scnindent}
		\begin{scnrelfromset}{декомпозиция}
			\scnitem{качественный метод оценки пользовательских интерфейсов}
			\begin{scnindent}
				\scntext{задача}{Задача \textit{качественных методов оценки пользовательских интерфейсов} --- помочь понять мотивы поведения, потребности и логику пользователей.}
				\scntext{примечание}{\textit{качественные методы} нацелены на сбор данных, описывающих предмет изучения. Они позволяют углубиться в предметную область для получения представления о мотивации, мышлении и взглядах пользователей.}
				\scnsuperset{тестирование удобства использования пользовательских интерфейсов}
				\begin{scnindent}
					\scntext{примечание}{\textbf{\textit{тестирование удобства использования пользовательских интерфейсов}} является важной частью процесса проектирования и разработки. \textit{тестирование удобства использования} гарантирует, что эти интерфейсы удобны и полезны для потенциальных пользователей.}
					\begin{scnrelfromset}{этапы}
						\scnfileitem{Определение целевых пользователей. Это может потребовать проведения исследований пользователей для понимания потребностей и предпочтений потенциальных пользователей.}
						\scnfileitem{Разработка тестовых сценариев, которые отражают типичные случаи использования \textit{интерфейса}. Эти сценарии должны быть разработаны для проверки адаптивных возможностей \textit{интерфейса}, а также его удобства использования.}
						\scnfileitem{Проведение тестирования пользователей. Тестирование пользователей включает в себя наблюдение за пользователями во время взаимодействия с \textit{интерфейсом интеллектуальной системы}. Пользователи выполняют задачи, связанные с тестовыми сценариями. Тестирование пользователей может проводиться в контролируемой лабораторной среде или удаленно с использованием технологии совместного просмотра экрана.}
						\scnfileitem{Анализ результатов тестирования. Результаты тестирования пользователей анализируются для выявления проблем с удобством использования и областей для улучшения.}
						\scnfileitem{Итерация дизайна. На основе результатов тестирования пользователей и анализа результатов тестирования, \textit{интерфейс} может быть изменен для устранения проблем с удобством использования и улучшения общего пользовательского опыта.}
					\end{scnrelfromset}
				\end{scnindent}
				\scnsuperset{отслеживание движения глаз}
				\begin{scnindent}
					\scntext{примечание}{\textbf{\textit{отслеживание движения глаз}} --- это метод, который используется для анализа того, как пользователи взаимодействуют с \textit{пользовательским интерфейсом}. Отслеживание движения глаз определяет точки фиксации взгляда пользователя при взаимодействии с системой, а также переходы между ними.}
					\scntext{примечание}{При использовании метода определяется \textit{компонент пользовательского интерфейса}, на который пользователь смотрел дольше всего, сколько времени он уделяет каждому компоненту и как легко ему удается найти нужную информацию.}
					\scntext{примечание}{Метод выявляет \textit{компоненты интерфейса}, которым уделяется больше внимания, позволяет обнаружить области, вызывающие у пользователей затруднения.}
					\scntext{примечание}{Метод позволяет получить реалистичный образ отношения пользователя и \textit{интерфейса}, поскольку он фиксирует естественное движение глаз человека. Кроме того, \textit{отслеживание движения глаз} позволяет быстро найти проблемные места в \textit{пользовательском интерфейсе} и предложить улучшения, которые могут увеличить удобство использования \textit{интерфейса}.}
				\end{scnindent}
				\scnsuperset{экспертная оценка пользовательских интерфейсов}
				\begin{scnindent}
					\scntext{примечание}{Метод \textit{экспертной оценки пользовательских интерфейсов} заключается в исследовании \textit{пользовательского интерфейса} на соответствие заранее определенным правилам.}
					\begin{scnindent}
						\scnrelfrom{смотрите}{\scncite{ISO16982}}
					\end{scnindent}
					\scntext{примечание}{Достаточно часто метод \textit{экспертной оценки пользовательских интерфейсов} используется в тандеме с \textit{тестированием удобства использования}: \textit{экспертная оценка} используется для формирования гипотез о проблемах, а \textit{тестирование удобства использования} --- для их проверки.}
					\begin{scnrelfromvector}{этапы}
						\scnfileitem{Выявление экспертов.}
						\begin{scnindent}
							\scntext{примечание}{Первый этап --- поиск экспертов в предметной области, имеющих опыт работы как с \textit{интеллектуальными системами}, так и с моделью \textit{пользовательского интерфейса}. В число этих экспертов могут входить специалисты по взаимодействию человека с компьютером, специалисты по машинному обучению или эксперты в области, для которой разрабатывается интерфейс.}
							\scntext{примечание}{Экспертам предоставляется доступ к \textit{пользовательскому интерфейсу} и предлагается взаимодействовать с ним различными способами. Им может быть предложено выполнить задачи или сценарии, которые отражают типичные варианты использования интерфейса.}
						\end{scnindent}
						\scnfileitem{Проведение оценки.}
						\begin{scnindent}
							\scntext{примечание}{Эксперты оценивают \textit{пользовательский интерфейс} на основе набора установленных принципов удобства использования. Также может быть произведена оценка адаптивности \textit{интерфейса}.}
						\end{scnindent}
						\scnfileitem{Анализ результатов.}
						\begin{scnindent}
							\scntext{примечание}{Результаты оценки анализируются для выявления проблем с удобством использования и областей для улучшения. Эксперты могут давать рекомендации по улучшению интерфейса на основе своей оценки.}
						\end{scnindent}
					\end{scnrelfromvector}
				\end{scnindent}
			\end{scnindent}
			\scnitem{количественный метод оценки пользовательских интерфейсов}
			\begin{scnindent}
				\scntext{примечание}{\textit{количественные методы оценки пользовательских интерфейсов} позволяют выявить сложности или возможности, с которыми сталкиваются пользователи, и отделить реальные проблемы от предполагаемых. Такие методы измеряют числовые показатели.}
				\scntext{примечание}{\textit{количественными методы} формируют представление о том, чем занимаются пользователи.}
				\scnsuperset{A/Б тестирование пользовательских интерфейсов}
				\begin{scnindent}
					\scntext{примечание}{\textit{A/Б тестирование пользовательских интерфейсов} --- метод сравнения двух версий \textit{пользовательского интерфейса}. Результатом проведения метода является выявление версии \textit{интерфейса} наиболее подходящей для выполнения конкретной задачи.}
					\scntext{примечание}{Пользователи случайным образом разбиваются на два сегмента, каждый из которых видит только одну версию интерфейса.}
					\begin{scnrelfromvector}{этапы}
						\scnfileitem{Определение гипотезы.}
						\begin{scnindent}
							\scntext{примечание}{В первую очередь необходимо определить цель для проверки. В данном случае это могут быть как отдельные \textit{компоненты пользовательского интерфейса}, так и \textit{пользовательский интерфейс} в целом.}
							\scntext{примечание}{На основе анализа данных или личных предположений формулируется гипотеза, например, \scnqqi{если мы изменяем цвет и размер кнопки, пользователи будут чаще нажимать на нее} или \scnqqi{если мы добавим кнопку, которая будет видна на каждой странице, пользователи лучше будут понимать, как оставить отзыв}.}
						\end{scnindent}
						\scnfileitem{Определение размеров выборок.}
						\begin{scnindent}
							\scntext{примечание}{Чтобы получить репрезентативные результаты, нужно определить размер контрольной и экспериментальной групп. Например, 50\% пользователей попадут в контрольную группу, которая будет видеть старый интерфейс, а другие 50\% в экспериментальную, которая будет видеть измененный интерфейс.}
						\end{scnindent}
						\scnfileitem{Итерация интерфейса.}
						\begin{scnindent}
							\scntext{примечание}{На этом этапе вносятся изменения в интерфейс, которые соответствуют определенной гипотезе.}
						\end{scnindent}
						\scnfileitem{Наблюдение и сбор данных.}
						\begin{scnindent}
							\scntext{примечание}{Во время тестирования фиксируются действия пользователей, например, клики и время, проведенное на странице. Эти данные необходимы для определения изменений интерфейса, наиболее повлиявших на поведение пользователей.}
						\end{scnindent}
						\scnfileitem{Анализ результатов.}
						\begin{scnindent}
							\scntext{примечание}{После того, как тестирование завершено, проводится анализ данных для понимания, насколько значимы различия между контрольной и экспериментальной группами. Например, если пользователи, которые видели измененный интерфейс, кликали на кнопку в 2 раза чаще, чем пользователи, которые видели старый интерфейс, это говорит о том, что изменения были успешными и их необходимо внедрять в конечный \textit{пользовательский интерфейс}.}
						\end{scnindent}
					\end{scnrelfromvector}
				\end{scnindent}
				\scnsuperset{древовидное тестирование}
				\begin{scnindent}
					\scntext{примечание}{\textit{древовидное тестирование пользовательских интерфейсов} --- это метод оценки качества \textit{пользовательских интерфейсов}, который заключается в тестировании древовидной структуры навигации по системе.}
					\scntext{примечание}{Метод помогает определить, насколько эффективно пользователи могут находить нужную информацию и выполнять различные задачи.}
					\scntext{примечание}{В \textit{древовидном тестировании} объектом оценки является древовидная структура навигации по системе. Эта структура представляет собой дерево, в котором корневой элемент является главной страницей, а дочерние элементы --- подстраницы или разделы.}
					\scntext{цель}{Проверить, насколько легко пользователи могут навигировать по этой структуре и находить нужную информацию.}
					\begin{scnrelfromvector}{этапы}
						\scnfileitem{На основе структуры навигации создается тестовая среда, которая представляет собой виртуальный \textit{интерфейс}.}
						\scnfileitem{Респондентам предлагается выполнить задание, связанное с поиском нужной информации. Это может быть, например, поиск конкретной страницы или поиск информации по определенному тематическому разделу.}
						\scnfileitem{Респондентам дается доступ к тестовой среде, и они проходят по древовидной структуре навигации, пытаясь найти нужную им информацию.}
						\scnfileitem{В процессе прохождения теста респонденты записывают свои действия и комментарии о том, как они ориентируются в системе, какой путь выбирают для поиска нужной информации, какие ошибки и препятствия им приходится преодолевать и так далее.}
						\scnfileitem{По результатам тестирования делается анализ путей, выбранных респондентами, выявляются наиболее эффективные и неэффективные способы навигации, а также выделяются проблемные зоны \textit{интерфейса} системы.}
						\scnfileitem{На основе анализа результатов тестирования и выявленных проблем делается рекомендация по оптимизации древовидной структуры навигации или изменению \textit{пользовательского интерфейса} для улучшения пользовательского опыта.}
					\end{scnrelfromvector}
				\end{scnindent}
			\end{scnindent}
		\end{scnrelfromset}
		\scntext{примечание}{Представленные \textit{методы оценки пользовательских интерфейсов} предлагается применять для оценки \textit{пользовательских интерфейсов ostis-систем} в рамках рассмотренного ранее этапа \scnqqi{Анализ пользовательского интерфейса и его адаптация}. При этом особенностью проектирования \textit{пользовательских интерфейсов ostis-систем} является постоянная информационная поддержка пользователя на всех этапах проектирования \textit{интерфейса} за счет наличия в \textit{базе знаний} каждой \textit{ostis-системы} \textit{Предметной области методик проектирования пользовательских интерфейсов ostis-систем}, содержащей рассмотренные методики}
		
		\bigskip
	\end{scnsubstruct}
\end{SCn}


\scsection{Предметная область и онтология действий и методик \uline{обучения} проектированию ostis-систем}
\label{sd_actions_methodology_learning_design}

\scsection[
    \protect\scneditor{Шункевич Д.В.}
    ]{Предметная область и онтология средств проектирования ostis-систем}
\label{sd_fund_design}

% ---------------------
% Что за логико-семантические модели? Замена на предметные области

% Про что это?
\scsubsection[
    \protect\scneditor{Бутрин С.В.}
    \protect\scnmonographychapter{Глава 5.2. Методика и средства проектирования и анализа качества баз знаний интеллектуальных компьютерных систем нового поколения}
    ]{Логико-семантическая модель комплекса встраиваемых ostis-систем автоматизации проектирования баз знаний ostis-систем}
\label{logical_model_embed_automation_design}

% KBE, SC-builder, это вообще надо?
\scsubsubsection[
    \protect\scneditors{Бутрин С.В.}
    \protect\scnmonographychapter{Глава 5.2. Методика и средства проектирования и анализа качества баз знаний интеллектуальных компьютерных систем нового поколения}
    ]{Предметная область и онтология ostis-системы редактирования, сборки и ввода исходных текстов различных компонентов проектируемой базы знаний в память ostis-системы}
\label{edit_assem_logical_model}
\begin{SCn}
\scnsectionheader{Предметная область и онтология ostis-системы редактирования, сборки и ввода исходных текстов различных компонентов проектируемой базы знаний в память ostis-системы}
\scntext{аннотация}{Для решения задачи индивудуального наполения базы знаний предлагается использовать специализированный инструментарий, который включает в себя различного рода редакторы и трансляторы.

Текущая реализация \textit{ostis-платформы} и решателя задач поддерживает работу с файлами исходных текстов базы знаний. Для создания таких файлов исходных текстов на \textit{SCs-коде} можно воспользоваться любым текстовым редактором.

Для создания файлов исходных текстов в \textit{SCg-коде} может быть использован редактор \textbf{\textit{KBE}} (Knowledge Base source Editor, см. \scncite{Knowledge-base-editor2022}).}

\begin{scnsubstruct}

\scnheader{Knowledge Base source Editor}
\scntext{пояснение}{\textit{KBE} является приложением, которое направлено на помощь в создании и редактировании фрагментов баз знаний интеллектуальных систем, проектирование которых основано на \textit{Технологии OSTIS}. В основу данного редактора положен принцип визуализации данных, хранящихся в базе знаний, что намного упрощает процесс их редактирования и ускоряет процесс проектирования баз знаний.}

\scnnote{\textit{пользовательский интерфейс} инструмента представляет собой главное окно, в котором пользователь может создавать вкладки. В каждой вкладке может происходить редактирование различных файлов исходных текстов баз знаний, представленных с помощью \textit{SCg-кода}. В рамках главного окна имеется панель инструментов и меню приложения. На панель инструментов, как и в пользовательских интерфейсах большинства приложений, вынесены наиболее часто используемые команды.}

\scnheader{команды меню KBE}
\begin{scnrelfromset}{разбиение}
	\scnitem{команды, которые являются общими для всех вкладок}
	\begin{scnindent}
		\begin{scneqtoset}
		 	\scnitem{команда сохранения}
		 	\scnitem{команда загрузки}
		 	\scnitem{команда помощи}
		 \end{scneqtoset}
	\end{scnindent}
	\scnitem{команды, которые специфичны для активной вкладки}
	\begin{scnindent}
		\scntext{пояснение}{Такие команды зависят от типа активной вкладки}
	\end{scnindent}
\end{scnrelfromset}

\scnheader{Knowledge Base source Editor}
\scnnote{Основная идея, которая преследуется в данном редакторе SCg-кода --- это упрощение и ускорение процесса редактирования sc.g-текстов. В процессе редактирования пользователю доступны различные режимы редактирования.}

\scnheader{режимы редактирования KBE}
\begin{scnrelfromset}{разбиение}
	\scnitem{Режим выделения и создания узлов}.
	\begin{scnindent}
		\scnnote{В данном режиме пользователь может работать со всеми объектами выделяя и перемещая их, вызывая контекстное меню с командами.Отличительной особенностью данного режима является то, что в нем можно создавать sc.g-узлы}
	\end{scnindent}
	
	\scnitem{Режим создания sc.g-дуг}.
	\begin{scnindent}
		\scnnote{Создание sc.g-дуги начинается с того, что пользователь указывает объект из которого она будет выходить, далее он может указать точки излома дуги, завершается создание указанием конечного объекта.В процессе создания пользователь может отменять последнее действие (указание начального объекта, точки излома)}
	\end{scnindent}
	
	\scnitem{Режим создания sc.g-шин}.
	\begin{scnindent}
		\scnnote{sc.g-шины используются для увеличения контактной площади узла, поэтому они могут создаваться лишь для sc.g-узлов. Создание шины начинается с указания sc.g-узла, далее как и при создании sc.g-дуг указываются точки излома. Как и при создании дуг пользователь может отменять последнее действие нажатием правой клавиши мыши}
	\end{scnindent}
	
	\scnitem{Режим создания sc.g-контуров}
	\begin{scnindent}
		\scnnote{Создание sc.g-контура начинается с указаний первой его точки. Далее, как и в случае с sc.g-дугами и sc.g-шинами, указываются точки. Стоит отметить, что все объекты, которые попадут внутрь созданного контура, будут добавлены в него автоматически. Как и при создании дуг и шин пользователь может отменять последнее действие.}
	\end{scnindent}
\end{scnrelfromset}

\scnheader{команды редактирования KBE}
\begin{scnrelfromset}{разбиение}
		\scnitem{команда изменения основного текстового идентификатора элемента}
		\scnitem{команда изменения типа элемента}
		\scnitem{команда установки содержимого}
\end{scnrelfromset}

\scnnote{Полученные файлы исходных текстов в дальнейшем могут быть погружены в \textit{базу знаний }ostis-системы с помощью \textit{Реализации транслятора файлов исходных текстов \textit{базы знаний} в sc-память ostis-платформы}}

\scnheader{Реализация транслятора файлов исходных текстов базы знаний в sc-память ostis-платформы}
\scnidtf{sc-builder}
\scniselement{многократно используемый компонент ostis-систем, хранящийся в виде файлов исходных текстов}
\scnrelfrom{используемый язык}{SCs-код}
\begin{scnrelfromset}{зависимости компонента}
	\scnitem{Библиотека методов и структура данных C++ Standard Library}
\end{scnrelfromset}
\scnrelto{программный компонент}{Программный вариант реализации ostis-платформы}

\scnnote{\textit{Реализация транслятора файлов исходных текстов базы знаний в sc-память ostis-платформы} позволяет осуществить сборку \textit{базы знаний} из набора файлов исходных текстов, записанных в SCs-коде с ограничениями в бинарный формат, воспринимаемый \textit{Программной моделью sc-памяти} (см. \textit{\ref{sec_soft_platform_scin_code_example}~\nameref{sec_soft_platform_scin_code_example}}).
При этом возможна как сборка \scnqq{с нуля} (с уничтожением ранее созданного слепка памяти), так и аддитивная сборка, когда информация, содержащаяся в заданном множестве файлов, добавляется к уже имеющемуся слепку состояния памяти.
В текущей реализации сборщик осуществляет \scnqq{склеивание} (\scnqq{слияние}) sc-элементов, имеющих на уровне файлов исходных текстов одинаковые \textit{системные sc-идентификаторы}.}

\scnnote{Кроме \textit{KBE} существует редактор текстов базы знаний, являющийся частью \textit{Реализации интерпретатора sc-моделей пользовательских интерфейсов}, обладающий схожим с \textit{KBE} функционалом, но при этом позволяющий редактировать базу знаний в режиме реального времени и без создания файлов исходных текстов базы знаний, именно им рекомендуется пользоваться для редактирования базы знаний.
}
    \bigskip
\end{scnsubstruct}
\end{SCn}


% Что это?
\scsubsubsection[
    \protect\scneditor{Бутрин С.В.}
    \protect\scnmonographychapter{Глава 5.2. Методика и средства проектирования и анализа качества баз знаний интеллектуальных компьютерных систем нового поколения}
    ]{Логико-семантическая модель ostis-системы редактирования проектируемой базы знаний ostis-системы на уровне её внутреннего представления}
\label{edit_tools_proj_logical_model}

% Перенос из монографии
\scsubsubsection[
    \protect\scneditor{Бутрин С.В.}
    \protect\scnmonographychapter{Глава 5.2. Методика и средства проектирования и анализа качества баз знаний интеллектуальных компьютерных систем нового поколения}
    ]{Предметная область и онтология обнаружения и анализа ошибок и противоречий в базе знаний ostis-системы}
\label{sd_detec_error}
\begin{SCn}
\scnsectionheader{Предметная область и онтология обнаружения и анализа ошибок и противоречий в базе знаний ostis-системы}
\begin{scnsubstruct}

\scntext{анотация}{Важным этапом в разработке любой системы является контроль ее качества, так как именно на этом этапе определяется степень жизнеспособности и эффективности системы.}

\scnnote{Верификация является видом анализа качества и частью процесса разработки системы. Она заключается в проверке информации на правильность и полноту.
Ее целью является выявление ошибок, различных дефектов и недоработок для своевременного их устранения.

Существующие на данный момент методы верификации хорошо развиты и разработано большое количество различных моделей верификации, использующих расширенные таблицы решения, сети Петри, различные логики, например, логики с векторной семантикой и другие модели. Более того формируются специализированные онтологии для описания многообразия средств и моделей верификации баз знаний. Однако механизма взаимодействия средств, использующих данные методы, нет.}
\begin{scnindent}
	\begin{scnrelfromset}{источник}
		\scnitem{Arshinskiy2020}
		\scnitem{Rybina2007}
	\end{scnrelfromset}
\end{scnindent}


\scnheader{средства верификации баз знаний}
\begin{scnrelfromset}{проблемы}
	\scnfileitem{зависимость от формата представления информации, из-за чего приходится тратить дополнительно время на конвертирование информации}
	\scnfileitem{проблема невозможности быть переиспользованными, так как средства обычно делаются с учетом особенностей конкретной системы}
	\scnfileitem{проблема отсутствия механизма взаимодействия средств верификации и анализа знаний}
	\scnfileitem{высокая роль человека в процессе верификации, так как самым распространенным методом верификации баз знаний является ручная проверка базы экспертом, человек выступает как администратор, принимающий единогласное решение, навязывая свое мнение системе}
	\scnfileitem{современные средства не учитывают и не рассматривают процесс верификации в рамках взаимодействия систем друг с другом}
\end{scnrelfromset}
\begin{scnindent}
	\begin{scnrelfromset}{источник}
		\scnitem{Zhang2023}
	\end{scnrelfromset}
	\begin{scnrelfromset}{предлагаемое решение}
		\scnfileitem{использовать унифицированный и удобный формат представления знаний}
		\scnfileitem{системы создавались бы по общей методологии и были бы совместимы друг с другом}
		\scnfileitem{продумать и реализовать механизм позволяющий системе стремиться самой принимать решение относительно своего состояния и наличия в нем проблемных моментов и ошибок, система может допускать ошибки и не всегда принимать верные решения, но это должны быть ее ошибки, а не экспертов и разработчиков}
	\end{scnrelfromset}
\end{scnindent}

\scnheader{Преимущества \textit{Технологии OSTIS} в рамках задачи верификации}
\begin{scneqtoset}
	\scnfileitem{наличие общей методологии проектирования интеллектуальных систем, позволяющей решить проблему совместимости систем при их коллективном взаимодействии}
	\scnfileitem{все знания представлены в унифицированном виде, что позволяет эффективно их обрабатывать, сводя затраты на конвертирование к минимуму}
	\scnfileitem{средства, с помощью которых производится выявление, анализ и устранение противоречий, описаны в самой базе знаний, а также их спецификация представлены в самой базе знаний системы, тем самым обеспечивая легкость их расширения и позволяя системе знать, каким инструментарием она обладает}
	\scnfileitem{отсутствие семантических эквивалентных фрагментов, что обеспечивает локальность вносимых исправлений и исключает необходимость вносить исправления многократно в разных местах}
	\scnfileitem{многоагентный подход, который позволяет рассматривать средства анализа и верификации баз знаний как коллектив агентов, способных взаимодействовать друг с другом и в дальнейшем принимать общее решение касательно состояния базы знаний} 
\end{scneqtoset}

\scnheader{ostis-система верификации}
\begin{scnreltoset}{включение}
	\scnitem{Предметная область и онтология проблемных фрагментов базы знаний}
	\scnitem{Алгоритм выявления и устранения проблемных фрагментов базы знаний}
	\scnitem{Решатель задач выявления и устранения проблемных фрагментов}
\end{scnreltoset}

\scnnote{Качество базы знаний во многом определяется уровнем наличия/отсутствия \textit{не-факторов} в \textit{базе знаний}.}
\begin{scnindent}
	\begin{scnrelfromset}{источник}
		\scnitem{Narinjani2004}
	\end{scnrelfromset}
\end{scnindent}


\scnheader{не-фактор}
\scnidtf{группа семантических свойств, определяющих качество информации, хранимой в памяти кибернетической системы}
\begin{scneqtoset}
	\scnitem{корректность/некорректность информации, хранимой в памяти кибернетической системы}
	\scnitem{однозначность/неоднозначность информации, хранимой в памяти кибернетической системы}
	\scnitem{целостность/нецелостность информации, хранимой в памяти кибернетической системы}
	\scnitem{чистота/загрязненность информации, хранимой в памяти кибернетической системы}
	\scnitem{достоверность/недостоверность информации, хранимой в памяти кибернетической системы}
	\scnitem{точность/неточность информации, хранимой в памяти кибернетической системы}
	\scnitem{четкость/нечеткость информации, хранимой в памяти кибернетической системы}
	\scnitem{определенность/недоопределенность информации, хранимой в памяти кибернетической системы}
\end{scneqtoset}



\scnheader{проблемная структура}
\scnidtf{структура, описывающая проблемный фрагмент базы знаний}
\scnidtf{структура, описывающая некачественный фрагмент базы знаний}
\begin{scnreltoset}{объединение}
	\scnitem{некорректная структура}
	\begin{scnindent}
		\scnidtf{структура, содержащая фрагменты, противоречащие каким-либо правилам или закономерностям описанным в базе знаний}
	\end{scnindent}
	\scnitem{структура, описывающая неполноту в базе знаний}
	\begin{scnindent}
		\scnidtf{структура, в которой имеется неполнота (то есть имеется некоторое количество информационных дыр)}
		\scntext{примечание}{Под структурой, описывающей неполноту в базе знаний, понимается структура, содержащая фрагмент базы знаний, в котором отсутствует какая-либо информация, которая необходима (или, по крайней мере, желательна) для однозначного и полного понимания смысла данного фрагмента.}
		%% Переформулировать
	\end{scnindent}
	\scnitem{информационный мусор}
	\begin{scnindent}
		\scnidtf{структура, удаление которой существенно не усложнит деятельность системы}
		\scnidtf{структура, содержащая фрагмент базы знаний, который по каким-либо причинам стал ненужным и требует удаления}
	\end{scnindent}
\end{scnreltoset}

\scnheader{противоречие*}
\scnidtf{пара противоречащих друг другу фрагментов информации, хранимой в памяти кибернетической системы*}
\scntext{примечание}{Чаще всего противоречащими друг другу информационными фрагментами являются:
	\begin{scnitemize}
		\item явно представленная в памяти некоторая закономерность (некоторое правило);
		\item информационный фрагмент, не соответствующий (противоречащий) указанной закономерности.
	\end{scnitemize}
	В этом случае некорректность может присутствовать:
	\begin{scnitemize}
		\item либо в информационном фрагменте, который противоречит указанной закономерности;
		\item либо в самой этой закономерности;
		\item либо и там и там.
	\end{scnitemize}}


\scnheader{некорректная структура}
\begin{scnreltoset}{включение}
	\scnitem{дублирование системных идентификаторов}
	\scnitem{несоответствие элементов связки доменам отношения}
	\scnitem{цикл по отношению порядка}
	\scnitem{структура, противоречащая свойству единственности}
\end{scnreltoset}

\scnheader{структура, описывающая неполноту в базе знаний}
\begin{scnreltoset}{включение}
	\scnitem{не указан максимальный класс объектов исследования предметной области}
	\scnitem{для сущности указан системный, но не указаны основные идентификаторы для всех внешних языков}
	\scnitem{не указаны домены отношения}
	\scnitem{понятие не соотнесено ни с одной предметной областью}
\end{scnreltoset}

\scnheader{структура, требующее внимание разработчика}
\scnidtf{проблемная структура, для исправления которой требуется участие разработчика}
\scnheader{множество элементов, которые должны быть удалены для исправления структуры*}
\scnidtf{множество элементов, удаление которых из структуры позволяет устранить в ней противоречия}
\scnheader{множество элементов, которые должны быть добавлены для исправления структуры*}
\scnidtf{множество элементов, добавление которых в структуру позволяет устранить в ней противоречия}
\scnheader{структура, которую система не способна исправить сама}
\scnidtf{структура, в которой система не способна автоматически устранить противоречия}
	
	
\scnheader{следует отличать*}
\begin{scnhaselementset}
	\scnitem{структура, которую система не может решить сама}
	\begin{scnindent}
		\scntext{примечание}{Здесь структура, которую система не может решить сама, не может быть исправлена при взаимодействии с разработчиком и требует полного исправления от самого разработчика}
	\end{scnindent}
	\scnitem{требующее внимание разработчика}
	\begin{scnindent}
		\scntext{примечание}{Здесь структура, требующая внимания разработчика, может быть решена в процессе верификации, но потребуется участие разработчика}
	\end{scnindent}
\end{scnhaselementset}

	
\scnheader{Решатель задач средств выявления и устранения противоречий}
\begin{scnrelfromset}{декомпозиция абстрактного sc-агента}
	\scnitem{Неатомарный абстрактный sc-агент выявления противоречий}
	\begin{scnindent}
		\scnidtf{Множество агентов, обеспечивающих поиск и фиксирование противоречий в структуре}
	\end{scnindent}
	\scnitem{Неатомарный абстрактный sc-агент устранения противоречий}
	\begin{scnindent}
		\scnidtf{Множество агентов, создающих предложения по исправлению противоречий}
		\scntext{примечание}{Результатом работы таких агентов будут множества предлагаемых к удалению из структуры или добавлению в структуру элементов}
	\end{scnindent}
	\scnitem{Абстрактный sc-агент объединения структур}
	\begin{scnindent}
		\scnidtf{Агент, создающий структуру содержащую все элементы сливаемых структур}
	\end{scnindent}
	\scnitem{Абстрактный sc-агент применения предложений по устранению противоречий}
	\scnitem{Абстрактный sc-агент внесения исправлений в базу знаний}
	\begin{scnindent}
		\scntext{примечание}{Внесение изменений подразумевает не только исправление в базе знаний изначальной проблемной структуры, но и фиксацию самого факта изменения состояния базы знаний}
	\end{scnindent}
	\scnitem{Неатомарный абстрактный sc-агент верификации структуры}
	\begin{scnindent}
		\scntext{примечание}{Агент, обеспечивающий полный цикл верификации структуры и координирующий другие агенты}
	\end{scnindent}
\end{scnrelfromset}

\bigskip
\end{scnsubstruct}
\end{SCn}


% Объединить с предыдущим
%\scsubsubsection[
%    \protect\scneditor{Бутрин С.В.}
%    \protect\scnmonographychapter{Глава 5.2. Методика и средства проектирования и анализа качества баз знаний интеллектуальных компьютерных систем нового поколения}
%    ]{Логико-семантическая модель ostis-системы обнаружения и анализа информационных дыр в базе знаний ostis-системы}
%\label{detec_hole_logical_model}


% Коллективная разработка
\scsubsubsection[
    \protect\scneditors{Бутрин С.В.;Банцевич К.А.}
    \protect\scnmonographychapter{Глава 5.2. Методика и средства проектирования и анализа качества баз знаний интеллектуальных компьютерных систем нового поколения}
    ]{Предметная область и онтология взаимодействия разработчиков различных категорий в процессе проектирования базы знаний ostis-системы}
\label{sd_author}
\begin{SCn}
	\scnsectionheader{Предметная область и онтология взаимодействия разработчиков различных категорий в процессе проектирования базы знаний ostis-системы}
	\begin{scnsubstruct}
		
\scnheader{модель деятельности, направленной на создание \textit{гибридных баз знаний} коллективом разработчиков}

\scntext{пояснение}{данная модель базируется на модели деятельности различных субъектов и реализована в виде онтологии предметной области деятельности разработчиков, направленной на разработку и модификацию гибридных баз знаний.}

\scnnote{процесс создания и редактирования \textit{базы знаний} \textit{ostis-системы} сводится к формированию разработчиками предложений по редактированию того или иного раздела \textit{базы знаний} и последующему рассмотрению этих предложений администраторами \textit{базы знаний}.}

\scnnote{предполагается, что в случае необходимости для верификации поступающих предложений по редактированию базы знаний могут привлекаться эксперты, а управление процессом разработки осуществляется менеджерами соответствующих проектов по разработке базы знаний.
При этом формирование проектных заданий и их спецификация осуществляются также при помощи механизма предложений по редактированию соответствующего раздела базы знаний.}

\scnnote{вся информация, связанная с текущими процессами разработки базы знаний, историей и планами ее развития, хранится в той же базе знаний, что и ее предметная часть, то есть часть базы знаний, доступная конечному пользователю системы. Такой подход обеспечивает широкие возможности автоматизации процесса создания баз знаний, а также последующего анализа и совершенствования базы знаний.}

\scnnote{каждое предложение по редактированию базы знаний представляет собой структуру, содержащую sc-текст, который предлагается включить в состав согласованной части базы знаний. В состав таких предложений могут входить знаки действий по редактированию базы знаний, которые автоматически инициируются и выполняются соответствующими агентами после утверждения предложения.}


\scnheader{пользователь базы знаний ostis-системы*}
\scnidtf{бинарное отношение, связывающее sc-модель базы знаний ostis-системы и sc-элемент, обозначающий персону, участвующую в разработке или эксплуатации этой базы знаний}
\scniselement{бинарное отношение}
\scniselement{ориентированное отношение}

\scnrelfrom{разбиение}{\scnkeyword{Типология отношений между базами знаний ostis-систем и их пользователями по наличию факта прохождения регистрации в этих ostis-системах\scnsupergroupsign}}
\begin{scnindent}
	\begin{scneqtoset}
		\scnitem{зарегистрированный пользователь*}
		\begin{scnindent}
			\scntext{пояснение}{\textit{зарегистрированный пользователь} имеет доступ на чтение всей базы знаний и внесение предложений ко всей базе знаний, может выполнять роль конечного пользователя ostis-системы, то есть работать в режиме эксплуатации, а также роль ее разработчика. }
		\end{scnindent}
		\scnitem{незарегистрированный пользователь*}
	\end{scneqtoset}
\end{scnindent}

\scnnote{Данное отношение отражает связь \textit{пользователя} и \textit{базы знаний} в целом, при этом тот же самый пользователь может быть связан другими более частными отношениями с какими-либо фрагментами этой же \textit{базы знаний}.}

\scnnote{Независимо от роли, которую выполняет тот или иной \textit{пользователь}, он может делать предложения по редактированию любой из частей базы знаний, которые в зависимости от его уровня будут либо приняты автоматически, либо будут отдельно рассматриваться.}

\scnheader{пользователь, обладающий правом просмотра sc-структуры базы знаний ostis-систем*}
\scnidtf{бинарное отношение, связывающее sc-элемент, обозначающий sc-структуру (например, фрагмент sc-модели базы знаний), и sc-элемент, обозначающий пользователя этой ostis-системы, который обладает правом просмотра этой sc-структуры.}
\scniselement{бинарное отношение}
\scniselement{ориентированное отношение}

\scnnote{\textbf{\textit{пользователь, обладающий правом просмотра sc-структуры базы знаний ostis-системы*}} может быть зарегистрирован или не зарегистрирован в \textit{sc-модели базы знаний}.}

	\scnheader{пользователь, обладающий правом редактирования sc-структуры базы знаний ostis-систем*}
	\scnidtf{бинарное отношение, связывающее sc-элемент, обозначающий sc-структуру (например, фрагмент sc-модели базы знаний), и sc-элемент, обозначающий зарегистрированного пользователя ostis-системы, который обладает правом редактирования этой sc-структуры.}
	\scniselement{бинарное отношение}
	\scniselement{ориентированное отношение}
	\scnsuperset{пользователь, обладающий правом просмотра sc-структуры*}
	\begin{scnrelfromset}{покрытие}
		\scnitem{пользователь, обладающий правом редактирования sc-структуры посредством формирования предложений по внесению изменений в согласованную часть базы знаний этой ostis-системы*}
		\scnitem{пользователь, обладающий правом редактирования sc-структуры с автоматическим формированием и принятием предложений по внесению изменений в согласованную часть базы знаний этой ostis-системы*}
	\end{scnrelfromset}

\scnnote{Связки отношения \textit{пользователя, обладающего правом редактирования sc-структуры ostis-системы*} связывают sc-структуру (не обязательно всю sc-модель базы знаний) и пользователя, зарегистрированного в этой sc-модели базы знаний.}

	\scnheader{разработчик*}
	\scnsubset{пользователь, обладающий правом редактирования sc-структуры*}
	\scnidtf{бинарное отношение, связывающее sc-элемент, обозначающий некоторый раздел базы знаний (в пределе --- всю базу знаний), и sc-элемент, обозначающий пользователя ostis-системы, который может быть разработчиком данного раздела базы знаний, то есть выполнять проектные задачи в рамках данного раздела}
	
	\scnrelfrom{разбиение}{\scnkeyword{Типология разработчиков баз знаний ostis-систем\scnsupergroupsign}}
	\begin{scnindent}
		\begin{scneqtoset}
			\scnitem{администратор*}
			\scnitem{менеджер*}
			\scnitem{эксперт*}
		\end{scneqtoset}
	\end{scnindent}
	
	\scnheader{администратор*}
	\scnidtf{бинарное отношение, связывающее sc-элемент, обозначающий некоторый раздел базы знаний (в пределе --- всю базы знаний), и sc-элемент, обозначающий пользователя ostis-системы, который является администратором данного раздела базы знаний}
	\begin{scnrelfromset}{функции}
		\scnfileitem{контроль целостности и непротиворечивости всей базы знаний}
		\scnfileitem{определение уровней доступа других пользователей}
		\scnfileitem{принятие решения относительно принятия или отклонения предложений в различные части базы знаний, в том числе при необходимости отправка их на экспертизу}
		\scnfileitem{самостоятельное внесение изменений в различные части базы знаний путем использования соответствующих команд редактирования (при этом изменения автоматически оформляются как предложения и заносятся в раздел истории развития ostis-системы)}
	\end{scnrelfromset}
	
	
	\scnheader{менеджер*}
	\scnidtf{бинарное отношение, связывающее sc-элемент, обозначающий некоторый раздел базы знаний (в пределе --- всю базу знаний), и sc-элемент, обозначающий персону, которая является менеджером данного раздела базы знаний}
	\begin{scnrelfromset}{функции}
		\scnfileitem{планирование объемов работ по разработке базы знаний}
		\scnfileitem{детализация проектных задач на подзадачи, непосредственно формулирование проектных задач, назначение исполнителей проектных задач}
		\scnfileitem{установка приоритетов и сроков выполнения задач}
		\scnfileitem{контроль сроков выполнения проектных задач}
	\end{scnrelfromset}
	
	
	\scnheader{эксперт*}
	\scnidtf{бинарное отношение, связывающее sc-элемент, обозначающий какой-либо проект по разработке раздела базы знаний ostis-системы (в общем случае --- всей базы знаний), и sc-элемент, обозначающий персону, которая является экспертом данного раздела базы знаний}
	\begin{scnrelfromset}{функции}
		\scnfileitem{верификация результатов выполнения проектных задач}
		\scnfileitem{при необходимости эксперт может оставлять комментарии к любому фрагменту базы знаний относительно его корректности. Все комментарии попадают в раздел, описывающий план развития компьютерной системы}
	\end{scnrelfromset}
	

\scnnote{При необходимости разработки объемной \textit{базы знаний} может вводиться иерархия разработчиков, соответствующая иерархии разделов разрабатываемой \textit{базы знаний}.

В этом случае утверждение какого-либо предложения администратором раздела нижнего уровня не приводит к интеграции предложения в соответствующий раздел, а требует рассмотрения администраторами более высокого уровня. Окончательное решение принимается администратором всей базы знаний.}

\scnnote{любой участник процесса разработки имеет возможность оставить естественно-языковой комментарий к любому фрагменту или элементу базы знаний, таким образом, может осуществляться обсуждение каких-либо вопросов, связанных с указанным фрагментом или элементом базы знаний. Такого рода комментарии попадают в раздел базы знаний текущие процессы развития компьютерной системы.}

    \bigskip
\end{scnsubstruct}
\end{SCn}


%-------------------------

\scsubsection[
    \protect\scneditor{Шункевич Д.В.}
    \protect\scnmonographychapter{Глава 5.3. Методика и средства компонентного проектирования решателей задач интеллектуальных компьютерных систем нового поколения}
    ]{Логико-семантическая модель комплекса ostis-систем автоматизации проектирования решателей задач ostis-систем}
\label{problem_solver_design_tools}

\scsubsubsection[
    \protect\scneditor{Шункевич Д.В.}
    \protect\scnmonographychapter{Глава 5.3. Методика и средства компонентного проектирования решателей задач интеллектуальных компьютерных систем нового поколения}
    ]{Логико-семантическая модель ostis-системы автоматизации проектирования программ Базового языка программирования ostis-систем}
\label{programs_design_tools}

\scsubsubsection[
    \protect\scneditor{Шункевич Д.В.}
    \protect\scnmonographychapter{Глава 5.3. Методика и средства компонентного проектирования решателей задач интеллектуальных компьютерных систем нового поколения}
    ]{Логико-семантическая модель ostis-системы автоматизации проектирования внутренних агентов ostis-систем, а также коллективов таких агентов}
\label{agents_design_tools}

\scsubsubsection[
    \protect\scneditors{Ковалев М.В.;Крощенко А.А.;Михно Е.В.}
    \protect\scnmonographychapter{Глава 3.6. Конвергенция и интеграция искусственных нейронных сетей с базами знаний в интеллектуальных компьютерных системах нового поколения}
    ]{Логико-семантическая модель ostis-системы автоматизации проектирования искусственных нейронных сетей, семантически совместимых с базами знаний ostis-систем}
\label{autom_design_neural_network_logical_model}

\scsubsection[
    \protect\scneditors{Садовский М.Е.;Жмырко А.В.}
    \protect\scnmonographychapter{Глава 5.4. Методика и средства компонентного проектирования интерфейсов интеллектуальных компьютерных систем нового поколения}
    ]{Логико-семантическая модель ostis-системы автоматизации проектирования интерфейсов ostis-систем}
\label{autom_design_interface_logical_model}

\scsubsection[
    \protect\scnmonographychapter{Глава 7.2. Экосистема интеллектуальных компьютерных систем нового поколения (Экосистема OSTIS) и реализация рынка знаний на ее основе}
    ]{Предметная область и онтология ostis-систем автоматизации проектирования различных классов ostis-систем}
\label{sd_autom_design_class}

\scsection{Предметная область и онтология ostis-систем обучения проектированию ostis-систем и их компонентов}
\label{sd_learn_design_component}