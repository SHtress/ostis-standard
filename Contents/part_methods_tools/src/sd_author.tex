\begin{SCn}
	\scnsectionheader{Предметная область и онтология взаимодействия разработчиков различных категорий в процессе проектирования базы знаний ostis-системы}
	\begin{scnsubstruct}
		
\scnheader{модель деятельности, направленной на создание \textit{гибридных баз знаний} коллективом разработчиков}

\scntext{пояснение}{данная модель базируется на модели деятельности различных субъектов и реализована в виде онтологии предметной области деятельности разработчиков, направленной на разработку и модификацию гибридных баз знаний.}

\scnnote{процесс создания и редактирования \textit{базы знаний} \textit{ostis-системы} сводится к формированию разработчиками предложений по редактированию того или иного раздела \textit{базы знаний} и последующему рассмотрению этих предложений администраторами \textit{базы знаний}.}

\scnnote{предполагается, что в случае необходимости для верификации поступающих предложений по редактированию базы знаний могут привлекаться эксперты, а управление процессом разработки осуществляется менеджерами соответствующих проектов по разработке базы знаний.
При этом формирование проектных заданий и их спецификация осуществляются также при помощи механизма предложений по редактированию соответствующего раздела базы знаний.}

\scnnote{вся информация, связанная с текущими процессами разработки базы знаний, историей и планами ее развития, хранится в той же базе знаний, что и ее предметная часть, то есть часть базы знаний, доступная конечному пользователю системы. Такой подход обеспечивает широкие возможности автоматизации процесса создания баз знаний, а также последующего анализа и совершенствования базы знаний.}

\scnnote{каждое предложение по редактированию базы знаний представляет собой структуру, содержащую sc-текст, который предлагается включить в состав согласованной части базы знаний. В состав таких предложений могут входить знаки действий по редактированию базы знаний, которые автоматически инициируются и выполняются соответствующими агентами после утверждения предложения.}


\scnheader{пользователь базы знаний ostis-системы*}
\scnidtf{бинарное отношение, связывающее sc-модель базы знаний ostis-системы и sc-элемент, обозначающий персону, участвующую в разработке или эксплуатации этой базы знаний}
\scniselement{бинарное отношение}
\scniselement{ориентированное отношение}

\scnrelfrom{разбиение}{\scnkeyword{Типология отношений между базами знаний ostis-систем и их пользователями по наличию факта прохождения регистрации в этих ostis-системах\scnsupergroupsign}}
\begin{scnindent}
	\begin{scneqtoset}
		\scnitem{зарегистрированный пользователь*}
		\begin{scnindent}
			\scntext{пояснение}{\textit{зарегистрированный пользователь} имеет доступ на чтение всей базы знаний и внесение предложений ко всей базе знаний, может выполнять роль конечного пользователя ostis-системы, то есть работать в режиме эксплуатации, а также роль ее разработчика. }
		\end{scnindent}
		\scnitem{незарегистрированный пользователь*}
	\end{scneqtoset}
\end{scnindent}

\scnnote{Данное отношение отражает связь \textit{пользователя} и \textit{базы знаний} в целом, при этом тот же самый пользователь может быть связан другими более частными отношениями с какими-либо фрагментами этой же \textit{базы знаний}.}

\scnnote{Независимо от роли, которую выполняет тот или иной \textit{пользователь}, он может делать предложения по редактированию любой из частей базы знаний, которые в зависимости от его уровня будут либо приняты автоматически, либо будут отдельно рассматриваться.}

\scnheader{пользователь, обладающий правом просмотра sc-структуры базы знаний ostis-систем*}
\scnidtf{бинарное отношение, связывающее sc-элемент, обозначающий sc-структуру (например, фрагмент sc-модели базы знаний), и sc-элемент, обозначающий пользователя этой ostis-системы, который обладает правом просмотра этой sc-структуры.}
\scniselement{бинарное отношение}
\scniselement{ориентированное отношение}

\scnnote{\textbf{\textit{пользователь, обладающий правом просмотра sc-структуры базы знаний ostis-системы*}} может быть зарегистрирован или не зарегистрирован в \textit{sc-модели базы знаний}.}

	\scnheader{пользователь, обладающий правом редактирования sc-структуры базы знаний ostis-систем*}
	\scnidtf{бинарное отношение, связывающее sc-элемент, обозначающий sc-структуру (например, фрагмент sc-модели базы знаний), и sc-элемент, обозначающий зарегистрированного пользователя ostis-системы, который обладает правом редактирования этой sc-структуры.}
	\scniselement{бинарное отношение}
	\scniselement{ориентированное отношение}
	\scnsuperset{пользователь, обладающий правом просмотра sc-структуры*}
	\begin{scnrelfromset}{покрытие}
		\scnitem{пользователь, обладающий правом редактирования sc-структуры посредством формирования предложений по внесению изменений в согласованную часть базы знаний этой ostis-системы*}
		\scnitem{пользователь, обладающий правом редактирования sc-структуры с автоматическим формированием и принятием предложений по внесению изменений в согласованную часть базы знаний этой ostis-системы*}
	\end{scnrelfromset}

\scnnote{Связки отношения \textit{пользователя, обладающего правом редактирования sc-структуры ostis-системы*} связывают sc-структуру (не обязательно всю sc-модель базы знаний) и пользователя, зарегистрированного в этой sc-модели базы знаний.}

	\scnheader{разработчик*}
	\scnsubset{пользователь, обладающий правом редактирования sc-структуры*}
	\scnidtf{бинарное отношение, связывающее sc-элемент, обозначающий некоторый раздел базы знаний (в пределе --- всю базу знаний), и sc-элемент, обозначающий пользователя ostis-системы, который может быть разработчиком данного раздела базы знаний, то есть выполнять проектные задачи в рамках данного раздела}
	
	\scnrelfrom{разбиение}{\scnkeyword{Типология разработчиков баз знаний ostis-систем\scnsupergroupsign}}
	\begin{scnindent}
		\begin{scneqtoset}
			\scnitem{администратор*}
			\scnitem{менеджер*}
			\scnitem{эксперт*}
		\end{scneqtoset}
	\end{scnindent}
	
	\scnheader{администратор*}
	\scnidtf{бинарное отношение, связывающее sc-элемент, обозначающий некоторый раздел базы знаний (в пределе --- всю базы знаний), и sc-элемент, обозначающий пользователя ostis-системы, который является администратором данного раздела базы знаний}
	\begin{scnrelfromset}{функции}
		\scnfileitem{контроль целостности и непротиворечивости всей базы знаний}
		\scnfileitem{определение уровней доступа других пользователей}
		\scnfileitem{принятие решения относительно принятия или отклонения предложений в различные части базы знаний, в том числе при необходимости отправка их на экспертизу}
		\scnfileitem{самостоятельное внесение изменений в различные части базы знаний путем использования соответствующих команд редактирования (при этом изменения автоматически оформляются как предложения и заносятся в раздел истории развития ostis-системы)}
	\end{scnrelfromset}
	
	
	\scnheader{менеджер*}
	\scnidtf{бинарное отношение, связывающее sc-элемент, обозначающий некоторый раздел базы знаний (в пределе --- всю базу знаний), и sc-элемент, обозначающий персону, которая является менеджером данного раздела базы знаний}
	\begin{scnrelfromset}{функции}
		\scnfileitem{планирование объемов работ по разработке базы знаний}
		\scnfileitem{детализация проектных задач на подзадачи, непосредственно формулирование проектных задач, назначение исполнителей проектных задач}
		\scnfileitem{установка приоритетов и сроков выполнения задач}
		\scnfileitem{контроль сроков выполнения проектных задач}
	\end{scnrelfromset}
	
	
	\scnheader{эксперт*}
	\scnidtf{бинарное отношение, связывающее sc-элемент, обозначающий какой-либо проект по разработке раздела базы знаний ostis-системы (в общем случае --- всей базы знаний), и sc-элемент, обозначающий персону, которая является экспертом данного раздела базы знаний}
	\begin{scnrelfromset}{функции}
		\scnfileitem{верификация результатов выполнения проектных задач}
		\scnfileitem{при необходимости эксперт может оставлять комментарии к любому фрагменту базы знаний относительно его корректности. Все комментарии попадают в раздел, описывающий план развития компьютерной системы}
	\end{scnrelfromset}
	

\scnnote{При необходимости разработки объемной \textit{базы знаний} может вводиться иерархия разработчиков, соответствующая иерархии разделов разрабатываемой \textit{базы знаний}.

В этом случае утверждение какого-либо предложения администратором раздела нижнего уровня не приводит к интеграции предложения в соответствующий раздел, а требует рассмотрения администраторами более высокого уровня. Окончательное решение принимается администратором всей базы знаний.}

\scnnote{любой участник процесса разработки имеет возможность оставить естественно-языковой комментарий к любому фрагменту или элементу базы знаний, таким образом, может осуществляться обсуждение каких-либо вопросов, связанных с указанным фрагментом или элементом базы знаний. Такого рода комментарии попадают в раздел базы знаний текущие процессы развития компьютерной системы.}

    \bigskip
\end{scnsubstruct}
\end{SCn}
