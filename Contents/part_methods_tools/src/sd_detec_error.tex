\begin{SCn}
\scnsectionheader{Предметная область и онтология обнаружения и анализа ошибок и противоречий в базе знаний ostis-системы}
\begin{scnsubstruct}

\scntext{анотация}{Важным этапом в разработке любой системы является контроль ее качества, так как именно на этом этапе определяется степень жизнеспособности и эффективности системы.}

\scnnote{Верификация является видом анализа качества и частью процесса разработки системы. Она заключается в проверке информации на правильность и полноту.
Ее целью является выявление ошибок, различных дефектов и недоработок для своевременного их устранения.

Существующие на данный момент методы верификации хорошо развиты и разработано большое количество различных моделей верификации, использующих расширенные таблицы решения, сети Петри, различные логики, например, логики с векторной семантикой (см. \scncite{Arshinskiy2020}) и другие модели. Более того формируются специализированные онтологии для описания многообразия средств и моделей верификации баз знаний (см. \scncite{Rybina2007}). Однако механизма взаимодействия средств, использующих данные методы, нет.}

\scnnote{Поэтому средства верификации баз знаний, существующие на данный момент, обладают рядом таких проблем как (см. \scncite{Zhang2023}):
\begin{itemize}
	\item зависимость от формата представления информации, из-за чего приходится тратить дополнительно время на конвертирование информации;
	\item проблема невозможности быть переиспользованными, так как средства обычно делаются с учетом особенностей конкретной системы;
	\item проблема отсутствия механизма взаимодействия средств верификации и анализа знаний;
	\item высокая роль человека в процессе верификации, так как самым распространенным методом верификации баз знаний является ручная проверка базы экспертом, человек выступает как администратор, принимающий единогласное решение, навязывая свое мнение системе;
	\item современные средства не учитывают и не рассматривают процесс верификации в рамках взаимодействия систем друг с другом.
\end{itemize}

Эти проблемы могли бы быть решены, если:
\begin{itemize}
	\item использовать унифицированный и удобный формат представления знаний;
	\item системы создавались бы по общей методологии и были бы совместимы друг с другом;
	\item продумать и реализовать механизм позволяющий системе стремиться самой принимать решение относительно своего состояния и наличия в нем проблемных моментов и ошибок, система может допускать ошибки и не всегда принимать верные решения, но это должны быть ее ошибки, а не экспертов и разработчиков.
\end{itemize}

Преимуществами \textit{Технологии OSTIS} в рамках задачи верификации являются:
\begin{itemize}
	\item наличие общей методологии проектирования интеллектуальных систем, позволяющая решить проблему совместимости систем при их коллективном взаимодействии;
	\item все знания представлены в унифицированном виде, что позволяет эффективно их обрабатывать, сводя затраты на конвертирование к минимуму;
	\item средства, с помощью которых производится выявление, анализ и устранение противоречий, описаны в самой базе знаний, а также их спецификация представлены в самой базе знаний системы, тем самым обеспечивая легкость их расширения и позволяя системе знать, каким инструментарием она обладает;
	\item отсутствие семантических эквивалентных фрагментов, что обеспечивает локальность вносимых исправлений и исключает необходимость вносить исправления многократно в разных местах;
	\item многоагентный подход, который позволяет рассматривать средства анализа и верификации баз знаний как коллектив агентов, способных взаимодействовать друг с другом и в дальнейшем принимать общее решение касательно состояния базы знаний. 
\end{itemize}

Предлагаемый подход сводится к разработке:
\begin{itemize}
	\item специализированной \textit{предметной области и онтологии}, которая бы содержала в себе все необходимые знания о возможных видах проблемных фрагментов базы знаний и методах их исправления;
	\item алгоритма, позволяющего системе выявить в себе проблемные фрагменты и устранить их, при этом обеспечив согласованность работы средств самой системы;
	\item \textit{специализированного решателя задач}, содержащего необходимые агенты для выявления и устранения проблемных фрагментов.
\end{itemize}
}
\scnnote{Качество базы знаний во многом определяется уровнем наличия/отсутствия \textit{не-факторов} (см. \scncite{Narinjani2004}) в \textit{базе знаний}.}


\scnheader{не-фактор}
\scnidtf{группа семантических свойств, определяющих качество информации, хранимой в памяти кибернетической системы}
\begin{scneqtoset}
	\scnitem{корректность/некорректность информации, хранимой в памяти кибернетической системы}
	\scnitem{однозначность/неоднозначность информации, хранимой в памяти кибернетической системы}
	\scnitem{целостность/нецелостность информации, хранимой в памяти кибернетической системы}
	\scnitem{чистота/загрязненность информации, хранимой в памяти кибернетической системы}
	\scnitem{достоверность/недостоверность информации, хранимой в памяти кибернетической системы}
	\scnitem{точность/неточность информации, хранимой в памяти кибернетической системы}
	\scnitem{четкость/нечеткость информации, хранимой в памяти кибернетической системы}
	\scnitem{определенность/недоопределенность информации, хранимой в памяти кибернетической системы}
\end{scneqtoset}



\scnheader{проблемная структура}
\scnidtf{структура, описывающая проблемный фрагмент базы знаний}
\scnidtf{структура, описывающая некачественный фрагмент базы знаний}
\begin{scnreltoset}{объединение}
	\scnitem{некорректная структура}
	\begin{scnindent}
		\scnidtf{структура, содержащая фрагменты, противоречащие каким-либо правилам или закономерностям описанным в базе знаний}
	\end{scnindent}
	\scnitem{структура, описывающая неполноту в базе знаний}
	\begin{scnindent}
		\scnidtf{структура, в которой имеется неполнота (то есть имеется некоторое количество информационных дыр)}
		\scntext{примечание}{Под структурой, описывающей неполноту в базе знаний, понимается структура, содержащая фрагмент базы знаний, в котором отсутствует какая-либо информация, которая необходима (или, по крайней мере, желательна) для однозначного и полного понимания смысла данного фрагмента.}
		%% Переформулировать
	\end{scnindent}
	\scnitem{информационный мусор}
	\begin{scnindent}
		\scnidtf{структура, удаление которой существенно не усложнит деятельность системы}
		\scnidtf{структура, содержащая фрагмент базы знаний, который по каким-либо причинам стал ненужным и требует удаления}
	\end{scnindent}
\end{scnreltoset}

\scnheader{противоречие*}
\scnidtf{пара противоречащих друг другу фрагментов информации, хранимой в памяти кибернетической системы*}
\scntext{примечание}{Чаще всего противоречащими друг другу информационными фрагментами являются:
	\begin{scnitemize}
		\item явно представленная в памяти некоторая закономерность (некоторое правило);
		\item информационный фрагмент, не соответствующий (противоречащий) указанной закономерности.
	\end{scnitemize}
	В этом случае некорректность может присутствовать:
	\begin{scnitemize}
		\item либо в информационном фрагменте, который противоречит указанной закономерности;
		\item либо в самой этой закономерности;
		\item либо и там и там.
	\end{scnitemize}}


\scnheader{некорректная структура}
\begin{scnreltoset}{включение}
	\scnitem{дублирование системных идентификаторов}
	\scnitem{несоответствие элементов связки доменам отношения}
	\scnitem{цикл по отношению порядка}
	\scnitem{структура, противоречащая свойству единственности}
\end{scnreltoset}

\scnheader{структура, описывающая неполноту в базе знаний}
\begin{scnreltoset}{включение}
	\scnitem{не указан максимальный класс объектов исследования предметной области}
	\scnitem{для сущности указан системный, но не указаны основные идентификаторы для всех внешних языков}
	\scnitem{не указаны домены отношения}
	\scnitem{понятие не соотнесено ни с одной предметной областью}
\end{scnreltoset}

\scnheader{структура, требующее внимание разработчика}
\scnidtf{проблемная структура, для исправления которой требуется участие разработчика}
\scnheader{множество элементов, которые должны быть удалены для исправления структуры*}
\scnidtf{множество элементов, удаление которых из структуры позволяет устранить в ней противоречия}
\scnheader{множество элементов, которые должны быть добавлены для исправления структуры*}
\scnidtf{множество элементов, добавление которых в структуру позволяет устранить в ней противоречия}
\scnheader{структура, которую система не способна исправить сама}
\scnidtf{структура, в которой система не способна автоматически устранить противоречия}
	
	
\scnheader{следует отличать*}
\begin{scnhaselementset}
	\scnitem{структура, которую система не может решить сама}
	\begin{scnindent}
		\scntext{примечание}{Здесь структура, которую система не может решить сама, не может быть исправлена при взаимодействии с разработчиком и требует полного исправления от самого разработчика}
	\end{scnindent}
	\scnitem{требующее внимание разработчика}
	\begin{scnindent}
		\scntext{примечание}{Здесь структура, требующая внимания разработчика, может быть решена в процессе верификации, но потребуется участие разработчика}
	\end{scnindent}
\end{scnhaselementset}

	
\scnheader{Решатель задач средств выявления и устранения противоречий}
\begin{scnrelfromset}{декомпозиция абстрактного sc-агента}
	\scnitem{Неатомарный абстрактный sc-агент выявления противоречий}
	\begin{scnindent}
		\scnidtf{Множество агентов, обеспечивающих поиск и фиксирование противоречий в структуре}
	\end{scnindent}
	\scnitem{Неатомарный абстрактный sc-агент устранения противоречий}
	\begin{scnindent}
		\scnidtf{Множество агентов, создающих предложения по исправлению противоречий}
		\scntext{примечание}{Результатом работы таких агентов будут множества предлагаемых к удалению из структуры или добавлению в структуру элементов}
	\end{scnindent}
	\scnitem{Абстрактный sc-агент объединения структур}
	\begin{scnindent}
		\scnidtf{Агент, создающий структуру содержащую все элементы сливаемых структур}
	\end{scnindent}
	\scnitem{Абстрактный sc-агент применения предложений по устранению противоречий}
	\scnitem{Абстрактный sc-агент внесения исправлений в базу знаний}
	\begin{scnindent}
		\scntext{примечание}{Внесение изменений подразумевает не только исправление в базе знаний изначальной проблемной структуры, но и фиксацию самого факта изменения состояния базы знаний}
	\end{scnindent}
	\scnitem{Неатомарный абстрактный sc-агент верификации структуры}
	\begin{scnindent}
		\scntext{примечание}{Агент, обеспечивающий полный цикл верификации структуры и координирующий другие агенты}
	\end{scnindent}
\end{scnrelfromset}

\bigskip
\end{scnsubstruct}
\end{SCn}
