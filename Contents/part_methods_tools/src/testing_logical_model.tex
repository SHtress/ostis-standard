\begin{SCn}
	\scnsectionheader{Логико-семантическая модель ostis-системы тестирования и верификации}
	\begin{scnsubstruct}
		\scntext{аннотация}{}
		
		\scnheader{агент верификации работы агента}
		\scntext{пояснение}{агент служит для проверки работоспособности путем анализа результатов прохождения программой набора различных тестов и проверок}
		
		\scnheader{тесты ostis-систем}
		\scnrelfrom{разбиение}{Типология тестов ostis-систем по признаку решаемых задач}
		\begin{scnindent}
			\begin{scneqtoset}
				\scnitem{модульные тесты}
				\scnitem{интеграционные тесты}
				\scnitem{системные тесты}
			\end{scneqtoset}
		\end{scnindent}
		
		\scnheader{тестовый сценарий}
		\scntext{пояснение}{документ, в котором содержатся условия, шаги и другие параметры для проверки реализации тестируемой функции или её части}
			
		\begin{scnrelfromvector}{атрибуты}
			\scnfileitem{Предусловия, они используются, если предварительно систему нужно приводить к состоянию пригодному для проведения проверки; т.е. указываются либо действия, с помощью которых система оказывается в нужном состоянии, либо список условий, выполнение которых говорит о том, что система находится в нужном состоянии для основного теста}
			
			\scnfileitem{Шаги — cписок действий, переводящих систему из одного состояния в другое, для получения результата}
			
			\scnfileitem{Ожидаемый результат, на основании которого можно делать вывод о удовлетворении поставленным требованиям}
		\end{scnrelfromvector}
		
		\scnheader{агент тестирования scp-программы}	
		
		%% возможно объеденить с тестовым сценарием
		\scnheader{тест}
		\begin{scnreltoset}{включение}
			\scnitem{входные данные}
			\scnitem{выходные данные}
			\begin{scnreltoset}{включение}
				\scnitem{ожидаемые выходные данные}
				\scnitem{действительные выходные данные}
			\end{scnreltoset}
		\end{scnreltoset}
	
		\scnnote{различные уровни тестирования. Тестирование абстрактного агента - проверка системы на возможность решить данный класс задачи, проверка конкретной реализации агента, проверка программ используемых агентом.}
		
		
		%% Обобщить программы и агенты в одну тестируемую сущность
		\scnheader{Тестирование ostis-системы}
		\scnrelfrom{разбиение}{Типология тестов ostis-систем по объекту тестирования}
		\begin{scnindent}
			\begin{scneqtoset}
				\scnitem{тестирование абстрактного sc-агента}
				\scnitem{тестирование конкретной реализации абстрактного sc-агента}
				\scnitem{тестирование программы, используемой в sc-агенте}
			\end{scneqtoset}
		\end{scnindent}
		
		\scnnote{Отчет о результатах тестирования хранится в системе, так же как и записи о всех проблемах возникших при работе агента во время работы системы.}
		
		\scnheader{Решатель задач ostis-системы тестирования}
		\begin{scnrelfromset}{декомпозиция}
			\scnitem{Абстрактный sc-агент сравнения структур}
			\scnitem{Абстрактный sc-агент подготовки структур для тестов}
			\scnitem{Абстрактный sc-агента проверки соостветствию спецификации}
			\scnitem{Абстрактный sc-агент запуска набора тестов для sc-агента}
			\scnitem{Абстрактный sc-агент формирования отчета о результатах теста}
		\end{scnrelfromset}
		
		
%%		Описать алгоритм проведения тестирования
		\scnheader{Алгоритм тестирования }
		\begin{scnrelfromset}{этапы}
			\scnfileitem{Проверка спецификации агента}
			\scnfileitem{Нахождение наборов тестов}
			\begin{scnindent}
				\scnnote{Тесты находятся на основе общей спецификации описания тестов для агентов/программ}
			\end{scnindent}
			\scnfileitem{Подготовка тестовых данных}
			\begin{scnindent}
				\scnnote{Для проверки теста выделяется отдельная область базы знаний в рамках которой фиксируется работа программы/агенты. Все структуры используемые и полученные в рамках работы агента, кроме случаев работы в режиме детального логирования, удаляются, чтобы не загромождать базу знаний мусором.}
			\end{scnindent}
			\scnfileitem{Запуск теста}
			\scnfileitem{Ожидание результата работы}
			\begin{scnindent}
				\scnnote{Ожидается что будет сгенерирована структура, свидетельствующая о завершении работы агента либо будет истечен лимит времени на работу агента (лимит подбирается таким образом, чтобы сигнализировать о явно некорректной работе агента)}
			\end{scnindent}
			\scnfileitem{Проверка результата работы}
			\begin{scnindent}
				\scnnote{Производится проверка структуры действительного ответа струтуре ожидаемого, в случае несовпадения, фиксируются различия}
			\end{scnindent}
			\scnfileitem{Формирование отчета}
			\begin{scnindent}
				\scnnote{Отчет включает в себя статистику по прохождению тестов, отмечая проблемные места, если таковые имеются}
			\end{scnindent}
		\end{scnrelfromset}
		
		
	\end{scnsubstruct}
\end{SCn}