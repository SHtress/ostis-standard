\begin{SCn}
	\scnsectionheader{Логико-семантическая модель ostis-системы тестирования и верификации}
	\begin{scnsubstruct}
		\scntext{аннотация}{}
		
		\scnheader{агент верификации работы агента}
		\scntext{пояснение}{агент служит для проверки работоспособности путем анализа результатов прохождения программой набора различных тестов}
		
		\scnheader{тесты ostis-систем}
		\scnrelfrom{разбиение}{Типология тестов ostis-систем по признаку решаемых задач}
		\begin{scnindent}
			\begin{scneqtoset}
				\scnitem{модульные тесты}
				\scnitem{интеграционные тесты}
				\scnitem{системные тесты}
			\end{scneqtoset}
		\end{scnindent}
		
		\scnheader{тестовый сценарий}
		\scntext{пояснение}{документ, в котором содержатся условия, шаги и другие параметры для проверки реализации тестируемой функции или её части.
			
		Атрибуты тест кейса:
		\begin{scnitemize}
			\item Предусловия (PreConditions) используются, если предварительно систему нужно приводить к состоянию пригодному для проведения проверки; т.е. указываются либо действия, с помощью которых система оказывается в нужном состоянии, либо список условий, выполнение которых говорит о том, что система находится в нужном состоянии для основного теста.
			
			\item Шаги (Steps) — cписок действий, переводящих систему из одного состояния в другое, для получения результата.
			
			\item Ожидаемый результат (Expected result), на основании которого можно делать вывод о удовлетворении поставленным требованиям.
		\end{scnitemize}
		}
		
		\scnheader{агент тестирования программы}	
		
		
		\scnheader{тест}
		\begin{scnreltoset}{включение}
			\scnitem{входные данные}
			\scnitem{выходные данные}
			\begin{scnreltoset}{включение}
				\scnitem{ожидаемые выхдные данные}
				\scnitem{действительные выходные данные}
			\end{scnreltoset}
		\end{scnreltoset}
	
		\scnnote{различные уровни тестирования. Тестирование абстрактного агента - проверка системы на возможность решить данный класс задачи, проверка конкретной реализации агента, проверка программ используемых агентом.}
		
		\scnnote{отчет о результатах тестирования и возникших проблемах хранится в системе, так же как и записи о всех проблемах возникших во время работы системы.}
		
	\end{scnsubstruct}
\end{SCn}