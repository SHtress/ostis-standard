\begin{SCn}
	\scnsectionheader{Предметная область и онтология ostis-системы редактирования, сборки и ввода исходных текстов различных компонентов проектируемой базы знаний в память ostis-системы}
	\begin{scnsubstruct}


\scnnote{Для решения задачи индивудуального наполения базы знаний предлагается использовать специализированный инструментарий, который включает в себя различного рода редакторы и трансляторы.

Текущая реализация \textit{ostis-платформы} и решателя задач поддерживает работу с файлами исходных текстов базы знаний. Для создания таких файлов исходных текстов на \textit{SCs-коде} можно воспользоваться любым текстовым редактором.

Для создания файлов исходных текстов в \textit{SCg-коде} может быть использован редактор \textbf{\textit{KBE}} (Knowledge Base source Editor, см. \scncite{Knowledge-base-editor2022}). \textit{KBE} является приложением, которое направлено на помощь в создании и редактировании фрагментов баз знаний интеллектуальных систем, проектирование которых основано на \textit{Технологии OSTIS}. В основу данного редактора положен принцип визуализации данных, хранящихся в базе знаний, что намного упрощает процесс их редактирования и ускоряет процесс проектирования баз знаний.

\textit{пользовательский интерфейс} инструмента представляет собой главное окно, в котором пользователь может создавать вкладки. В каждой вкладке может происходить редактирование различных файлов исходных текстов баз знаний, представленных с помощью \textit{SCg-кода}.

В рамках главного окна имеется панель инструментов и меню приложения.}

\scnnote{Меню приложения представляет собой некоторый набор команд. Команды, которые отображаются в меню делятся на два типа:
\begin{itemize}
	\item команды, которые являются общими для всех вкладок. В частности к ним относятся команды сохранения, загрузки, помощи и так далее;
	\item команды, которые специфичны для активной вкладки. Зависят от типа активной вкладки.
\end{itemize}

На панель инструментов, как и в пользовательских интерфейсах большинства приложений, вынесены наиболее часто используемые команды:
\begin{itemize}
	\item Создать новый файл;
	\item Открыть файл;
	\item Сохранить;
	\item Сохранить как;
	\item Закрыть.
\end{itemize}
}

\scnnote{Основная идея, которая преследуется в данном редакторе SCg-кода --- это упрощение и ускорение процесса редактирования sc.g-текстов.

В процессе редактирования пользователю доступны различные режимы редактирования.

Всего выделено 4 режима:
\begin{itemize}
	\item \textbf{\textit{Режим выделения и создания узлов}}.
	В данном режиме пользователь может работать со всеми объектами выделяя и перемещая их, вызывая контекстное меню с командами.
	Отличительной особенностью данного режима является то, что в нем можно создавать sc.g-узлы;
	
	\item \textbf{\textit{Режим создания sc.g-дуг}}.
	Создание sc.g-дуги начинается с того, что пользователь указывает объект из которого она будет выходить, далее он может указать точки излома дуги, завершается создание указанием конечного объекта.
	В процессе создания пользователь может отменять последнее действие (указание начального объекта, точки излома);
	
	\item \textbf{\textit{Режим создания sc.g-шин}}.
	sc.g-шины используются для увеличения контактной площади узла, поэтому они могут создаваться лишь для sc.g-узлов. 
	Создание шины начинается с указания sc.g-узла, далее как и при создании sc.g-дуг указываются точки излома. 
	Как и при создании дуг пользователь может отменять последнее действие нажатием правой клавиши мыши;
	
	\item \textbf{\textit{Режим создания sc.g-контуров}}.
	Создание sc.g-контура начинается с указаний первой его точки. Далее, как и в случае с sc.g-дугами и sc.g-шинами, указываются точки.
	Стоит отметить, что все объекты, которые попадут внутрь созданного
	контура, будут добавлены в него автоматически.
	Как и при создании дуг и шин пользователь может отменять последнее действие.
\end{itemize}

Кроме перечисленных выше команд существует еще целый ряд команд редактирования:
\begin{itemize}
	\item Команда изменения основного текстового идентификатора элемента;
	\item Команда изменения типа элемента;
	\item Команда установки содержимого.
\end{itemize}

Полученные файлы исходных текстов в дальнейшем могут быть погружены в \textit{базу знаний }ostis-системы с помощью \textit{Реализации транслятора файлов исходных текстов \textit{базы знаний} в sc-память ostis-платформы}.
}

\begin{SCn}
	\scnheader{Реализация транслятора файлов исходных текстов базы знаний в sc-память ostis-платформы}
	\scnidtf{sc-builder}
	\scniselement{многократно используемый компонент ostis-систем, хранящийся в виде файлов исходных текстов}
	\scnrelfrom{используемый язык}{SCs-код}
	\begin{scnrelfromset}{зависимости компонента}
		\scnitem{Библиотека методов и структура данных C++ Standard Library}
	\end{scnrelfromset}
	\scnrelto{программный компонент}{Программный вариант реализации ostis-платформы}
\end{SCn}

\scnnote{\textit{Реализация транслятора файлов исходных текстов базы знаний в sc-память ostis-платформы} позволяет осуществить сборку \textit{базы знаний} из набора файлов исходных текстов, записанных в SCs-коде с ограничениями в бинарный формат, воспринимаемый \textit{Программной моделью sc-памяти} (см. \textit{\ref{sec_soft_platform_scin_code_example}~\nameref{sec_soft_platform_scin_code_example}}).
При этом возможна как сборка \scnqq{с нуля} (с уничтожением ранее созданного слепка памяти), так и аддитивная сборка, когда информация, содержащаяся в заданном множестве файлов, добавляется к уже имеющемуся слепку состояния памяти.
В текущей реализации сборщик осуществляет \scnqq{склеивание} (\scnqq{слияние}) sc-элементов, имеющих на уровне файлов исходных текстов одинаковые \textit{системные sc-идентификаторы}.

Кроме \textit{KBE} существует редактор текстов базы знаний, являющийся частью \textit{Реализации интерпретатора sc-моделей пользовательских интерфейсов}, обладающий схожим с \textit{KBE} функционалом, но при этом позволяющий редактировать базу знаний в режиме реального времени и без создания файлов исходных текстов базы знаний, именно им рекомендуется пользоваться для редактирования базы знаний.
}
    \bigskip
\end{scnsubstruct}
\end{SCn}
